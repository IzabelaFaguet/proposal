% template for a task
% each task should be added to exactly one workpackage
% in the workpackage task list
\begin{task}[
  title=Multi-device Real-time Collaboration,
  id=collaboration,
  lead=UPSUD,
  PM=20,
  wphases={0-48},
  partners={SRL,QS}
]
  % For applications such as real-time collaboration and others,
  % it can be beneficial.
  % We will explore different possible mechanisms for moving document state
  % to the server-side in JupyterLab.

Current practices, in research and industry, are often collaborative and would benefit tremendously from the ability to collectively edit notebooks. Unfortunately there is no ``Google doc'' equivalent to a Jupyter notebook that is widely available today. 
Moreover, we routinely use multiple devices to manipulate or present content, although it is often  difficult to juggle devices and apps to transfer content from one device to the next. 
Jupyter already recognizes the distributed nature of the digital environment by supporting remote kernels for computation. 
But the user interfaces are stuck in single devices. 
BOSSEE will embrace this multi-device world and facilitate the distribution and real-time collaborative editing of content across multiple devices for presentation, interaction and collaboration purposes.

We will create a real-time collaborative notebook or JupyterLab-like application by leveraging well-known synchronization techniques such as Operational Transformation (OT) or Conflict-free Replicated Data Types (CRDTs), with server-side hosting of the document state. 
We will build on our experience with Webstrates (\url{http://webstrates.net}), a web-based environment that supports real-time sharing of web content. The CodeStrates extension to Webstrates uses the layout of Jupyter notebooks, and we have created a proof-of-concept showing that it can work with Jupyter kernels. However CodeStrates, and there are interesting unresolved questions about code execution in such a shared environment: Should it be synchronized or not among participants? Should there be a single kernel or one per participant? etc.

We will also enable selective distribution, aggregation and control of content across devices. 
We have used Webstrates to distribute and synchronize content across multiple devices such as a tablet, a laptop and a large wall-sized display. Yet this does not cover all use cases. 
For example, in a meeting, the participants should each be able to run their own notebook and pick which content to share with the group on a large display. 
We will create an environment where a notebook can collect specific cells from another notebook, or where a JupyterLab widget aggregates data from a collection of widgets running on each user’s device. 
We also want to support remote interaction using one device to control another, e.g. a widget on a smartphone to control a parameter in a computation taking place in a particular notebook, whose result is shown on a large shared display.

  % \begin{compactitem}
  % \item ...
  %   (\localdelivref{deliv-id})
  % \end{compactitem}

\end{task}
