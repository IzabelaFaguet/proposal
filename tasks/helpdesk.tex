\begin{task}[
  title=Local Help Desk,
  id=helpdesk,
  lead=UPSUD,
  PM=4, % UPSUD=2PM,...
          % please update according to your site involvement,
  wphases={0-48}, % At Paris-Sud, and unless complemented by other
                  % funding, this will stop after 18 months
  partners={SRL,XFEL,QS,CDS,WTT,SIL,INSERM,EP}
]

  Dissemination events and tutorials are very effective tools for
  engaging scientists and giving them the desire to acquire new
  technologies and best practices. The next barrier to adoption comes
  when, back home, the scientists start using them on their daily
  problem. Having access to a local expert -- even for a small amount
  of time -- makes a huge difference, saving on the wasted time and
  frustration on the inevitable rough corners, and getting first hand
  advice and guidance in the rich landscape of available tools that
  could otherwise soon feel overwhelming.

  At several of our sites, our Research Software Engineers will
  dedicate some fraction of their time to deliver such help to the
  local community, experimenting with various formats: help desk hours
  where scientists can drop by to get help; regular meet-ups where
  scientists can reconvene to work on their problems or on-demand
  tutorials with expert supervision and mutual help; in-lab visits to
  the scientists for more in-depth discussions; etc.

  An explicit aim of this task is to foster the creation of
  sustainable Research Software Engineer groups within institutions to
  support their scientists.

  This will be the occasion for our Research Software Engineers to
  witness first hand how users adopt or struggle with the projects
  technologies and services, and escalate the hurdles and barriers to
  adoptions as well as success stories. The sites will keep in close
  contact to exchange on the effectiveness of the various formats, and
  the outcome will be reported on in deliverables
  \localdelivref{report1}, \localdelivref{report2}, \localdelivref{report3}.
\end{task}
