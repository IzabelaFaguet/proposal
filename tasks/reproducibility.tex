% template for a task
% each task should be added to exactly one workpackage
% in the workpackage task list
\begin{task}[title=Archiving software for reproducible workflows,
  id=reproducibility,
  lead=XXX,
  PM=1,
  wphases={0-48},
  partners={SRL,XFEL}
]
  By combining explanation with code and output, Jupyter notebooks are
  valuable tools for making scientific computing more reproducible.
  Reproducible research can inspire greater confidence in scientific results,
  and make it easier for future research to build on those results.
  However, the code in a notebook invariably relies on external code: libraries
  and programs which are not saved as part of the notebook.
  This task concerns ways to record the versions of these tools in use, and to
  make sure them available for practical reproduction of the computation.

  The task includes the following activities:
  \begin{compactitem}
  \item Facilitating reproducible creation and long-term archiving of container
    images. Container technologies, such as Docker, offer exciting possibilities
    for capturing a computational environment, but much of the development of
    these tools is focused on short-term operational uses, not long-term
    preservation. We will develop tools and guidelines for building containers
    for scientific computing purposes so that they remain useful in the longer
    term, building on existing technology such as repo2docker wherever possible.

    There is a trade-off between preserving binary container images,
    and preserving the source code and instructions to build a container.
    Preserving sources is more transparent, and makes it easier to modify the
    code to explore a result, but without special care, the instructions may not
    continue to work in the future, or may not build an equivalent container.
    We will explore both approaches, with a particular interest in how to
    make build instructions that can still work many years in the future.
    (\delivref{workpackage}{deliv-id})
  \end{compactitem}
  \begin{compactitem}
  \item ...
    (\delivref{workpackage}{deliv-id})
  \end{compactitem}

  This techology will be developed with a real scientific use case at European
  XFEL (see WP3).
\end{task}
