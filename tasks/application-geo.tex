\begin{task}[
  title=Demonstrator: Geosciences,
  id=geoscience,
  lead=UIO,
  PM=22,
  wphases={0-48},
  partners={EGI,QS,SRL,UPSUD}
]

% UPSud involvement: UPSud has a geoscience group (GEOPS) and will be
% interested in using the tools developed here. No formal PM.

The aim of this task (see page \pageref{sec:concept-demonstrators-geo}) is to build on the Jupyter ecosystem to create a standardized and shareable computing, data analysis and visualization framework for Geosciences. This task will focus on filling gaps that hinder open science and will include the following activities:

\emph{Visualization}

\begin{compactitem}
  \item Improvement upon existing mapping tools for specialized
    visualization of in-situ and model-generated data arizing in
    specific use cases (Land, river-runoff, ocean, ice, wave and
    atmosphere models, particle dispersion models, oil spill models,
    etc.).

  \item Improvements of the tooling for 3-D visualization of
    geographical datasets in the Jupyter notebook, for use cases such as
    displaying clouds, volcanic plumes, atmospheric rivers.
\end{compactitem}

\emph{Collaboration with Jupyter with specialized tools for earth sciences}

\begin{compactitem}
  \item adding the ability to interactively integrate information or corrections
    observed during field trips, correspdonding to specific geographical locations.

  \TODO{Concurrent editing links to real-time editing from the
    core WP2 - mention link?}

  \item adding the ability to deploy Jupyter-based applications together with
    the correspdonding execution environment, both in the form of a runnable
    notebook with \emph{Binder} or as a read-only yet interactive \emph{Voila}
    dashboard.
\end{compactitem}

This work will be carried out in two stages with first local development and deployment of \TheProject EOSC services for Geosciences 
(such as a BinderHub for Big data geosciences and \emph{voila} innovative interactive \emph{App}) and then deployment of these services on EOSC (\localdelivref{demonstrators}).
\end{task}
