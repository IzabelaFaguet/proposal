\begin{task}[
  title=Demonstrator: Geosciences,
  id=geoscience,
  lead=UIO,
  PM=22,
  wphases={0-48},
  partners={EGI,QS,SRL,UPSUD}
]

% UPSud involvement: UPSud has a geoscience group (GEOPS) and will be
% interested in using the tools developed here. No formal PM.

This task (see page \pageref{sec:concept-demonstrators-geo}) aims at
deploying a BinderHub for Big data geosciences and \emph{voila}
innovative interactive \emph{App}. Co-design will take place from the
beginning of \TheProject to tune the development of Jupyter ecosystem
components and fulfill the need of the geoscience community.

\textbf{Visualization}

\begin{compactitem}
  \item Improvement upon existing mapping tools for specialized
    visualization of in-situ and model-generated data arizing in
    specific use cases (Land, river-runoff, ocean, ice, wave and
    atmosphere models, particle dispersion models, oil spill models,
    etc.).

    This may take the form of additions to the \emph{ipyleaflet} and
    \emph{folium} extentions to JupyterLab, as well as the production of
    curated examples in the documentation of ipyleaflet addressing these
    specific use cases.

  \item Improvements of the tooling for 3-D visualization of
    geographical datasets in the Jupyter notebook, for use cases such as
    displaying volcanic plumes (injection of aerosols in the various
    atmospheric layers), the quasi biennial oscillation (inversion of
    the wind direction in the tropical stratosphere), atmospheric rivers
    (flowing column of condensed water vapour in the atmosphere) and
    also at smaller scales to visualize 3-D discrete particle simulations
    of sheared granular fault zones.
\end{compactitem}

\textbf{Collaboration with Jupyter with specialized tools for earth sciences}

\begin{compactitem}
  \item adding the ability to interactively integrate information or corrections
    observed during field trips, correspdonding to specific geographical locations.

  \TODO{Concurrent editing links to real-time editing from the
    core WP2 - mention link?}

  \item adding the ability to deploy Jupyter-based applications together with
    the correspdonding execution environment, both in the form of a runnable
    notebook with \emph{Binder} or as a read-only yet interactive \emph{Voila}
    dashboard.
\end{compactitem}

\textbf{Data processing tools}

\begin{compactitem}
  \item streamlining visualization of standard data formats such as \emph{NetCDF}
  with ipyleaflet, and ipyvolume.

  \item better support for \emph{geopandas} dataframes in Jupyter interactive
  visualization tools.
\end{compactitem}
\end{task}
