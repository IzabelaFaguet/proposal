\begin{task}[
  title=Geosciences application,
  id=geoscience,
  lead=UIO,
  PM=24,
  wphases={0-48},
  partners={QS,SRL,UPSUD}
]

% UPSud involvement: UPSud has a geoscience group (GEOPS) and will be
% interested in using the tools developed here. No formal PM.

% Scientific description

The amount of geospatial data from a variety of sources, including satellite observations, 4D simulations and in-situ observations, contributed by volunteers
or state agencies keeps increasing. In many disciplines, managing this large volume
has become a challenge, and the old approach of downloading datasets for a local
analysis has become intractable.

The heterogeneity of the tools used in different institutions to deal with
large geographical datasets makes it difficult for researchers to share the outcome
of their work in a reproducible fashion.

In this context, Jupyter is now emerging as a standard exploration tool for
geospatial analysis, climate science, geology and by data providers in these areas.

To mention a few,

\begin{itemize}
\item
   the \emph{PanGeo} platform \cite{Pangeo2018} (Funded by the NSF, NASA, and the
   Alfred P. Sloan Foundation) is built upon Jupyter, JupyterHub, Binder, and Dask.
\item
   the \emph{Joint Research Centre Earth Observation Data and Processing Platform}
   (JEODPP) \cite{Soille2018} relies on Jupyter, JupyterHub and ipyleaflet as
   its main user interface.
\item
   the \emph{Google Earth Engine} platform also offers a jupyter-based user
   interface allowing the visual exploration of the data with ipyleaflet
   \cite{GEEJupyterLeaflet2017}.
\end{itemize}

In these three cases, deferred processing is used to restrict computation to
the extent of the area displayed in the map viewer, which allowed these
platforms to scale up to petabytes of data. In all examples, interactive
visualization is a key feature of the platform. Beyond tile-based
2-D visualization, the ability to efficiently process and visualize vector
or 3-D  data is also becoming critical.

The BOSSEE team, which comprises the main authors of the technologies upon
which these platforms are built (Jupyter, JupyterHub, Binder, ipyleaflet),
together with the Department of Geosciences of the University of Oslo, are
in a unique position to bring these technologies together in the context of
EOSC.

This demonstrator will focus on tools for two transversal research projects

\begin{itemize}
\item LATICE (Land-Atmosphere Interactions in Cold Environments)
\url{https://www.mn.uio.no/geo/english/research/groups/latice/}
\item EarthFlows (Interface Dynamics in Geophysical Flows)
\url{https://www.mn.uio.no/geo/english/research/groups/earthflows/}
\end{itemize}

The work items for this demonstrator fall in two main categories:
visualization and geographical data processing tools.

\textbf{Visualization}

\begin{compactitem}
  \item Improvement upon existing mapping tools for specialized
    visualization of in-situ and model-generated data arizing in
    specific use cases (Land, river-runoff, ocean, ice, wave and
    atmosphere models, particle dispersion models, oil spill models,
    etc.).

    This may take the form of additions to the \emph{ipyleaflet} and
    \emph{folium} extentions to JupyterLab, as well as the production of
    curated examples in the documentation of ipyleaflet addressing these
    specific use cases.

  \item Improvements of the tooling for 3-D visualization of
    geographical datasets in the Jupyter notebook, for use cases such as
    displaying volcanic plumes (injection of aerosols in the various
    atmospheric layers), the quasi biennial oscillation (inversion of
    the wind direction in the tropical stratosphere), atmospheric rivers
    (flowing column of condensed water vapour in the atmosphere) and
    also at smaller scales to visualize 3-D discrete particle simulations
    of sheared granular fault zones.
\end{compactitem}

\textbf{Collaboration with Jupyter with specialized tools for earth sciences}

\begin{compactitem}
  \item adding the ability to interactively integrate information or corrections
    observed during field trips, correspdonding to specific geographical locations.

  \TODO{Concurrent editing links to real-time editing from the
    core WP2 - mention link?}

  \item adding the ability to deploy Jupyter-based applications together with
    the correspdonding execution environment, both in the form of a runnable
    notebook with \emph{Binder} or as a read-only yet interactive \emph{Voila}
    dashboard.
\end{compactitem}

\textbf{Data processing tools}

\begin{compactitem}
  \item streamlining visualization of standard data formats such as \emph{NetCDF}
  with ipyleaflet, and ipyvolume.

  \item better support for \emph{geopandas} dataframes in Jupyter interactive
  visualization tools.
\end{compactitem}

\textbf{Teaching geo-sciences with Jupyter}

Beyond their use in scientific research, these development will be used in
the class room for teaching master's students with best practices in open
science.

\TODO{The transversal research plays into the desired ``services that
  encourate collaborative interdisciplinary work'' that are desired
  by this call; this is good. Can you imagine that some of these
  facilities can be used via the EOSC? That would be a good
  addition. For example, one could use the BinderHub installation
  that we expect on the EOSC. }

\end{task}
