\begin{task}[
  title=Astro application,
  id=astro,
  lead=CDS,
  PM=18,
  wphases={0-48},
  partners={CDS}
]

  Context: CDS runs astronomical services accessed by world-wide astronomers. 
Its three main services (SIMBAD, VizieR and Aladin) add up to one million
queries per day.  They can be accessed through web interfaces, mainly for human
interaction, as well as through programmatic interfaces, including the
standardized protocols defined by Virtual Observatory project..
    
  We will develop a Jupyter-based framework to efficiently access, explore, 
visualize and analyze CDS data. 
 We will provide users with a set of customizable Jupyter notebooks targeted
to the completion of some typical visualization and analysis tasks. We will
in particular focus on the two following user stories:
    \begin{compactitem}
        \item analysis of catalogue data results. Tabular data is the typical
              output of SIMBAD and VizieR data.
         \item modular dashboard-like interface summarizing available data
               for a given astronomical object and enabling loading and
               analysis of those data.
    \end{compactitem}


  This task will rely on existing Python libraries to access CDS data
(astroquery.[cds/simbad/vizier/xmatch]) and to visualize them (GLUE, ipyvolume).
We will also improve existing Jupyter widgets (ipyaladin, interactive sky atlas
running in the notebook) and develop a new widget to offer a tree-like view
of available data.
    
  We will also develop Python libraries to allow integration and usage in notebook of
existing CDS infrastructure services, namely CDSLogin (which provides
authentication) and CDS MyData (remote storage space for tabular data).
This will allow the user to interact with one's personal storage space from
the notebook. It will also allow for advanced customisation.
  
  Initially, the notebooks generated will run locally on user machines.
Later on, we will run them on European Binder Service developed in WP5.

  Access to the notebooks will be provided as a one-click action option from
SIMBAD and VizieR results pages.
  Thus, providing with a one-click way of visualizing, filtering and analyzing
these potentially large tables will bridge the gap between access and analysis
of the data, with zero installation for the user.

    Deliverable: \localdelivref{application-astro}
\end{task}
