% template for a task
% each task should be added to exactly one workpackage
% in the workpackage task list
\begin{task}[
  title=JupyterHub / BinderHub convergence,
  id=jh-bh-conv,
  lead=EP,
  PM=12,
  wphases={0-48},
  partners={EP,WTT}]
  % Lead: Loic

  An institution – typically a university, a national lab, a transnational
  research infrastructure such as the European XFEL, or transnational 
  infrastructure provider like EGI – wishes to provide its members and 
  users with a Jupyter service.

  The service lets user spawn and access personal or collaborative virtual
  environments: namely a web interface to a light weight virtual machine,
  in which they can use Jupyter notebooks, run calculations, etc.

  To cater for a large variety of use cases in teaching and research,
  the main aim of the upcoming specifications is to make the service as
  versatile as possible. In particular, it should empower the users to 
  customize the service (available software stack, storage setup, ...),
  without a need for administrator intervention.

  JupyterHub already provides authentication, persistent storage and some
  default environments for its users. On the other hand, BinderHub offers
  the possibility to define more precisely what you need for your teaching
  or research environmment which makes it very flexible. But, unfortunetly,
  it's not possible yet to have authentication and persistent storage with
  BinderHub.

  The purpose of this task is to have the same services offered by JupyterHub
  (authentication, persistent storage, ...) with the flexibility of BinderHub
  (construction of your own environment for teaching or research).

  % Motivation: \TODO{reuse material from the blog post:
  % \url{https://opendreamkit.org/2018/03/15/jupyterhub-binder-convergence/}}
  % \TODO{reuse material from the hackmd notes with Tim:
  % \url{https://hackmd.io/0DLDCXcmRzC_dOwjEak3hg#}}
  The task includes the following activities:
  \begin{compactitem}
  \item Interoperability with more authentication standards (OIDC,
    ...)
  \item Credential management
  \item Customizable persistence (admin configuration, api, UI)
  \item Choice of container registry
  \item Support for authenticated git repo for repo2docker?
  \item Resource configuration
  \end{compactitem}

  \TODO{Deliverable: report on the JupyterHub/Binder convergence, M18?
    M36? Sooner is better, but only if really feasible.}
\end{task}
