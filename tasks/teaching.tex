\begin{task}[
  title=Teaching with Jupyter technology,
  id=teaching,
  lead=EP,
  PM=8, % EP: 4PM, UPSUD: 4PM
  wphases={0-48},
  partners={EP,UPSUD}
  ]

  The Mathematics department at Ecole polytechnique as well as ... at UPSUD..
  {\bf preciser} have started a reform of their various teaching offer based on
  Jupyter for two years and several courses of the Bachelor program, 2nd and 3rd
  year of Engineering school and Master program have already begun relying on a
  strong use of Jupyter notebooks / JupyterHub\footnote{MAP551 - 2nd yer course
  MAP411 - AMS X02 - Mooc INRIA preciser?} and this will continue with a strong
  support of the Dean of undergraduate studies and of graduate studies. Besides
  several software and research engineers in applied mathematics have been
  recruited and participate in this effort, as well as a administration engineer
  in order to help in terms of building an infrastructure dedicated to Jupyter
  with the support of the head of the Ecole polytechnique in order to diffuse
  the effort into various other departments (Physics, Mechanical Engineering,
  Biology...), where already some courses are starting based in Jupyter. The
  link with the computer science club of students of Ecole polytechnique (Binet
  Reeau) has also been created with the project and a community is emerging.

  This framework is ideal for the project for two reasons : 1- it offers both
  the test infrastructure and community, and some identified courses and
  Professor / Engineers / Students, who will test the newly designed tools
  (nbgrader / novel JupyterHub technology) and provide some feedback in order to
  improve them (this can be conducted within the timing of the project since the
  framework is already present), 2- It creates a reactive community of students
  and researchers within Ecole polytechnique and Paris-Sud, where the
  dissemination of the project tools and experience will be easy, which will
  organize Jupyter days on a regular basis, where the demonstrators of the
  project for tasks XXX will be presented and discussed. {\bf link with WP
  dissemination}

  The task includes the following activities
  \begin{compactitem}
  \item Reinforcement of the use of Jupyter technology in Courses at all the level of Maths and Data Science at Ecole polytechnique and Psud...,
  \item Test of the new developments and feed back to tasks ... (List of courses? Link Data science - JB?)
  \item Organization of a Jupyter day\footnote{In 2018 \url{http://www.cmap.polytechnique.fr/~massot/Personal_web_page_of_Marc_Massot/JupyterX.html}} every year with demonstration of Jupyter new tools
    (\localdelivref{deliv-id})
  \end{compactitem}
\end{task}
