\begin{task}[
  title=Teaching with Jupyter technology,
  id=teaching,
  lead=EP,
  PM=7, % EP: 3PM, UPSUD: 4PM
  wphases={0-48},
  partners={EP,UPSUD}
  ]

  As argued in Task~\ref{teaching-tools}, the Jupyter ecosystem
  provides versatile tools and infrastructure which have been widely
  adopted in higher education in the recent years. École Polytechnique
  and Université Paris Sud have been at the forefront of this movement
  (see below). In this task, we propose to further support the use of
  Jupyter in their curriculum, as a test bed for the tools,
  infrastructure and best practices developed in
  Task~\ref{teaching-tools}.

  \TODO{NT: wondering whether the development of best practice belongs
    to the other task or this one}

  The Bossee project framework is ideal for two reasons:
  \begin{enumerate}
    \item it offers both the test infrastructure and community, and some identified courses and
  Professor / Engineers / Students, who will test the newly designed tools
  (nbgrader / novel JupyterHub technology) and provide some feedback in order to
  improve them (this can be conducted within the timing of the project since the
  framework is already present);
    \item It creates a reactive community of students
  and researchers within Ecole polytechnique and Paris-Sud, where the
  dissemination of the project tools and experience will be easy, which will
  organize Jupyter days as in 2018\footnote{\url{http://www.cmap.polytechnique.fr/~massot/Personal_web_page_of_Marc_Massot/JupyterX.html}} on a regular basis, where the demonstrators of the
  project for tasks xxx will be presented and discussed. {\bf link with WP
    dissemination}
  \end{enumerate}

  \TODO{HF: Loic, can you complete this, please?}

  A variety of courses are delivered at Université Paris Sud using
  Jupyter technologies. This includes for example programming classes
  in C++ at lower undergraduate level (400 students per year since
  2017), a series of undergraduate and graduate math courses (computer
  aided mathematics, computer algebra, numerical methods), or courses
  in physics, bio-informatics, etc. To support these courses, a
  JupyterHub service has been deployed in 2017 and progressively
  improved since, on Paris Sud's local cloud infrastructure, enabling
  students and teachers to work from anywhere and any device.

  The Mathematics department at Ecole polytechnique has started a reform of their various teaching offer based on
  Jupyter for two years and several courses of the Bachelor program, 2nd and 3rd
  year of Engineering school and Master program have already begun relying on a
  strong use of Jupyter notebooks / JupyterHub\footnote{MAP551 - 2nd yer course
  MAP411 - AMS X02 - Mooc INRIA preciser?} and this will continue with a strong
  support of the Dean of undergraduate studies and of graduate studies. Besides
  several software and research engineers in applied mathematics have been
  recruited and participate in this effort, as well as a administration engineer
  in order to help in terms of building an infrastructure dedicated to Jupyter
  with the support of the head of the Ecole polytechnique in order to disseminate
  the effort into various other departments (Physics, Mechanical Engineering,
  Biology...), where already some courses are starting based in Jupyter. The
  link with the computer science club of students of Ecole polytechnique (Binet
  R\'eseau) has also been created with the project and a community is emerging.

  The task includes the following activities
  \begin{compactitem}
  \item Reinforce the use of Jupyter technology in Courses at
    all levels, notably in Mathematics and Data Science, in close
    collaboration between Ecole polytechnique and Université Paris Sud;
  \item Test the new developments and feed back to tasks ... (List of courses? Link Data science - JB?)
  \item Organize a Jupyter event every year with demonstration of Jupyter new tools
    (\localdelivref{deliv-id})
  \item Foster sharing of experience, best practices and course
    material, at the local level, and then worldwide, through meetups,
    blogs, etc.
  \end{compactitem}
  The outcome of this task will be reported on in \delivref{}{teaching}.
\end{task}
