\begin{task}[
  title=Teaching with Jupyter technology,
  id=teaching,
  lead=EP,
  PM=4, % EP: 3PM, UPSUD: 1PM, EuXFEL 1 PM
  wphases={0-48},
  partners={UPSUD,XFEL}
  ]

  The Jupyter ecosystem provides versatile tools and infrastructure which have
  been widely adopted in higher education in the recent years (data8 ??). École
  Polytechnique and Université Paris Sud have been at the forefront of this
  movement (see the description of \site{EP} and \site{UPSUD} for more
  information) since many years. 

  Nevertheless, the implementation of such infrastructure requires a certain
  expertise in different fields (DevOps, software development,...) and is
  therefore not accessible to the greatest number of people.

  As part of the Bossee project, we are trying to remove these locks so that
  anyone can easily use the tools developed for their teaching. This task will
  therefore serve as a demonstrator to show Jupyter's full potential in a
  teaching environment. Therefore, we will put together several tasks and use them for courses given at \'Ecole polytechnique and Université Paris Sud.

  \begin{itemize}
    \item Tasks \taskref{core}{jh-bh-conv} and \taskref{eosc}{jh-bh-deployment} will provide the needed infrastructure to work in the best conditions for teachers and students. It will also permit to teachers to create their own teaching environment.
    \item Task \taskref{ecosystem}{teaching-tools} will provide the needed tools for teachers to share material with their students, give them feedback on their progress,...
    \item Finally, \taskref{ecosystem}{task xeus-cpp} will provide a new way to learn C++ language in scientific computing. It can be difficult to a student to learn all the process to make working a C++ program (syntax, compilation, link, output,...). Xeus will make it possible to avoid all these new knowledges and focus only on the syntax.   
  \end{itemize}

  Another important thing to notice is that \'Ecole polytechnique and Université Paris Sud
  \begin{enumerate}
    \item will offer both the test infrastructure and community, and some identified courses and
  Professor / Engineers / Students, who will test the newly designed tools and provide some feedback in order to
  improve them;
    \item will create a reactive community of students
  and researchers, where the
  dissemination of the project tools and experience will be easy, which will
  organize Jupyter days as in 2018\footnote{\url{http://www.cmap.polytechnique.fr/~massot/Personal_web_page_of_Marc_Massot/JupyterX.html}} on a regular basis.
  \end{enumerate}

  % \TODO{HF: Loic, can you complete this, please?}

  % A variety of courses are delivered at Université Paris Sud using
  % Jupyter technologies. This includes for example programming classes
  % in C++ at lower undergraduate level (400 students per year since
  % 2017), a series of undergraduate and graduate math courses (computer
  % aided mathematics, computer algebra, numerical methods), or courses
  % in physics, bio-informatics, etc. To support these courses, a
  % JupyterHub service has been deployed in 2017 and progressively
  % improved since, on Paris Sud's local cloud infrastructure, enabling
  % students and teachers to work from anywhere and any device.

  % The Mathematics department at Ecole polytechnique has started a reform of their various teaching offer based on
  % Jupyter for two years and several courses of the Bachelor program, 2nd and 3rd
  % year of Engineering school and Master program have already begun relying on a
  % strong use of Jupyter notebooks / JupyterHub\footnote{MAP551 - 2nd yer course
  % MAP411 - AMS X02 - Mooc INRIA preciser?} and this will continue with a strong
  % support of the Dean of undergraduate studies and of graduate studies. Besides
  % several software and research engineers in applied mathematics have been
  % recruited and participate in this effort, as well as a administration engineer
  % in order to help in terms of building an infrastructure dedicated to Jupyter
  % with the support of the head of the Ecole polytechnique in order to disseminate
  % the effort into various other departments (Physics, Mechanical Engineering,
  % Biology...), where already some courses are starting based in Jupyter. The
  % link with the computer science club of students of Ecole polytechnique (Binet
  % R\'eseau) has also been created with the project and a community is emerging.

  The task includes the following activities
  \begin{compactitem}
  \item Reinforce the use of Jupyter technology in Courses at
    all levels, notably in Mathematics and Data Science, in close
    collaboration between Ecole polytechnique and Université Paris Sud;
  \item Test the new developments and feed back to tasks \taskref{core}{jh-bh-conv} and \taskref{ecosystem}{teaching-tools}
  \item Organize a Jupyter event every year with demonstration of Jupyter new tools for teaching and research (\localdelivref{deliv-id})
  \item Foster sharing of experience, best practices and course
    material, at the local level, and then worldwide, through meetups,
    blogs, etc.
  \end{compactitem}
  The outcome of this task will be reported on in \delivref{}{teaching}.
\end{task}
