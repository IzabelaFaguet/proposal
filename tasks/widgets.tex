\begin{task}[
  title=Jupyter Interactive Widgets,
  id=jupyter-widgets,
  lead=QS,
  PM=40,
  wphases={0-48},
  partners={XFEL,SIL}
]

The task includes the following activities:

\begin{compactenum}
\item
  Improvements to the core Jupyter Widgets package

  \begin{compactenum}
  \item
    Improve the testing tools for Widget libraries, including means to
    test messaging, schemas, and actual rendering of widgets with a headless
    browser.
  \item
    Modernize of the \emph{@jupyter-widgets\/base} JavaScript package:
    drop \emph{backbone.js} and adopt a more modern MVC framework, support
    streaming of widgets messages to multiple frontends for future support of
    live collaboration.
  \item
    Iterate on the core \emph{@jupyter-widgets\/controls} package. This may
    involve the adoption of a modern framework such as \emph{React.js} for
    the widget view implementation, rather than the current custom implementation.
  \item
    Create new controls for the core Jupyter widgets package such as token
    inputs, typeahead, and tree views. These common controls tend to be implemented
    in several downstream packages, which causes unnecessary duplication of work
    and harms sustainability.
  \end{compactenum}

\item
  Dealing with large widget state

  \begin{compactenum}
  \item Create a generic mechanism for widgets to refer to external data services
  or companion files rather than storing their state entirely in the notebook format
  for offline view.
  \end{compactenum}

\item
  Simplify the authoring of complex GUIs with Jupyter widgets.

  \begin{compactenum}
  \item
    Provide pre-defined Jupyter widget layouts based on the recently
    introduced CSS Grid Layout, with named areas to which widgets can be
    assigned, such as a centra area, left and right tab bars, footer and
    headers, or layouts including areas for logging results on long-running
    tasks.
  \item
    Experiment with the generation of a Jupyter-widgets based GUI from
    a declarative configuration file, or by introspecting a configuration object.
  \end{compactenum}

\end{compactenum}

\end{task}
