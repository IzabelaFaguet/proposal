% template for a task
% each task should be added to exactly one workpackage
% in the workpackage task list
\begin{task}[
  title=Reproducible X-ray crystallography workflows at European XFEL,
  id=reproducibility-euxfel,
  lead=XFEL,
  PM=36,
  wphases={6-48},
  partners={XFEL}
  ]

  European XFEL is a research facility that provides X-ray Free
  Electron Laser (XFEL) light to image structures at the nanoscale. It
  is currently the world's most brilliant laser, created in a 3.4km
  long tunnel, and supporting user experiments since September
  2017. These imaging capabilities from European XFEL and similar
  services from synchrotron and neutron sources, underpin lots of
  fundamental and research, in domains covering fundamental physics,
  material science to biochemistry and drug design.

  European XFEL's data policy \cite{EuXFEL-datapolicy-2017} imposes a
  3-year embargo period on data acquired at the facility during user
  experiments, but makes the data accessibly to the public
  subsequently. A task in the EC funded project Photon and Neutron
  Open Science Cloud (PaNOSC) is using the Jupyter Ecosystem as of
  2019 to provide data analysis services to complement the data:
  through use of Jupyter Notebook and exploitation of the mybinder.org
  service, this activity will reduce the barrier for understanding and
  making use of the data, and eventually make those services available
  through EOSC.

  Here, we want to combine and use the new developments of this
  proposal to enable new qualities of open science services, and to
  demonstrate the potential impact of these improvements.

  \medskip Context: The very first experiments at European XFEL
  produced as little as 45 TeraByte of data on average, but as the
  facility develops, the amount of data produced per time is expected
  to grow substantially: Given the rate of light pulses, there is the
  potential to produce up to a PetaByte of data within the beam time
  of one experiment. These significant amounts of data need to be
  complemented by complicated workflows to convert the data into
  insight through data analysis. Results of such data analysis are
  typically much smaller in size and useful to archive together with
  the raw data. To explain how they have been obtained, the particular
  workflow of data analysis also needs to be archived.

  At European XFEL, it is explored how Jupyter Notebooks can be used
  as programs to facilitate this workflow: the simplest model would be
  to use one Notebook per workflow. This notebook can be executed
  (without being displayed in a webbrowser) once the data capture is
  completed to start processing the data. When the notebook has
  completed execution, it is saved, and contains the analysis results
  (it may of course also created files on disk as part of the
  process).

  A particularly useful aspect of the notebooks is that they mix data
  analysis commands with outputs, and that the notebook provides a
  reproducible summary of the data analysis when it succeeds with the
  execution. Should the execution fail, for example half-way through
  the notebook, then results obtained prior to the error occurring are
  preserved and can be inspected. The error is embedded in the
  notebook and appears after the command that has triggered the error;
  which helps with debugging the process.

  This is of particular interest as the data analysis processes may
  fail not because of software errors but due to variation in the data
  that require (manual) adjustments of parameters. The ``failure'' of
  such an analysis script is thus not exceptional, but a common
  occurrence. The scientist conducting the experiment is sufficiently
  skilled to modify the parameters and wants to either re-execute the
  script from the beginning or to continue from the point of
  failure. The notebook caters for both use cases. The modified
  notebook would need to be preserved of course.

  We are aiming for re-executability of the notebook for the lifetime
  of the data. At the moment, this is guaranteed for 5 years and aimed
  to be 10 years. It is possible though, that data used for
  publications will be preserved for longer, and it would be highly
  desirable to keep the data analysis re-executable for the same
  period of time, potentially exceeding 10 years.







  Using the improvements of task
  \taskref{server-state} in \WPref{core}), will allow us to l, moving
  the Notebook document state into the server



% applicability

  There is stiff competition for beam time to do research at European XFEL,
  and it is important to the 12 countries \TOWRITE{}{How many are EU member states?}
  which fund the facility that the best possible use is made of the data collected.
  All of the data will be made freely available after an embargo period of
  three years, and we are keen to complement this with reproducible data
  analysis, to confirm conclusions drawn and to allow re-analysis with new
  tools or for new purposes. \TODO{Cite data policy}
  If the analysis steps are not carefully recorded, there is a risk that the
  necessary understanding of the data is lost by the time it is made public,
  greatly reducing its scientific value.

  A major category of experiments already begun at European XFEL revolve around
  X-ray crystallography, using X-ray scattering to resolve molecular structures.
  Processing the data collected to produce meaningful structures is a
  significant computational task, and researchers have developed complex
  software for this, such as CrystFEL, Cheetah and CCTBX.
  % How do I properly cross-reference another task or workpackage?
  % using \WPref or \taskref.
  As an application of the reproducible workflows task in WP2, we will build
  workflows around such software, aiming to allow 'one button'' replication
  of published structures once the raw data is made available.
  As part of European XFEL's data policy, all data is to be publicly available
  ?? years after it is collected.
  The task includes the following activities:
  \begin{compactitem}
  \item ...
    (\localdelivref{deliv-id})
  \end{compactitem}


  \TODO{Mention PaNOSC}
\end{task}
