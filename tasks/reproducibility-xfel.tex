% template for a task
% each task should be added to exactly one workpackage
% in the workpackage task list
\begin{task}[
  title=Reproducible X-ray crystallography workflows at European XFEL,
  id=reproducibility-euxfel,
  lead=XFEL,
  PM=36,
  wphases={0-48},
  partners={XFEL}
  ]

  \TODO{which other partners are interested?}
  There is stiff competition for beam time to do research at European XFEL,
  and it is important to the 12 countries \TOWRITE{}{How many are EU member states?}
  which fund the facility that the best possible use is made of the data collected.
  All of the data will be made freely available after an embargo period of
  three years, and we are keen to complement this with reproducible data
  analysis, to confirm conclusions drawn and to allow re-analysis with new
  tools or for new purposes. \TODO{Cite data policy}
  If the analysis steps are not carefully recorded, there is a risk that the
  necessary understanding of the data is lost by the time it is made public,
  greatly reducing its scientific value.

  A major category of experiments already begun at European XFEL revolve around
  X-ray crystallography, using X-ray scattering to resolve molecular structures.
  Processing the data collected to produce meaningful structures is a
  significant computational task, and researchers have developed complex
  software for this, such as CrystFEL, Cheetah and CCTBX.
  % How do I properly cross-reference another task or workpackage?
  % using \WPref or \taskref.
  As an application of the reproducible workflows task in WP2, we will build
  workflows around such software, aiming to allow 'one button' replication
  of published structures once the raw data is made available.
  As part of European XFEL's data policy, all data is to be publicly available
  ?? years after it is collected.
  The task includes the following activities:
  \begin{compactitem}
  \item ...
    (\localdelivref{deliv-id})
  \end{compactitem}

  \TODO{Thomas, Hans, lots of work to be done here.}
\end{task}
