\begin{task}[
  title=Easy deployment of JupyterHub and BinderHub on a variety of
  infrastructure,
  id=jh-bh-deployment,
  lead=EP,
  PM=14, % EP: 11PM
  wphases={0-18}, % ???
  partners={UPSUD,SRL,WTT}]

  The JupyterHub or BinderHub installation tutorials mostly focuses on
  deployment on corporate cloud services (mainly GoogleCloud). For many academic
  institutions and SMEs, a deployment on their own infrastructure is preferable
  for a combination of reasons (better control on private data, easier
  integration with the local information system, the difficulty of funding the
  costs of external services, ...).

  There are many pre-existing academic clouds managed by people with a high
  level of expertise and a thorough knowledge of their infrastructure and
  associated tools. These are often more cost-effective for research and
  education. This is also true for SMEs. The purpose of this task is to make the
  deployment of JupyterHub or BinderHub easier upon these academic cloud
  computing and to provide scalable and high-quality infrastructure for
  education and research. Most of them are based on two open-source softwares
  which can manage this kind of infrastructure: OpenStack and OpenShift. We will
  then focus on these two tools.
  
  While platforms such as GoogleCloud already offer effective tools to deploy a
  Kubernetes cluster, this is not the case when we deploy our own
  infrastructure. The tools available for the two solutions mentioned above are
  numerous and it is often difficult to choose which ones are the most suitable
  and best maintained. We will therefore first take an interest in making a
  state of the art and explaining our choices. We already began this study at
  Ecole polytechnique (see the post in
  medium\footnote{\url{https://blog.jupyter.org/how-to-deploy-jupyterhub-with-kubernetes-on-openstack-f8f6120d4b1}})
  and we want to deepen it. At each step, we will use infrastructures available
  locally (OpenShift at Ecole polytechnique and OpenStack at University of
  Paris-Sud), available in France (OpenShift at Mathrice which is a "Groupements
  de Service" at CNRS of the research laboratories IT in French mathematics),
  available in Europe (EGI). We will document all the process to allow the
  community to easily deploy a JupyterHub or BinderHub on their own
  infrastructure based on OpenStack and/or OpenShift.

  Once the infrastructure is in place, even if JupyterHub or BinderHub provide
  efficient teaching or research environments, one issue remains: the lack of
  choice in terms of storage devices and data persistence. Thus, we will provide
  a unified environment and  focus on the possibility to mount various  storage
  devices on the containers deployed on the Kubernetes cluster and provide
  persistence storage for Binder. This will be a key issue for the task
  \taskref{core}{jh-bh-conv}.
  
  The task includes the following activities:
  \begin{compactitem}
  \item Prototypes / POC deployment on OpenStack and/or OpenShift
  \item Partial / Full automation of the deployment
  \item Documentation ($\rightarrow$
  \localdelivref{openstack-openshift-documentation})
  \end{compactitem}
\end{task}
