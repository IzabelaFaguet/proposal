% template for a task
% each task should be added to exactly one workpackage
% in the workpackage task list
\begin{task}[title=Easy deployment of JupyterHub and BinderHub on a variety of
  infrastructure,
  id=openstack,
  lead=EP,
  PM=6, % ???
  wphases={0-24}, % ???
  partners={EP,UPSUD}] % others? OVH?

  The JupyterHub or BinderHub installation tutorials mostly focuses on deployment
  on corporate cloud services (mainly GoogleCloud). For many academic institutions
  and SMEs, a deployment on their own infrastructure is preferable for a combination
  of reasons (better control on private data, easier integration with the local
  information system, difficulty of funding the costs of external services, ...).

  The purpose of this task is to make the deployment of JupyterHub or BinderHub
  easier on a variety of cloud infrastructure, in particular to help institutions
  and SMEs deploying them on premises.
  Since the deployment of JupyterHub or BinderHub is well documented for an
  installation upon Kubernetes cluster, we will focus on two opensource 
  softwares which can manage this kind of infrastructure (OpensStack and OpenShift). 

  While platforms like GoogleCloud already offer effective tools to deploy a 
  Kubernetes cluster, this is not the case when we deploy our own infrastructure.
  The tools available for the two solutions mentioned above are numerous and it
  is often difficult to choose which ones are the most suitable and best maintained.
  We will therefore first take an interest in making a state of the art and explaining
  our choices. We will then put it into practice to allow the community to easily
  deploy a JupyterHub or BinderHub on their own infrastructure based on OpenStack
  and/or OpenShift.

  Particular attention will be focused on the possibility to mount storage devices
  on the containers deployed on the Kubernetes cluster. This will be useful for
  the task xxx.

  The task includes the following activities:
  \begin{compactitem}
  \item Prototypes / POC deployment on OpenStack and/or OpenShift
  \item Partial / Full automation of the deployment
  \item Documentation
    % Deliverable: a report based on the documentation
    %(\localdelivref{deliv-id})
  \end{compactitem}
\end{task}
