\begin{task}[
  title=Application: Visualisation and control of fluid dynamics in Jupyter notebook,
  id=application-gpu,
  lead=SIL,
  PM=12,
  wphases={4-36},
  % don't include lead here
  partners={EGI}
]

\textbf{Context: Fluid dynamics with Lattice Boltzmann method on GPU}

In recent years, the lattice Boltzmann method (LBM) emerged as an interesting alternative to more established methods for fluid flow simulations. Sailfish-cfd \cite{januszewski2014sailfish} is an open source implementation of LBM on  General Purpose Graphical Processing Unit (GPGPU) devices. It is written in Python with real-time generation of CUDA-C code.  
In order to harvest capabilities of GPGPU one needs to access the specialized hardware, which usually is avaiable to researchers as remote HPC resources.  The typical fluid dynamics research workflow consists of three stages: preparing boundary conditions, running a simulation, and data analysis. The first and last stage require capable and responsive  user interface for maniputation and inspection of 3d data. 
The Jupyter  3d visualization widgets developed in \taskref{ecosystem}{jupyter-widgets} can fulfill such needs. 

\textbf{This task}

In this task we will improve and adapt existing open software to easily work in Jupyter notebook and fully use its interactive features. Based on previous experience with K3D-jupyter\cite{K3D} widget we know that web browser based software can display moderate dataset during the simulation
As the dataset is becoming larger the visualization in the browser turns out to be nontrivial due to limitations of the browser itself and large data transfers. There is an open question on how much of data processing should be performed on server-side and what can be done on the widget (browser side). Our experience suggests that there is no clear answer and it depends on the size of the data and its nature. For example, volume rendering technique if very effective on the browser side but infers large data transfers. One can perform it the server-side, in a distributed way if the simulation uses many nodes, but the interactivity is limited by network latency. In this task we will attempt to provide practical solution to this issue. 
Furthermore, we will contruct tools for editing and inspecting boundary conditions. Having such tools as Jupyter widgets will allow to complete the workflow without leaving Jupyter notebook.

The task includes the following activities

  \begin{compactitem}
  \item Development of Jupyter notebooks using LBM based fluid simulation based on the high-performance Sailfish-cfd solver.
  \item Implementing advanced widgets for data visualization of large fluid dynamics simulations.
  \item Implementing widgets for inspection and editing boundary conditions in LBM.
  \end{compactitem}

This part will closely interact in with the task \taskref{ecosystem}{jupyter-widgets}: it will both provide guidelines for the development to \taskref{ecosystem}{jupyter-widgets} and serve as test case for implemented features in \taskref{ecosystem}{jupyter-widgets}.


The deliverable of this part will be a demonstrator (\localdelivref{lbm-jupyter}) available via the EOSC hub.



\end{task}
