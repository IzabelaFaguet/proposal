\begin{task}[
  title=Demonstrator: Visualisation and control of fluid dynamics in Jupyter notebook,
  id=application-gpu,
  lead=SIL,
  PM=12,
  wphases={4-36},
  % don't include lead here
  partners={EGI}
]


In this task (see page \pageref{sec:concept-demonstrator-gpu}), we will contruct tools for editing and inspecting
boundary conditions in fluid dynamics simulation as well as capable and optimized visuzalization utilities. Having such tools as Jupyter widgets will allow to complete the  typical CFD workflow without leaving Jupyter notebook. We plan the following activities
\begin{compactitem}

\item Implement advanced widgets for data visualisation of large
  fluid dynamics simulations.
\item Implement widgets for inspection and editing boundary
  conditions in LBM.
\item Develop Jupyter notebooks demonstrating the full workflow of  fluid
  dynamics simulation based on the high-performance Sailfish-cfd solver.
  \item Publish selected demonstration notebooks for interactive use on
    GPU-enabled \TheProject's EOSC services (\delivref{applications}{demonstrators}).
\end{compactitem}

This work will closely interact in with the task
\taskref{ecosystem}{jupyter-widgets}: it will both provide guidelines and inspirations
for the development to \taskref{ecosystem}{jupyter-widgets} and serve
as test case for implemented features in
\taskref{ecosystem}{jupyter-widgets}.
%
  It will be reported on in
  \delivref{applications}{applications-report}.
\end{task}
