\begin{task}[
  title=Application: Visualisation and control of fluid dynamics in Jupyter notebook,
  id=application-gpu,
  lead=SIL,
  PM=13,
  wphases={4-36},
  % don't include lead here
  partners={EGI}
]

\textbf{Context}

In recent years, the lattice Boltzmann method (LBM) emerged as an
interesting alternative to more established methods for fluid flow
simulations. Sailfish-cfd \cite{januszewski2014sailfish} is an open
source implementation of the LBM on General Purpose Graphical Processing
Unit (GPGPU) devices. It is written in Python with real-time
generation of CUDA-C code.  In order to harvest capabilities of GPGPUs
one needs to access the specialized hardware, which usually is
available to researchers as remote HPC resources.  The typical fluid
dynamics research workflow consists of three stages: preparing
boundary conditions, running a simulation, and data analysis. The
first and last stage require capable and responsive user interface for
maniputation and inspection of 3d data.  The Jupyter 3d visualisation
widgets developed in \taskref{ecosystem}{jupyter-widgets} can fulfil
such needs.

Based on previous experience with K3D-jupyter\cite{K3D}
widgets we know that web browser based software can display moderate
dataset during the simulation. As the dataset is becoming larger the
visualisation in the browser turns out to be nontrivial due to
limitations of the browser itself and required large data transfers. It is
an open question how much of data processing should be performed on
server-side and what can be done on the client hardware (i.e. in the
widget in the browser side of the user). Our
experience suggests that there is no clear answer and it depends on
the size of the data and its nature. For example, volume rendering
technique can be very effective on the browser side but infers large data
transfers. One can perform it the server-side, in a distributed way if
the simulation uses many nodes, but the interactivity is limited by
network latency. We will attempt to provide practical
solutions to this issue.
%


\textbf{This task}


In this task, we will contruct tools for editing and inspecting
boundary conditions. Having such tools as Jupyter widgets will allow
to complete the workflow without leaving Jupyter notebook. We plan the
following activities
\begin{compactitem}
\item Development of Jupyter notebooks using fluid
  simulation based on the high-performance Sailfish-cfd solver.
\item Implementing advanced widgets for data visualisation of large
  fluid dynamics simulations.
\item Implementing widgets for inspection and editing boundary
  conditions in LBM.
\end{compactitem}

This work will closely interact in with the task
\taskref{ecosystem}{jupyter-widgets}: it will both provide guidelines
for the development to \taskref{ecosystem}{jupyter-widgets} and serve
as test case for implemented features in
\taskref{ecosystem}{jupyter-widgets}.

The deliverable of this part will be a demonstrator
(\localdelivref{lbm-jupyter}) available via the EOSC hub, and
contributions to report \localdelivref{applications-report}.

\end{task}
