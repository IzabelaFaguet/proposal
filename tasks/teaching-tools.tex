\begin{task}[
  title={Teaching tools, infrastructure, and best practices},
  id=teaching-tools,
  lead=EP,
  PM=21, % EP: 19PM, UPSud: 2PM
  wphases={0-36},
  partners={UPSUD}
]

  \TODO{argue about the relevance for EOSC: how will the outcome of
    this task make the generic Jupyter services provided by EOSC
    useful for education?}

  \TODO{List success stories: Data8 course @ Berkeley, OpenAI Gym,
    QuantEcon book C++ Orsay, courses on Coursera}

  In the recent years, Jupyter technologies have been widely adopted
  worldwide in higher education -- and even in high schools -- for
  teaching in all areas of sciences. The Jupyter notebook indeed
  provides a very versatile environment -- with a smooth learning
  curve -- for authoring interactive material such as class notes,
  exercise sheets, dedicated applets; all the way to complete books
  such as those produced by OpenDreamKit for biology, physics, and
  mathematics. The interactivity engages the students to take an
  active role, for example playing with code, exploring the effect of
  tweaking the parameters in a simulation, changing visualizing tools,
  adding personal notes. This lets them take progressively ownership
  of the material and better understands the issues, and encourages
  them to create their own documents and share their experience with
  colleagues and teachers.

  Setting up a comfortable working environment for both teachers and
  students, requires tools for easy sharing, collecting, self
  assessment, and semi-automatic grading of course material, class
  management, and integration with the local e-learning infrastructure
  such as, e.g., Moodle or OpenEDX.

  In this task, we will review the state of the art: existing tools,
  within the Jupyter ecosystem (e.g. nbgrader \cite{Hamrick2016} or okpy) and outside;
  course services (e.g. gryd.us or cocalc); course infrastructure that
  have been designed and deployed at many institutions (Berkeley,
  École Polytechnique, Université Paris Sud), etc.

  To build and share a better vision of the needs, we will conduct a
  survey in the education community about the usage of Jupyter. In
  particular we will seek feedback from Jupyter-based MOOC courses,
  e.g. on
  Coursera\footnote{\url{https://www.coursera.org/courses?query=jupyter}},
  and
  Fun\footnote{\url{https://www.fun-mooc.fr/courses/course-v1:inria+41016+session01bis/about}}.

  The collected requirements will be exposed in a first report
  \delivref{ecosystem}{teaching-report} and largely disseminated.

  The core of the task will then be to \textbf{further develop
    teaching tools, infrastructure, and course templates to contribute
    to the emergence of versatile solutions and best practices around
    them}.

  The outcome will be put into production by the participants of the
  BOSSEE project (and beyond!) who will deliver a large number of
  courses using Jupyter technology (see \taskref{applications}{teaching}). The variety of
  use cases and infrastructure will provide a rich test bed and
  immediate feedback at each iteration, ensuring that the developments
  are informed and steered by demand (co-design), and battle field
  tested.

  At this stage, we already envision specific development in the
  following directions:
  \begin{compactitem}
  % \item Review and follow up on related efforts: gryd.us, cocalc, Coursera/Fun,
  %   Berkeley, Ecole polytechnique, University of Paris Sud, ...
  % \item Survey of the needs in the education community
  \item Collaborative grade management
  \item Insulation through container of the automatic grading
  \item Integration with e-learning platforms (e.g. Moodle, OpenEDX
    (Coursera/Fun)), through an LTI connector.
  \item Develop course templates for various use cases.
  \item Disseminate tutorials on all of the above.
  \end{compactitem}

  A second report \delivref{ecosystem}{nbgrader-like} will review the
  developments, our in-class experience with them, and best practices.
\end{task}
