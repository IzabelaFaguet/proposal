% template for a task
% each task should be added to exactly one workpackage
% in the workpackage task list
\begin{task}[
  title=Teaching tools,
  id=teaching-tools,
  lead=EP,
  PM=20, % 18 for RSE, 2 for supervision
  wphases={0-48},
  partners={EP,UPSUD}]

  The Jupyter notebooks offer much more interesting perspectives than the usual
  course provided in pdf format. The teacher can illustrate his course with
  videos, code,... The course becomes interactive. The student is no longer
  passive. He can use the notebook and play with the code provided by
  the teacher. He can thus change the parameters of the simulation, change the
  numerical method, represent the results in another way,... He takes ownership
  of the course and better understands the issues. He can then make his own
  notebooks and share his experiences with his colleagues and teachers.

  In order to have the best working environment for both teacher and student, it
  is necessary to have tools for easy sharing,
  collecting, self assessment and semi-automatic grading of course material,
  class management, and integration with the local e-learning infrastructure.
  
  There are already many tools available to manage classes (Moodle, OpenEDX) and tools to
  share course documents and grading (nbgrader, okpy, ...). The purpose of this
  task is therefore not to rewrite a new tool but to try to use existing ones
  and improve or enrich them in order to achieve a unified framework.

  A survey will be delivered to the education community using Jupyter in order to have
  a better vision of the needs. Then we will then propose a tool that meets
  these needs. It will be also important for us to have feedbacks from MOOC courses which use Jupyter.

  It is also important to notice that the partners of the BOSSEE project will
  deliver a large number of courses using Jupyter technology. The variety of use
  cases and infrastructure will provide a rich test bed for the further
  development of tools and best practices around them.

  The task includes the following activities:
  \begin{compactitem}
  \item Review and follow up on related efforts: gryd.us, cocalc, Coursera/Fun,
    Berkeley, Ecole polytechnique, University of Paris Sud, ...
  \item Survey of the needs in the education community
  \item Collaborative grade management
  \item Insulation through container of the automatic grading
  \item Integration with e-learning platforms (e.g. Moodle, OpenEDX
    (Coursera/Fun)), through an LTI connector.
  \item Develop course templates for various use cases
  \end{compactitem}
\end{task}
