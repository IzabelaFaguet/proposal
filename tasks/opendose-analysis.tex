\begin{task}[
  title=Demonstrator: Nuclear Medicine,
  id=opendose-analysis,
  lead=INSERM,
  PM=22,
  wphases={3-33},
  partners={EGI,XFEL}
]


  The objective of this task is to build on the Jupyter ecosystem to create a
  unified data analysis framework for the OpenDose project.
  The task includes the following activities:
  \begin{compactitem}
  \item Developing tools to work seamlessly on the SQL database holding the
    dosimetric data.
  \item Developing data analysis tools using the Python data science ecosystem
    where possible.
  \item Developing visualization tools, exploring Widgets inside the Notebook
    for interactivity.
  \item Evaluate the relevance of Jupyter Notebooks from user feedback
  \item Disseminating results.
  \item Providing support to users.
  \end{compactitem}
  First, these developments will be available to users as Jupyter Notebooks to
  be run locally on their machines. In a second time, these Notebooks will be
  ported on the European Binder instance following developments of task
  \taskref{eosc}{eu-binder}. The reproducibility of generated results will be
  ensured by archiving the software environment with developments of task
  \taskref{ecosystem}{reproducibility}.

  The deliverable of this task will be a demonstrator
  (\localdelivref{opendose-analysis}) available via the EOSC hub, contributing
  to the development of innovatives services for the scientific community.

  The evaluation of the developed tools from user feedback will contribute to
  the deliverable \localdelivref{applications-report}.

\end{task}
