% template for a task
% each task should be added to exactly one workpackage
% in the workpackage task list
\begin{task}[
  title=Interactive C++ in Jupyter with XEUS,
  id=xeus-cpp,
  lead=QS,
  PM=12, % EP: 1PM
  wphases={0-48},
  partners={QS,UPSUD,EP}
]
%\hypertarget{interactive-computing-with-interpreter-c}{%
%\section{Interactive Computing with Interpreter
%C++}\label{interactive-computing-with-interpreter-c}}

\begin{compactenum}

\item
  cling \& xeus-cling

  \begin{compactenum}
  \item
    further development of xeus \& xeus-cling for a fully-fledged
    \textbf{Jupyter} experience with the C++ programming language.
  \item
    improved \textbf{packaging} of cling and related packages for conda
    and other package managers, including full \textbf{windows} support.
  \item
    continued development of C++ \textbf{interactive widgets} and
    backends for ipyvolume, bqplot, ipyleaflet (xvolume, xplot,
    xleaflet), and better testing for the feature parity with
    ipywidgets.
  \item
    production of \textbf{teaching material} for C++ with xeus-cling and
    interactive widgets.
  \item
    integration with read-only notebook viewers such as
    \href{https://github.com/QuantStack/voila}{\textbf{voila}} to
    produce a fully compiled application from a notebook interacting with
    Jupyter frontend components such as interactive widgets.
  \item
    enable different levels of \textbf{compiler optimization} for code
    interpreted by cling.
  \item
    improve the \textbf{magics} plugin system to simplify authoring of
    custom magics for xeus-cling.
  \end{compactenum}

\item
  core xeus library

  \begin{compactenum}
  \item
    improve language bindings for the C++ interactive widgets to enable
    interactive widgets in all xeus-based kernels.
  \item
    enable pluggable history managers, with specialized implementations
    for both SQLite-based history and in-memory history.
  \end{compactenum}

\item
  cling support for common C++ scientific computing packages

  \begin{compactenum}
  \item
    cling offers the possibility to automatically load the runtime of
    libraries upon the inclusion of its headers using special pragmas.
    Several libraries such as \texttt{xtensor}, and \texttt{symengine}
    now make use of this possibility. We propose to generalize this
    approach to the main C++ libraries for scientific computing that may
    accept such pragmas to be included upstream.
  \end{compactenum}
\end{compactenum}

Scientists, educators, and engineers not only use programming languages
to build software systems, but also in \textbf{interactive workflows},
using the tools at hand to explore and reason about problems.

Running some code, looking at a visualization, loading data, and running
more code. Quick iteration is especially important during the
exploratory phase of a project.

While C++ is ubiquitous in scientific computing for close-to-the-metal
performance number crunching, \emph{we lack a good story for interactive
computing in C++.} This hurts the productivity of C++ developers:

\begin{itemize}
\item
  Progress in software projects often comes from
  \textbf{incrementalism}. Obstacles to fast iteration hinder progress.
\item
  his also makes C++ more \textbf{difficult to teach}. The first hours
  of a C++ class are rarely rewarding as the students must learn how to
  set up a small project before writing any code. And then, a lot more
  time is required before their work can result in any visual outcome.
\end{itemize}

The cling interpreter fills the gap of interactivity for the C++
programming language and its use at scale in at LHC proves that C++ can
be a language for interactive scientific computing.

\textbf{The xeus-cling kernel}

The goal of the xeus-cling project is to improve the integration with
Project Jupyter and make C++ a \textbf{first-class citizen} of the
ecosystem.

By leveraging the rich ecosystem of tools built upon Project Jupyter,
which is language agnostic, we can lift C++ from a language reserved to
high-performance performance computing to a high-level language of data
science like Python, R, or Julia.

\textbf{Teaching the C++ programming language}

Since September 2017, the 400 first-year students at Paris-Sud
University who take the ``Info 111: Introduction to Computer Science''
class write their first lines of C++ in a Jupyter notebook, with the
xeus-cling kernel.

The use of project Jupyter for teaching C++ is especially useful for the
first classes where students can focus on the syntax of the language
without distractions such as compiling and linking a program.

The availability of \textbf{Jupyter interactive widgets} in that
environment offers a simple means to obtain a \textbf{visual outcome} in
a few lines of code, for a more \textbf{rewarding learning experience}.
It is also not typical for the C++ programming language.

Finally, the \textbf{cloud hosting} of the environment removes the
hurdle of installing a development environment on a large variety of
student's machines.

\textbf{C++ as a common denominator}

The fragmentation of the ecosystem between the main languages of data
science causes a lot of duplication of work and harms sustainability in
the long run. Often, implementation of standard protocols, file formats,
and reference implementations of numerical methods are duplicated in
each language.

A common denominator of the three main languages of data Science (Julia,
Python, R, forming the Jupyter name) is their ability to call into
natively built libraries. All three interpreters have a clean C API that
can also be used from the C++ programming language.

However, a solid C++ implementation can always be exposed to Julia,
Python and R, making the common implementation more sustainable in the
long run. The static typing of the language may make the initial
implementation less immediate, but for greater stability.

For this reason, we strive to provide solid C++ implementation of

\begin{compactenum}
\item
  standard protocols, such as the Jupyter kernel protocol (with xeus),
  now used to make other kernels than the C++ kernel.
\item
  standard in-memory and file formats such as apache arrow, FITS,
  NetCDF, HDF5.
\item
  data structures, such as N-D arrays and dataframes
\item
  reference implementation of numerical methods.
\end{compactenum}

Reference scientific Python projects such as Sympy have moved their core
to a solid C++ engine (such as symengine), which has then been exposed
to Julia and R.

\end{task}
