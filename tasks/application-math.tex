\begin{task}[
  title=Application: Interactive Mathematics with Jupyter Widgets,
  id=math,
  lead=UPSUD,
  PM=18, % UPSUD:17, QS:1
  wphases={0-36},
  partners={EP,QS}
  ]

  \TODO{Ideas to reinforce the ties with EOSC services welcome!}

  Computations have played a long time and ever increasing role for
  research and teaching in (pure) mathematics, to explore, search and
  check for conjectures, or better understand algorithmic ideas. This
  led to the development of a whole ecosystem of mathematical
  software, many of which are open source. Given the huge variety of
  mathematical objects and workflows, the Read-Eval-Print-Loop (REPL)
  paradigm -- on which Jupyter is based -- is particularly suitable:
  the user interacts with the system by typing commands that use its
  library of mathematical features, often combined with personal code.
  In fact, the REPL and notebook paradigms of Jupyter as well as some
  of its interactive features were largely inspired by that of
  computer algebra systems such as Maple, Mathematica, or SageMath.

  One major action of the OpenDreamKit project was to foster the
  convergence between the Jupyter and math software ecosystems:
  nowadays Jupyter can be used as a uniform user interface for most
  major systems: e.g. GAP, OSCAR, Pari/GP, SageMath, Singular, and
  even for C++ libraries. This interface is being widely adopted: for
  example, Jupyter has become the standard user interface for
  SageMath, enabling to phase out its former bespoke notebook; by now,
  thousands of jupyter notebooks for SageMath are publicly shared
  (6000+ on GitHub alone).

  Thanks to this prior art, the mathematical community will
  immediately enjoy all the benefits brought by EOSC-based generic
  Jupyter services, including eased collaboration, sharing, archival,
  and reproducibility.

  The next step to maximize attractivity and impact in the
  mathematical community, and this is the aim of this task, is to go
  beyond the REPL paradigm, and \textbf{leverage the real time
    interactivity and flexibility brought by Jupyter widgets for
    Mathematical purposes}. Think making it easy for a teacher or
  researcher to build and disseminate via the EOSC a mini applications
  or dashboard enabling the graphical exploration of a whole range of
  mathematical inputs, with real-time visualization of the associated
  outputs.

  The unique challenge comes from the huge variety of mathematical
  objects that the user may want to visualize and interact with, and
  the variety of graphical representations. Co-design is central here,
  as building a bespoke interactive visualization entails a
  combination of technology skills (e.g. javascript development) and
  business knowledge (designing the interaction and visualization).
  The role of Research Software Engineers is to leverage the
  technology by encapsulating the technical difficulties into flexible
  and easy to use tool boxes from which mathematicians can build
  mini-applications tailored to their needs.

  Within OpenDreamKit, we conducted experiments to explore this
  venue~\cite{ODK_D4.16}. One specific focus was to enable not only
  \emph{interactive visualization}, but also \emph{interactive
    editing}: being able to graphically modify the mathematical object
  being visualized; this enable the interactive exploration of how the
  modifications affect its properties, or to use the editor as input
  widget for a larger applications or dashboards. The outcome of this
  task are the development of two prototypes in SageMath
  (\software{sage-combinat-widget}, a library of widgets for
  combinatorics, and \software{sage-explorer} a generic dashboard for
  interactive browsing and introspection of mathematical objects), and
  contributions to \software{Francy}, an Interactive Discrete Math
  Framework for \software{GAP} and \software{SageMath}.

  The aim of this task is to build on this experience to further
  develop and promote the use of Jupyter widgets for interactive
  Mathematics. This will include the following actions:
  \begin{compactitem}
  \item Engage with the community through tutorials, workshops, online
    discussions, for codesign and for dissemination of the outcomes.
  \item Tackle hurdles to real-time interactivity, typically by
    modernizing the existing 2D and 3D visualization tools in
    SageMath. % E.g.: We don't use Matplotlib's integration in Jupyter
  \item Bring \software{sage-combinat-widgets} and
    \software{sage-explorer} from usable prototypes to standard tools,
    and further contribute to the development of the \software{Francy}
    framework.
  \item Develop other generic mathematical widgets according to the
    users popular requests.
  \item Demonstrate the value all of the above through applications in
    research and teaching.
  \end{compactitem}
  The work carried over will be reported on in~\localdelivref{math}.
\end{task}
