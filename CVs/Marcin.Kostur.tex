\begin{participant}[type=leadPI,PM=3,gender=male]{Marcin Kostur}
is an assistant Professor at the Institute of Physics. He is the author of over 50
publication cited over 2000 times in the field of statistical physics,
solid state physics (Josephson Junction dynamics), microfluidics and
biophysics. He is experienced in application of GPU architecture to
numerical simulations of stochastic processed in physics. His recent
computational interests are focused at the Open Source project
\software{Sailfish} -- HPC implementation of Lattice Boltzmann Method on GPU.
He is leader few projects including  computations in the science education and e-infrastructure:
\begin{compactitem}
\item Infrastructure for cloud-based system education: scalable implementation of Jupyter notebook system for scientific explorations, project funded by Erasmus+, Key Action 2 - ``Strategic Partnership'', (budget: \euro{160}k, 2017-2019)
\item Computing in high school science education - iCSE4schools,
  project funded by Erasmus+, Key Action 2 - ``Strategic Partnerships'',
  (budget: \euro{263}k, 2014-2017)
\item ``Computers in Science Education: iCSE'' http://icse.us.edu.pl
  (budget: \euro{1}m, funded by EFS, 2011-2014)
  
  \item  PAAD (Platform for Analysis and Archiving of Data) project funded by POIG program for 2014-2015 with a total budget
  of \euro{4}m. The task coordinator``Interactive HPC services for science''. 
\end{compactitem}
\end{participant}