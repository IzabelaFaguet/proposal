\begin{participant}[type=R,PM=8,gender=male]{Loic Gouarin}

  Loic Gouarin is Research Engineer in scientific computing at CMAP (Centre de
  Mathématiques Appliquées) at \'Ecole polytechnique. He works on several
  scientific computing open-source projects in different fields such as
  Lattice-Boltzmann methods, Stokes solvers for fluid particles interaction,
  adaptive mesh refinement, ...

  He is also director of the ``GdR Calcul'' where his role is to animate the
  scientific and high performance computing community in France, in particular
  by organising conferences, meetings, and seminars. In this context, he
  organises himself 3 to 4 training and development workshops per year, and
  promotes the use of Python and c++ for teaching and research in France.
  
  For several years, he has been very involved in promoting the Jupyter project
  and its use for teaching and research in the French community. He is one of
  the core developers of xeus-cling and he is working on the possibility of
  easily deploying a JupyterHub or BinderHub on academic clouds. He also
  believes that reproducible research is an essential part of promoting new
  computation codes, new numerical methods, ... introduced in related
  publications and therefore he uses Jupyter as a first approach to achieve this.
  
  \end{participant}
  
  %%% Local Variables:
  %%% mode: latex
  %%% TeX-master: "../proposal"
  %%% End:
  