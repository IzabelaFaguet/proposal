\begin{participant}[type=PI,PM=1,gender=male]{Sandor Brockhauser}
  % type is one of:
  % - leadPI: leader of the participating institution
  % - PI: Principal Investigator
  % - R: researcher?
  % Who is the coordinator is specified elsewhere

  % PM=YYY:
  % A fair evaluation of the number of months you will be
  % spending on this specific project along the four years.
  % Typical numbers:
  % - full time hired personnel: 48 months
  % - lead PI or proposal coordinator: 8-12 months
  % - PI: 4-5 months
  % - participant: 2-6 months

  % salary=ZZZ:
  % Approximate monthly gross salary (in term of total cost for the
  % employer). This is optional. If you are uncomfortable having this
  % information in a public file, you can alternatively send the
  % information to Eugenia Shadlova, or to your institution
  % leader/manager if he is willing to fill in himself the budget
  % forms on the eu portal.

  % The above information is used to fill in various tables in the
  % proposal file, and to evaluate the cost of the project for the
  % institutions.

  % You may remove all those comments.

  % About half a page of free text; for whatever it's worth, you may see
  % Nicolas.Thiery.tex for an example.


  \medskip

  Sandor Brockhauser is the head of the Control and Analysis Software
  Group at the European XFEL. He received his M.Sc. in Informatics
  from Technical University of Budapest, Hungary, earned a Ph.D. from
  University of Leoben, Austria, and received his HDR degree in
  physics at the University of Joseph Fourier, Grenoble in
  France.

  Since 2004, when he joined the European Molecular Biology
  Laboratory (EMBL) in Grenoble, France, he worked in Macromolecular
  Crystallography and became the scientist in charge of the Multi-
  Wavelength Anomalous Dispersion Beamline ID14-4 at European
  Synchrotron Radiation Facility (ESRF), France. Between 2013-15, he
  moved to Szeged, Hungary where he joined the Extreme Light
  Infrastructure, ELI-ALPS, and has established and built up its
  Scientific Engineering Division. In the same time, he also
  established the X-ray Crystallography Laboratory at a European
  Center of Excellence, the Biological Research Center, Szeged of the
  Hungarian Academy of Sciences. Moving to the European XFEL in 2016,
  he became responsible for the full control system of the beamlines
  and scientific instruments, Karabo, that enables the integration of
  Experiment Control, Data Acquisition and Analysis. During the last
  two years, Karabo has been deployed, photon beamlines were
  successfully commissioned and two of the initial scientific
  instruments have been put in operation producing 0,5PT of data in 5
  weeks of experiments. Jupyter tools are embedded in the Karabo
  system and European XFEL analysis activities.

\end{participant}

%%% Local Variables:
%%% mode: latex
%%% TeX-master: "../proposal"
%%% End:
