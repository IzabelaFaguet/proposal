\begin{participant}[type=leadPI,PM=24,gender=female]{Anne Fouilloux}
  % type is one of:
  % - leadPI: leader of the participating institution
  % - PI: Principal Investigator
  % - R: researcher?
  % Who is the coordinator is specified elsewhere

  % PM=YYY:
  % A fair evaluation of the number of months you will be
  % spending on this specific project along the four years.
  % Typical numbers:
  % - full time hired personnel: 48 months
  % - lead PI or proposal coordinator: 8-12 months
  % - PI: 4-5 months
  % - participant: 2-6 months

  % salary=ZZZ:
  % Approximate monthly gross salary (in term of total cost for the
  % employer). This is optional. If you are uncomfortable having this
  % information in a public file, you can alternatively send the
  % information to Eugenia Shadlova, or to your institution
  % leader/manager if he is willing to fill in himself the budget
  % forms on the eu portal.

  % The above information is used to fill in various tables in the
  % proposal file, and to evaluate the cost of the project for the
  % institutions.

  % You may remove all those comments.

  % About half a page of free text; for whatever it's worth, you may see
  % Nicolas.Thiery.tex for an example.

  \medskip PhD, is a highly experienced Research Software Engineer dedicated to supporting
  researchers towards the adoption of Open Science best practices.

  With a solid background in Computer Sciences, she worked in various application fields, including environmental sciences, Intelligent Transport Systems, High-Performance computing, bio-informatics, meteorology and Geosciences.

  She is currently working in the IT group of the department of Geosciences at the \href{https://www.mn.uio.no/geo/english}{University of Oslo} and holds a 25\% at the \href{https://neic.no}{Nordic e-Infrastructure Collaboration} (NeIC) where she is involved on the \href{https://neic.no/nicest/}{Nordic Collaboration on e-Infrastructures for Earth System Modeling} (NICEST) and \href{https://coderefinery.org}{CodeRefinery}\ref{desc:coderefinery} projects on Training and e-Infrastructure for Research Software Development. 

   Since 2015, Anne Fouilloux has been very active with \href{https://carpentries.org}{The Carpentries}, a diverse and global community of volunteers and she teaches foundational coding and data science skills to students and young researchers. She is a certified \href{https://carpentries.org/instructors/}{Carpentries instructor}, \href{https://carpentries.org/trainers/}{instructor trainer} and \href{https://carpentries.org/maintainers/}{maintainer}. She has volunteered to help build \href{http://www.carpentrycon.org/}{CarpentryCon 2020} a biannual conference for members of the global Carpentries community and people with similar interests. 

  She is a member of the core team of the \href{https://www.uio.no/english/for-employees/support/research/research-data/training/carpentry/}{Carpentry@UiO} and is leading the \href{https://uio-carpentry.github.io/studyGroup/}{studyGroup@UiO} where students and researchers at the University of Oslo are committed to sharing skills, experiences, and ideas around open science, open source, code, and community in research.
\end{participant}

%%% Local Variables:
%%% mode: latex
%%% TeX-master: "../proposal"
%%% End:
