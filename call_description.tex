Specific Challenge:
Develop an agile, fit-for-purpose and sustainable service offering accessible through the EOSC hub that can satisfy the evolving needs of the scientific community by stimulating the design and prototyping of novel innovative digital services. Innovative models of collaboration that genuinely include incentive mechanisms for a user oriented open science approach should be considered.

Scope:
Research and Innovation Actions that target gaps in the service offering of the EOSC hub and develop innovative services that address relevant aspects of the research data cycle (from inception to publication, curation, preservation and reuse), for example allowing implementation of new scientific data-related developments and intelligent linking and discovering of all research artefacts.

Whereas initially the new services would have to respond to specific needs of particular scientific communities by the end of the project they should be leveraged to foster interdisciplinary research, serving a wider remit of research needs, as well as new users like industry and the public sector. Scalability of the new solution should be tested by user communities from different disciplines during the project lifetime. These services should be based on systems and technologies that have reached TRL 6 before the start of the project and will be brought to at least TRL 8 by the end of the project. Proposals should demonstrate how the resulting services complement, enrich and could potentially be integrated into the EOSC hub. Proposals retained for funding under this topic should take due consideration of any accessibility requirements set under the projects funded under EINFRA-12-2017 topic that may be available at the time the call will be open, in view of their integration into the mainstream services of the EOSC hub.

Consortia are encouraged to include SMEs that are willing to develop or contribute to the development of new innovative interdisciplinary services with a view of future integration in the EOSC hub.

The Commission considers that proposals requesting a contribution from the EU of between EUR 5 and 6 million would allow this challenge to be addressed appropriately. Nonetheless, this does not preclude submission and selection of proposals requesting other amounts.

Expected Impact:
Integrating co-design into research and development of new services to better support scientific, industrial and societal applications benefiting from a strong user orientation;
Supporting the objectives of Open Science by improving access to content and resources, and facilitating interdisciplinary collaborations;
Fostering the innovation potential by opening up the EOSC ecosystem of e-infrastructure service providers to new innovative actors.
Cross-cutting Priorities:
Open Science
