\TOWRITE{NT/...}{Finalise}
\TOWRITE{JC}{Proofread concept and approach pass 1
  \begin{compactitem}
  \item Checking whether the overall structure of the narrative
    makes sense
  \item Checking how much this fulfills all the points raised by
    Wolfram
  \end{compactitem}
}
\TOWRITE{ALL}{Proofread concept and approach pass 2}

\subsection{Concept and Approach}\label{sec:concept}
\eucommentary{5-8 pages}
\eucommentary{
-- Describe and explain the overall concept underpinning the project.
Describe the main ideas, models or assumptions involved. Identify
any trans-disciplinary considerations;
-- Describe and explain the overall approach and methodology, distinguishing, as
appropriate, activities indicated in the relevant section of the work programme, e.g.
Networking Activities, Service Activities and Joint Research Activities, as detailed in
the Part E of the Specific features for Research Infrastructures of the Horizon 2020
European Research Infrastructures (including e-Infrastructures) Work Programme 2014-
2015;\\
-- Describe how the Networking Activities will foster a culture of co-operation between the
participants and other relevant stakeholders.\\
-- Describe how the Service activities will offer access to state-of-the-art infrastructures,
high quality services, and will enable users to conduct excellent research.\\
-- Describe how the Joint Research Activities will contribute to quantitative and qualitative
improvements of the services provided by the infrastructures.\\
-- As per Part E of the Work Programme, where relevant, describe how the project will
share and use existing basic operations services (e.g. authorisation and accounting
systems, service registry, etc.) with other e-infrastructure providers and justify why such
services should be (re)developed if they already exist in other e-infrastructures. Describe
how the developed services will be discoverable on-line.\\
-- Where relevant, describe how sex and/or gender analysis is taken into account in the
project's content.}

%\minitoc
