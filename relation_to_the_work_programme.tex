\subsection{Relation to the Work Programme}

The \TheProject project addresses the challenges of the ``Prototyping
new innovative services'' call (ID: INFRAEOSC-02-2019).

The Jupyter Project is already widely adopted in numerous communities and used by millions of researchers worldwide.

\begin{itemize}
\item \emph{Journalists} and practitioners of \emph{data-driven journalism},
\item \emph{Research institutions} such as CERN, JRC, and many more, operating institution-wide Jupyter deployment,
\item \emph{Universities} using Jupyter as a teaching platform,
\item \emph{Large cloud providers} building commercial products on the top of Jupyter (Google DataLab, Amazon Sagemaker, Microsoft Azure Notebooks),
\item \emph{Other EOSC projects}. Jupyter is already planned to become an important service on the European Open Science Cloud through the EOSC-04-funded PaNOSC project [1].
\end{itemize}

While these projects are building upon Jupyter, and are being supported by the core team in their endeavor. Our proposal deals with developing the next generation of the Jupyter tools, and their relation with EOSC.

[1] EINFRAOESC-02 call (\url{https://ec.europa.eu/info/funding-tenders/opportunities/portal/screen/opportunities/topic-details/infraeosc-02-2019;freeTextSearchKeyword=innovative;typeCodes=0,1;statusCodes=31094501,31094502;programCode=null;programDivisionCode=null;focusAreaCode=null;crossCuttingPriorityCode=null;callCode=Default;sortQuery=openingDate;orderBy=asc;onlyTenders=false)}
