\TOWRITE{ALL}{Add / check resources for your site}

\eucommentary{Please provide the following:
\begin{compactitem}
\item
a table showing number of person/months required (table 3.4a)
\item
a table showing 'other direct costs' (table 3.4b) for participants where
those costs exceed 15\% of the personnel costs (according to the budget
table in section 3 of the administrative proposal forms)
\end{compactitem}}

\subsubsection{Management Level Description of Resources and Budget}
\label{sect:budget-details}

\paragraph{Staff efforts}

\eucommentary{Please indicate the number of person/months over the whole
duration of the planned work, for each work package, for each participant.
Identify the work-package leader for each WP by showing the relevant
person-month figure in bold.}

The \TheProject project is gathering sites with core
developers from the Jupyter project and history in open source
software development, and brings them together with domain specialists
from a range of domains. The major investment of the project is in
software development, which is realised through person time.

\ifgrantagreement.\else{} %
displayed in the following table.
\wpfig[label=fig:staffeffort,caption=Summary of Staff Efforts]
\fi

\paragraph{Travel, dissemination, and outreach}

The nature of this proposal -- of providing a framework that allows
design and deployment of innovative services -- means that the project
has the potential to have high impact for EOSC. At the same time, it
requires input from and engagement with a significant number of
stakeholders, including potential users of the services such as
scientists, developers of other services and other EOSC-funded
projects, other open science software projects, and the developing
EOSC itself. Consequently, requirements capture, networking, feedback,
training and education workshops and outreach activities are all
important, and the second highest cost for this project.

\subparagraph{Guidelines for travel and dissemination}
\label{sect:budget-details-travel}

We use the following guidelines for expected travel expenses:
\euro{2500} for attendance of a typical one week international
conference outside Europe (including travel, subsistence,
accommodation and registration), \euro{1250} for a corresponding
conference in Europe, \euro{750} for a one-week visit of a project
partner, for instance for coding sprints and one-to-one
research visits. We expect a similar cost per week while hosting
visitors. For the half-yearly project meetings, we expect on average a
cost of \euro{500} for travel, accommodation and subsistence.

Anticipated activities:

\begin{enumerate}
\item \emph{Project meetings}: For the 9 project meetings that take place every 6 months, we expect
the PI to attend all of them (cost of 9 * 2 * 500 = \euro{9000}). For
a researcher, we also expect that the attend all such project meetings
(\euro{9000}).

\item \emph{Hosting visitors}: We expect that the site spends \euro{2000} per year to host
external visitors contributing to the project (total \euro{8000}).

\item \emph{Site visits}: We expect the researcher to carry out 3 one-week visits to other sites
(each at \euro{750}) every year, totalling 3 * 4 * 750 = \euro{9000}
over 4 years).

\item
\emph{ Conference dissemination}: We expect the researcher to attend on average 1
international conference and 1 European meeting per year (cost of 4 *
2500 + 4 * 1250 = \euro{15000}) and the investigator to attend the
equivalent of one international or two
European gatherings (totals \euro{10000}).
\end{enumerate}

Where there are multiple investigators per site, they will share the
travel and associated costs outlined above. Where there are multiple
researchers, or researchers not employed for the full 48 months, the
travel budget is adapted accordingly.



\subparagraph{Guidelines for outreach costs}

\label{sect:budget-outreach-publication-charges}
\emph{Publication charges}: We also request \euro{3000} per year per partner to pay for open
access publication charges.  (Some partners have other means do pay
these costs, and for them these are not needed.)

\label{sect:budget-outreach-workshops}
\emph{Workshops}: We request funds for dissemination and outreach activities such as
workshops that facilitate community building, provide training and
disseminate best practice and encourages sustained contributions of
the community to the project and beyond the lifetime of the
funding. For a one-week workshop that we organise, we assume a cost of
400 EUR per participant to provide accommodation and catering. A
workshop for 20 people will thus cost about \euro{8000}. Participants
donate their time and need to fund their travel from other sources. By
partially contributing to the attendance cost, we hope to enable
PhD students to engage with the project and expect positive effects on
the sustainability of the activities, by embedding the tools and
knowledge with the next generation of scientists.

Details are given in the tables below and in the work packages.

\bigskip


 \subsubsection{Resource summaries for consortium member sites}
 \label{resources.summary}

 %%%%%%%%%%%%%%%%%%%%%%%%%%%%%%%%%%%%%%%%%%%%%%%%%%%%%%%%%%%%%%%%
 %
 % Guidelines for completion of partner specific resource summary:
 %
 %
 % Please explain how many person months for each person are
 % requested. Say who is the local lead. Say anything that helps to
 % understand why people are recruited as you plan, in particular if
 % this deviates from having one research for 48 months.  We can also
 % use this bit of the proposal (and the table, see below) to address
 % any other unusual arrangements.
 %
 %
 % The table should contain all non-staff costs (the EU requests that
 % this table must be present if the non-staff costs exceed
 % 15% of the total cost, but it is good practice and will show
 % openness and transparency that we show the data for all partners).
 %
 % Link back from the table to the work packages and tasks for which
 % the expenses are required. Add information that makes it easier to
 % understand why the expenses are justified.
 %
 %     To refer to a task in a work package, use "\taskref{WP-ID}{TASK-ID}" where
 %     WP-ID is the ID of the work package:
 %        WP#: WP-ID - full title
 %        ----------------------
 %        WP1: 'management' - Management
 %        WP2: 'community' - Community Building and Engagement
 %        WP3: 'component-architecture' - Component Architecture
 %        WP4: 'UI' - User interfaces
 %        WP5: 'hpc' - High Performance Computing
 %        WP6: 'dksbases' - Data/Knowledge/Software-Bases
 %        WP7: 'social-aspects' - Social Aspects
 %        WP8: 'dissem' - Dissemination
 %
 %
 %     and "TASK-ID" is the ID of the task. You can set this using
 %
 %       \begin{task}[id=TASK-ID,title=Math Search Engine,lead=JU,PM=10,lead=JU]
 %
 %     To refer to deliverables, use "\delivref{WP-ID}{DELIV-ID}" where DELIV-ID is
 %     the ID of the deliverable that can be set like this:
 %
 %       \begin{wpdeliv}[due=36,id=DELIV-ID,dissem=PU,nature=DEM]
 %           {Exploratory support for semantic-aware interactive widgets providing views on objects
 %           represented and or in databases}
 %       \end{wpdeliv}
 %
 %
 % The table is pre-populated with entries most sites are likely
 % to need. If a line does not apply to you, just delete it. If you need
 % an extra line, then add it. Use common sense: the number of rows should not
 % be very big, but at the same time it is useful to give some breakdown/explanation
 % of costs.
 %
 %
 % Eventually, try to create you entry similar in style to the others.
 % (The Southampton entry is fully populated, so use this as guidance
 % if in doubt.)
 %
 %
 %%%%%%%%%%%%%%%%%%%%%%%%%%%%%%%%%%%%%%%%%%%%%%%%%%%%%%%%%%%%%%%%

 In this section we briefly describe the requested resources. See the
 participant descriptions in the description of the consortium for the
 specific role of each member.

 %%%%%%%%%%%%%%%%%%%%%%%%%%%%%%%%%%%%%%%%%%%%%%%%%%%%%%%%%%%%%%%%%%%%%%%%%%%%%%
 \paragraph{Resources Simula Research Laboratory}


 % Travel
 % Assume PI (50%) and 2 FTE researchers
 %
 % Project meetings:
 % 3 * 9000 = 27,000
 % Hosting visitors = 8000
 % Site visits 2 * 9000 = 18000
 % Conference dissemination = 2 * 15000 + 1 * 10000 = 25000
 %
 % Publication costs: 3000
 % Workshops: 4 x 10 people = 16,000
 %
 % Total: 27000+8000+18000+25000+3000+16000 = 97000
 \site{SRL} requests 123 person months to provide the effort required.

 \bigskip
 \begin{table}[H]
 \begin{tabular}{|r|r|p{8.5cm}|}
 \hline
 \textbf{\site{SRL}} & \textbf{Cost (\euro)} & \textbf{Justification} \\\hline
 \textbf{Travel}
   &  93000 & Travel (see the guidelines \ref{sect:budget-details-travel})\\\hline
 \textbf{Publication charges}
   &  3000 & Open access publication charges (see \ref{sect:budget-outreach-publication-charges})\\\hline
 %%\textbf{Equipment}
 %%  &   0 &  \\\hline    %\taskref{WP-ID}{TASK-ID}
 \textbf{Other goods and services}
   & 183,500 & Cloud computing (180k) and audit (3k)
  \\\hline   %\taskref{WP-ID}{TASK-ID} \delivref{WP-ID}{DELIV-ID}
 \textbf{Total}
  & XXX\\\cline{1-2}
 \end{tabular}
 \caption{Overview: Non-staff resources to be committed at CNRS (all in \texteuro)}\vspace*{-1em}
 \end{table}

%%  %%%%%%%%%%%%%%%%%%%%%%%%%%%%%%%%%%%%%%%%%%%%%%%%%%%%%%%%%%%%%%%%%%%%%%%%%%%%%%
%%  \paragraph{Resources Facility}
%%
%%  \site{...} requests
%%  X person months for
%%
%%  \taskref{wpid}{taskid}
%%
%%
%%  \bigskip
%%  \begin{table}[H]
%%  \begin{tabular}{|r|r|p{8.5cm}|}
%%  \hline
%%  \textbf{2: \site{SRL}} & \textbf{Cost (\euro)} & \textbf{Justification} \\\hline
%%  \textbf{Travel}
%%    &  XXX & Travel (see the guidelines \ref{sect:budget-details-travel})\\\hline
%%  \textbf{Publication charges}
%%    &   XXX & Open access publication charges (see \ref{sect:budget-outreach-publication-charges})\\\hline
%%  %%\textbf{Equipment}
%%  %%  &   0 &  \\\hline    %\taskref{WP-ID}{TASK-ID}
%%  \textbf{Other goods and services}
%%    & XXX &
%%   \\\hline   %\taskref{WP-ID}{TASK-ID} \delivref{WP-ID}{DELIV-ID}
%%  \textbf{Total}
%%   & XXX\\\cline{1-2}
%%  \end{tabular}
%%  \caption{Overview: Non-staff resources to be committed at CNRS (all in \texteuro)}\vspace*{-1em}
%%  \end{table}
%%
