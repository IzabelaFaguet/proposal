\begin{draft}
\section*{Outline of Project (for Proposers)}

\TODO{This is the place for various READMEs not included in the final submission}

\subsection*{Vision}

An internal attempt at specifying our vision through short
(unsubstantiated) answers.

\begin{verbatim}

> 1) Who are we?

We are researchers, educators, and developers in and around the Jupyter community
at top research institutions and SMEs throughout Europe.


> 2) What is our goal?

To improve the accessibility of EOSC resources to researchers and the general public,
and improve the accessibility, interactivity, and reproducibility of computational research.


> 2.5) What is our strategy?

Our strategy has four core components:

1. Develop the Jupyter and Binder infrastructure, to improve its utility, maturity, and bring it to broader communities.
2. Operate a European Binder service, with the goal of eventually running on EOSC
3. Demonstrate that these services have value to research communities by developing specific applications in domains ranging widely from physics to life sciences to personal data management
4. Develop materials and run training to educate researchers in the ability to do reproducible computational science using the tools we develop, among others

> 3) From where do we start?

Jupyter is widely adopted in numerous communities and used by millions of researchers worldwide.
JupyterHub is a system for building hosted Jupyter services which is rapidly maturing
and being deployed at numerous universities, businesses, and research institutions.
repo2docker is a tool for automatically building reproducible computational environments using best practices for environment specifications and community standard tools such as Docker, conda, and pip.
BinderHub is software for hosting a web service built on repo2docker and JupyterHub,
where individuals can share reproducible environments for immediate and free interaction by readers in their browser.
mybinder.org is an instance of the BinderHub software,
currently serving thousands of users each day.
There are many communities not yet reached by Jupyter,
and Binder must grow beyond a single deployment.


> 4) How do we connect or differ from other projects?

...

> 5) Why are we excellent?

Jupyter has received the prestigious ACM Software System Award in 2017.
Previous winners include UNIX, TCP/IP, and the World Wide Web.

\end{verbatim}

% \subsection*{Mission statement for the grant}

% Our mission is to promote the next generation of community-developed
% open source software, databases, and services adapted to the needs of
% collaborative research in pure mathematics and applications.

% Our research will cover a wide variety of aspects, ranging from
% software development models, user interfaces \TODO{virtual
%   environments?}, deployment frameworks and novel collaborative tools,
% component architecture, design, and standardization of software
% \TODO{system?} and databases, to links to publication, data archival
% and reproducibility of experiments, development models and tools, and
% social aspects.

% It will consolidate Europe's leading position in computational
% mathematics and build on the remarkable success of the ecosystem of
% projects GAP, Python/Sage, Pari, Singular, LMFDB.

\subsection*{Description of the call}

\verbatiminput{call_description}

% \TODO{What do we mean by ``new generation''}.

\renewcommand{\thepage}{\arabic{page}}
\setcounter{page}{1}
\black
\cleardoublepage
\end{draft}

%%% Local Variables:
%%% mode: latex
%%% TeX-master: "proposal"
%%% End:

%  LocalWords:  verbatiminput renewcommand thepage setcounter cleardoublepage
