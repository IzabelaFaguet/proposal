\begin{sitedescription}{XFEL}
  \label{sitedescription:euxfel}

% PIC:
% see: http://ec.europa.eu/research/participants/portal/desktop/en/orga

% See ../proposal.tex, section Members of the Consortium for a
% complete description of what should go there

  European X-Ray Free-Electron Laser Facility GmbH is a limited
  liability company under German law. At present, 12 countries are
  participating in the project: Denmark, France, Germany, Hungary,
  Italy, Poland, Russia, Slovakia, Spain, Sweden, Switzerland, and the
  United Kingdom.  The company is in charge of the operation and
  construction of the European XFEL, a 3.4 km long X-ray free-electron
  laser facility extending from Hamburg to the neighbouring town of
  Schenefeld in the German federal state of Schleswig-Holstein. Civil
  construction started in early 2009, and the user operation in
  September 2017. With its repetition rate of 27,000 pulses per second
  and a peak brilliance a billion times higher than that of the best
  synchrotron X-ray radiation sources, the European XFEL will allow
  the investigation of still open scientific problems in a variety of
  disciplines (physics, structural biology, chemistry, planetary
  science, study of matter under extreme conditions and many others).

  European XFEL has a data policy in place \cite{datapolicy-euxfel}
  which opens up facility data for open access after an embargo period
  of 3 years.
\subsubsection*{Curriculum vitae}

% Curriculum of the personnel at this institution
%
\begin{participant}[type=leadPI,PM=4,gender=male]{Hans Fangohr}
  % type is one of:
  % - leadPI: leader of the participating institution
  % - PI: Principal Investigator
  % - R: researcher?
  % Who is the coordinator is specified elsewhere

  % PM=YYY:
  % A fair evaluation of the number of months you will be
  % spending on this specific project along the four years.
  % Typical numbers:
  % - full time hired personnel: 48 months
  % - lead PI or proposal coordinator: 8-12 months
  % - PI: 4-5 months
  % - participant: 2-6 months

  % salary=ZZZ:
  % Approximate monthly gross salary (in term of total cost for the
  % employer). This is optional. If you are uncomfortable having this
  % information in a public file, you can alternatively send the
  % information to Eugenia Shadlova, or to your institution
  % leader/manager if he is willing to fill in himself the budget
  % forms on the eu portal.

  % The above information is used to fill in various tables in the
  % proposal file, and to evaluate the cost of the project for the
  % institutions.

  % You may remove all those comments.

  % About half a page of free text; for whatever it's worth, you may see
  % Nicolas.Thiery.tex for an example.



  \medskip Hans Fangohr is an academic at the University
  of Southampton in the United Kingdom since 2002 (full professor
  since 2010), and leading the data analysis services at European XFEL
  in Germany since 2017.

  He has been a long term proponent of Open Science, and in particular
  involved with the the use and further development of the Jupyter
  Notebook to enable this. He has hosted Thomas Kluyver at the
  University of Southampton since 2015 from where he contributed as a
  core developer of the Jupyter team. As a PI in the EC-funded e-INFRA
  OpenDreamKit project (2015-2019), he has pushed forward the use of
  Jupyter Notebooks for reproducible computational science, and
  started the notebook validation tool (NBVAL). He made use of the
  Jupyter Ecosystem for research and education at graduate and
  postgraduate level at the University of Southampton, and shared
  resources widely, including a text book provided through Jupyter
  Notebooks, which can be executed interactively online [1].

  Since 2017, he is designing data analysis services and
  infrastructure at the European XFEL research facility. European XFEL
  is using IPython and the Jupyter Notebook as core utilities in their
  large scale experiment control, data capture and data
  analysis. Within the e-INFRA project PaNOSC (Photon and Neutron
  Science Open Cloud, 2019-2022), he is leader of the Work Package 4,
  which is focused on data analysis services for the EOSC Hub, and the
  use of the Jupyter notebook with its existing features on the EOSC
  hub.

  In this project (BOSSEE), where new capabilities for the Jupyter
  notebook and ecosystem are being designed, Hans' wide experience and
  interaction with different science groups will be beneficial to
  ensure the outcome is of value to open science in many domains. This
  includes him chairing the interdisciplinary computational modelling
  group at the University of Southampton (200 academics, 2008-2017),
  chairing the national EPSRC scientific advisory committee on High
  Performance Computing in the UK (2014-2017) and interacting with a
  large variety of science users at European XFEL in his current role
  to support their data analysis.
\end{participant}

%%% Local Variables:
%%% mode: latex
%%% TeX-master: "../proposal"
%%% End:

\begin{participant}[type=PI,PM=1,gender=male]{Sandor Brockhauser}
  % type is one of:
  % - leadPI: leader of the participating institution
  % - PI: Principal Investigator
  % - R: researcher?
  % Who is the coordinator is specified elsewhere

  % PM=YYY:
  % A fair evaluation of the number of months you will be
  % spending on this specific project along the four years.
  % Typical numbers:
  % - full time hired personnel: 48 months
  % - lead PI or proposal coordinator: 8-12 months
  % - PI: 4-5 months
  % - participant: 2-6 months

  % salary=ZZZ:
  % Approximate monthly gross salary (in term of total cost for the
  % employer). This is optional. If you are uncomfortable having this
  % information in a public file, you can alternatively send the
  % information to Eugenia Shadlova, or to your institution
  % leader/manager if he is willing to fill in himself the budget
  % forms on the eu portal.

  % The above information is used to fill in various tables in the
  % proposal file, and to evaluate the cost of the project for the
  % institutions.

  % You may remove all those comments.

  % About half a page of free text; for whatever it's worth, you may see
  % Nicolas.Thiery.tex for an example.


  \medskip

  Sandor Brockhauser is the head of the Control and Analysis Software
  Group at the European XFEL. He received his M.Sc. in Informatics
  from Technical University of Budapest, Hungary, earned a Ph.D. from
  University of Leoben, Austria, and received his HDR degree in
  physics at the University of Joseph Fourier, Grenoble in
  France.

  Since 2004, when he joined the European Molecular Biology
  Laboratory (EMBL) in Grenoble, France, he worked in Macromolecular
  Crystallography and became the scientist in charge of the Multi-
  Wavelength Anomalous Dispersion Beamline ID14-4 at European
  Synchrotron Radiation Facility (ESRF), France. Between 2013-15, he
  moved to Szeged, Hungary where he joined the Extreme Light
  Infrastructure, ELI-ALPS, and has established and built up its
  Scientific Engineering Division. In the same time, he also
  established the X-ray Crystallography Laboratory at a European
  Center of Excellence, the Biological Research Center, Szeged of the
  Hungarian Academy of Sciences. Moving to the European XFEL in 2016,
  he became responsible for the full control system of the beamlines
  and scientific instruments, Karabo, that enables the integration of
  Experiment Control, Data Acquisition and Analysis. During the last
  two years, Karabo has been deployed, photon beamlines were
  successfully commissioned and two of the initial scientific
  instruments have been put in operation producing 0,5PT of data in 5
  weeks of experiments. Jupyter tools are embedded in the Karabo
  system and European XFEL analysis activities.

\end{participant}

%%% Local Variables:
%%% mode: latex
%%% TeX-master: "../proposal"
%%% End:

\begin{participant}[type=leadPI,PM=1,gender=male]{Krzysztof Wrona}
  % type is one of:
  % - leadPI: leader of the participating institution
  % - PI: Principal Investigator
  % - R: researcher?
  % Who is the coordinator is specified elsewhere

  % PM=YYY:
  % A fair evaluation of the number of months you will be
  % spending on this specific project along the four years.
  % Typical numbers:
  % - full time hired personnel: 48 months
  % - lead PI or proposal coordinator: 8-12 months
  % - PI: 4-5 months
  % - participant: 2-6 months

  % salary=ZZZ:
  % Approximate monthly gross salary (in term of total cost for the
  % employer). This is optional. If you are uncomfortable having this
  % information in a public file, you can alternatively send the
  % information to Eugenia Shadlova, or to your institution
  % leader/manager if he is willing to fill in himself the budget
  % forms on the eu portal.

  % The above information is used to fill in various tables in the
  % proposal file, and to evaluate the cost of the project for the
  % institutions.

  % You may remove all those comments.

  % About half a page of free text; for whatever it's worth, you may see
  % Nicolas.Thiery.tex for an example.



  \medskip Krzysztof Wrona has a background in computer physics. As
  the group leader of IT and Data Management at European XFEL, he is
  in charge of the management of scientific data in the frame of the
  user program of the European XFEL facility. He has more than 15
  years of experience in data storage, processing, and in general IT
  issues.
\end{participant}

%%% Local Variables:
%%% mode: latex
%%% TeX-master: "../proposal"
%%% End:

\begin{participant}[type=R,PM=1,gender=male]{Thomas Kluyver}
  % PM=YYY:
  % A fair evaluation of the number of months you will be
  % spending on this specific project along the four years.
  % Typical numbers:
  % - full time hired personnel: 48 months
  % - lead PI or proposal coordinator: 8-12 months
  % - PI: 4-5 months
  % - participant: 2-6 months

  % salary=ZZZ:
  % Approximate monthly gross salary (in term of total cost for the
  % employer). This is optional. If you are uncomfortable having this
  % information in a public file, you can alternatively send the
  % information to Eugenia Shadlova, or to your institution
  % leader/manager if he is willing to fill in himself the budget
  % forms on the eu portal.
  Thomas Kluyver is one of the core maintainers of the Jupyter and IPython
  projects, which form a central part of this proposal.
  He has been a contributor to these projects since 2011,
  and is closely familiar with key parts of the code.
  Besides Jupyter, Thomas has contributed to a wide range of open source
  software projects in the Python ecosystem, ranging from other scientific
  computing tools such as h5py to packaging utilities such as Flit, as well
  as improvements in the Python standard library.
  
  Thomas is part of the Data Analysis team at European XFEL.
  He has a specific remit to facilitate the use of Jupyter by internal groups
  and visiting researchers,
  but he also contributes to the general software engineering effort working to
  allow convenient, reproducible analysis of data collected at European XFEL.

  Before working as a software engineer, Thomas studied plant biology, gaining
  a PhD from the University of Sheffield in 2013. His work since then has
  continued in close connection with research, at the University of California,
  Berkeley, and the University of Southampton, before joining European XFEL.
\end{participant}

%%% Local Variables:
%%% mode: latex
%%% TeX-master: "../proposal"
%%% End:

%


\begin{participant}[PM=68, type=R]{NN}

We will hire two postdoctoral-level research software engineers (for 68 person
months in total) to carry out the required work for this projet at
European XFEL. They will work under supervision of Hans Fangohr, with
support from Sandor Brockhauser and Krzysztof Wrona for particular
aspects. The employees will either have a scientific background and
significant software engineering expertise, or an education in
computer science and an aptitude to work with scientists on
computational science and data science problems.
\end{participant}

\subsubsection*{Publications, products, achievements}

\begin{compactenum}
\item H.Fangohr, Python for Computational Science and Engineering
  (2018) DOI: 10.5281/zenodo.1411868 \newline
  https://github.com/fangohr/introduction-to-python-for-computational-science-and-engineering
\item H.Fangohr et al., “Data Analysis support in Karabo at European
  XFEL”, Proceedings of International Conference on Accelerator and
  Large Experimental Physics Control Systems 2017, ISBN 978-3-95450-
  193-9, Data Analytics, Barcelona, Spain, TUCPA01 (2017) DOI: 10.18429/JACoW-ICALEPCS2017-TUCPA01
\item H.Fangohr.
\emph{A Comparison of \software{C}, \Matlab and \Python as Teaching Languages in Engineering}
Lecture Notes on Computational Science \textbf{3039}, 1210-1217 (2004)
\item T. Kluyver, B. Ragan-Kelley, F. Perez, B. Granger, M. Bussonier, J. Frederic, K. Kelley, J. Hamrick, J. Grout, S. Corlay et al. Jupyter
\emph{Notebooks: a publishing format for reproducible computational workflows} In 20th International Conference on Electronic Publishing. IOS Press, 2016.
\end{compactenum}

\subsubsection*{Relevant projects or activities}

\begin{compactenum}
\item OpenDreamKit (GA No. 676541) Open Digital Research Environment
  Toolkit for the Advancement of Mathematics, participant
\item EOSCpilot (GA No. 739563) The European Open Science Cloud for
  Research Pilot Project, participant
\item PaNOSC (GA No. 823852) Photon and Neutron Open Science Cloud,
  participant
\item CALIPSOplus (GA No. 730872) Convenient Access to Light Sources
  Open to Innovation, Science and to the World, participant
\item ATTRACT (GA No. 777222) breAkThrough innovaTion pRogrAmme for a
  pan-European Detection and Imaging eCosysTem, participant

\end{compactenum}




\end{sitedescription}



%KEY-MORE-TODOS



%%% Local Variables:
%%% mode: latex
%%% TeX-master: "../proposal"
%%% End:

%  LocalWords:  sitedescription Programme organisations programmes Centres subsubsection
%  LocalWords:  micromagnetic Nmag Fischbacher Franchin Bordignon Fangohr emph textbf
%  LocalWords:  Multiphysics summarised Iridis TFlops Modelling
