\begin{sitedescription}{QS}

\par QuantStack was founded in 2016 by a team of developers and maintainers of key packages of the open-source scientific computing stack. QuantStack provides support and custom development services in the Jupyter and Scientific Python ecosystems. Clients and partners of QuantStack range from financial software companies to robotics startups and public research institutions. The team comprises several core developers of Jupyter subprojects and authors of popular scientific computing and visualization software used in both academic and industrial contexts.

\par Beyond Project Jupyter, projects developed at QuantStack include data visualization packages for Jupyter such as bqplot, ipyvolume, ipyleaflet, and ipysheet, as well as Jupyter language kernels such as xeus-cling and xeus-python, and JupyterLab extensions like te draw.io and sidecar. QuantStack is also behind the development of the xtensor framework, a high-level array computing library and C++ dataframe.

\subsubsection*{Curriculum vitae of the investigators}

\begin{participant}[type=leadPI,PM=6,gender=male]{Sylvain Corlay}
  % PM=YYY:
  % A fair evaluation of the number of months you will be
  % spending on this specific project along the four years.
  % Typical numbers:
  % - full time hired personnel: 48 months
  % - lead PI or proposal coordinator: 8-12 months
  % - PI: 4-5 months
  % - participant: 2-6 months

  % salary=ZZZ:
  % Approximate monthly gross salary (in term of total cost for the
  % employer). This is optional. If you are uncomfortable having this
  % information in a public file, you can alternatively send the
  % information to Eugenia Shadlova, or to your institution
  % leader/manager if he is willing to fill in himself the budget
  % forms on the eu portal.
  Sylvain Corlay is the founder and CEO of QuantStack. He holds a PhD in applied mathematics from University Paris VI.

  As an open source developer, Sylvain contributes to Project Jupyter in the area of interactive widgets for the notebook, and is steering committee member of the Project. He also serves as a member of the board of directors for the NumFOCUS foundation, and co-organizes the PyData Paris Meetup, a regular seminar series on open-source scientific computing.

  Sylvain was one of the 15 core Jupyter developers to receive the 2017 ACM Software System Award.

  Besides Jupyter, Sylvain authors and contributes to a number of scientific computing open-source projects such as bqplot, xtensor and ipyleaflet.

  Prior to founding QuantStack, Sylvain was a quant researcher at Bloomberg and an adjunct faculty member at the Courant Institute and Columbia University.
\end{participant}

%%% Local Variables:
%%% mode: latex
%%% TeX-master: "../proposal"
%%% End:

\begin{participant}[type=R,PM=0,gender=male]{Johan Mabille}
  % PM=YYY:
  % A fair evaluation of the number of months you will be
  % spending on this specific project along the four years.
  % Typical numbers:
  % - full time hired personnel: 48 months
  % - lead PI or proposal coordinator: 8-12 months
  % - PI: 4-5 months
  % - participant: 2-6 months

  % salary=ZZZ:
  % Approximate monthly gross salary (in term of total cost for the
  % employer). This is optional. If you are uncomfortable having this
  % information in a public file, you can alternatively send the
  % information to Eugenia Shadlova, or to your institution
  % leader/manager if he is willing to fill in himself the budget
  % forms on the eu portal.

  Johan Mabille is a scientific software developer at QuantStack specializing in high-performance computing in C++. He holds master's degree in computer science from Centrale-Supelec.

  As an open source developer, Johan coauthored xtensor and xeus , and is the main author of xsimd. Prior to joining QuantStack, Johan was a quant developer at HSBC.

\end{participant}

%%% Local Variables:
%%% mode: latex
%%% TeX-master: "../proposal"
%%% End:

\begin{participant}[type=R,PM=0,gender=male]{Martin Renou}
  % PM=YYY:
  % A fair evaluation of the number of months you will be
  % spending on this specific project along the four years.
  % Typical numbers:
  % - full time hired personnel: 48 months
  % - lead PI or proposal coordinator: 8-12 months
  % - PI: 4-5 months
  % - participant: 2-6 months

  % salary=ZZZ:
  % Approximate monthly gross salary (in term of total cost for the
  % employer). This is optional. If you are uncomfortable having this
  % information in a public file, you can alternatively send the
  % information to Eugenia Shadlova, or to your institution
  % leader/manager if he is willing to fill in himself the budget
  % forms on the eu portal.
  \par Martin Renou is a Scientific Software Developer at QuantStack. Prior to joining QuantStack, Martin also worked as a Software developer at Enthought. He studied at the French Aerospace Engineering School ISAE-Supaero, with major in autonomous systems and programming.

  \par As an open source developer, Martin has worked on a variety of projects, such as SciviJS (a JavaScript 3-D mesh visualization library) and simphony-remote (a web-service allowing to run desktop applications like Mayavi on the browser).

  \par Passionate about 3-D rendering, Martin has also developed an open source 3-D Chess GUI based on OpenGL during his spare time.

  \par Martin is the main author of xleaflet and is currently working on xtensor.
\end{participant}

%%% Local Variables:
%%% mode: latex
%%% TeX-master: "../proposal"
%%% End:

\begin{participant}[type=R,PM=0,gender=male]{Wolf Vollprecht}
  % PM=YYY:
  % A fair evaluation of the number of months you will be
  % spending on this specific project along the four years.
  % Typical numbers:
  % - full time hired personnel: 48 months
  % - lead PI or proposal coordinator: 8-12 months
  % - PI: 4-5 months
  % - participant: 2-6 months

  % salary=ZZZ:
  % Approximate monthly gross salary (in term of total cost for the
  % employer). This is optional. If you are uncomfortable having this
  % information in a public file, you can alternatively send the
  % information to Eugenia Shadlova, or to your institution
  % leader/manager if he is willing to fill in himself the budget
  % forms on the eu portal.

  \par Wolf Vollprecht is a scientific scientific software developer at QuantStack. He finished his Master in Robotics, Systems and Controls at ETH Zurich in 2017 with a specialization in AI and Deep Learning.

  \par During his thesis work at Stanford University he was involved in developing fast machine learning algorithms on Tensorflow to anticipate human driver behavior.

  \par His current work focuses on making xtensor faster, and more useful in the context of robotics and machine learning.

\end{participant}

%%% Local Variables:
%%% mode: latex
%%% TeX-master: "../proposal"
%%% End:


% For other to-be-hired person, please include here something like:
\begin{participant}[type=res,PM=41,salary=7000]{NN} %% Salary: standard SME cost
  We will hire a software engineer with experience working in large open-source
  projects. They will benefit from the mentoring of the other Jupyter contributors
  of the QuantStack team.
\end{participant}

\subsubsection*{Publications, products, achievements}

\begin{compactenum}

\item QuantStack developers participate in the continuous development of \emph{Project Jupyter}. The team is especially active in the area of interactive widgets, as well as JupyterLab and the Jupyter Server.

\item QuantStack is the main driving force behind the \emph{xtensor} project, a C++ tensor expression system for high-performance computing. Xtensor comes along with language bindings for Python, R, and Julia, as well as interfaces to BLAS, FFTW, and means to input and output a large number of standard file formats.

\item QuantStack also develops the \emph{xeus} project, a framework for creating Jupyter language kernels. Xeus is used as a foundation for the C++ Jupyter kernel "xeus-cling", built upon the Cling C++ interpreter from CERN. Xeus was also adopted in Kitware's \emph{Slicer} medical imaging software for its Jupyter integration.

\item The QuantStack team includes the authors and maintainers of some of the most popular Jupyter interactive widgets packages, including \emph{bqplot}, a 2-D interactive plotting system, \emph{ipyvolume}, a 3-D volume rendering package, \emph{ipyleaflet}, a maps visualization toolkit.

\item QuantStack contributes extensively to the \emph{conda-forge} project, a community-maintained collection of packages for scientific computing. Nearly a hundred "recipes" for conda-forge are maintained by QuantStack.

\item QuantStack developers are also behind the \emph{vaex} data decimation engine for interactive visualization of large datasets.

\end{compactenum}

\subsubsection*{Previous projects or activities}

\par Beyond open-source scientific computing development, QuantStack promotes scientific open source software development through the organization of events and by volunteering in non-profit organizations promoting the ecosystem.

\begin{compactenum}

\item QuantStack team members co-organize the regular \emph{PyData Paris Meetup}, a free event series taking place every two to three months. After a year, the group counts over two thousand members in Paris.

\item We also support the \emph{NumFOCUS Fondation} as volunteers as a member of the team is a member of the board of directors of the foundation.

\end{compactenum}

% \subsubsection*{Significant infrastructure}

\end{sitedescription}

%%% Local Variables:
%%% mode: latex
%%% TeX-master: "../proposal"
%%% End:
