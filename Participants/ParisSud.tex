\begin{sitedescription}{UPSUD} \label{desc:ParisSud}

Université Paris-Sud is among the 40 top universities worldwide in the
2013 Shanghai ranking, and is one of the top two French research
universities. With about 27000 students, 1800 permanent faculty
and 1300 permanent research scientists from national research
organisations (CNRS, Inserm, INRA, Inria), it is the largest campus in
France. Since 2006, scientists from the University were awarded two
Fields medals, one Nobel Prize and a number of other national and international prizes
(European Inventor Award 2013, Wolf Prize 2010, Holweck Prize 2009,
Japan prize 2007).  Université Paris-Sud offers a
wide range of qualifications, from the exact sciences to life and health
sciences (including medical practice), legal sciences and economics. 
Research at Université Paris-Sud is an essential part of academic understanding 
and includes research activities with high commercial potential. 
Research contracts and partnership with companies make
Université Paris-Sud a key actor and a major player in French
research.  The University is located partly on the Plateau de Saclay,
the largest cluster of public and private R\&D institutions in France
(with ca. 16000 research staff), and is one of the core members of 
University Paris-Saclay – a world-class university and a
world-renowned research and innovation hub.

In the context of this project, Université Paris-Sud is the
home of one of the largest group of \Sage developers worldwide.
It is a member of the Open Source Thematic Group of the Systematic
Paris Region Systems and ICT Cluster. 
The University also hosts a major research group in Human-Centered Computing
and manages the Digiscope network of high-end visualisation platforms,
which will provide critical assets to the project.

\subsubsection*{Curriculum vitae of the investigators}

\begin{participant}[type=PI,PM=1,gender=male]{Michel Beaudouin-Lafon}

Michel Beaudouin-Lafon (PhD, Université Paris-Sud) is a Professor of Computer Science, classe exceptionnelle, 
at Université Paris-Sud and a senior fellow of Institut Universitaire de France. 
His research interests include fundamental aspects of interaction, novel interaction techniques, 
computer-supported cooperative work and engineering of interactive systems. 
He has published over 180 papers and is a member of the ACM SIGCHI Academy. 
He is the laureate of an ERC Advanced Grant exploring instrumental interaction and information substrates. 
Michel was director of LRI, the laboratory for computer science joint between Université Paris-Sud and CNRS. 
He now heads the Human-Centered Computing lab at LRI and chairs the Computer Science department at Université Paris-Saclay. 
He was Technical Program Co-chair for CHI 2013 (3500 participants), sits on the editorial boards of ACM Books and ACM TOCHI, 
and has served on many ACM committees. He received the ACM SIGCHI Lifetime Service Award in 2015.

\end{participant}

%%% Local Variables:
%%% mode: latex
%%% TeX-master: "../proposal"
%%% End:

\begin{participant}[type=PI,PM=2,gender=female]{Viviane Pons}
  Maître de Conférences at the Laboratoire de Recherche en Informatique, Viviane Pons is a
  young researcher in Algebraic Combinatorics. She defended her thesis in 2013 and has 4
  papers in international journals and 5 communications in international
  conferences, including a talk at PyCon US 2015. 
  She also is in the editorial board of the Journal of Open Source Software.
  Before starting her research career,
  she worked for two years in industry as a Java and web developer.

  She discovered \Sage during her first \Sage Days in 2010 and has since been an active user
  and contributor with 10 (co)authored tickets improving the support of combinatorial
  objects in \Sage. She is heavily involved in the promotion of \Sage, participating in
  \Sage Days and running \Sage introduction tutorials or \Sage presentations at various
  conferences. She is also one of the main developers of the project \software{FindStat}
  dedicated to databases in combinatorics.

  Viviane is leading the very successful Community Building and
  Dissemination work package of the European Research Infrastructures
  project OpenDreamKit (2015-2019), in which 66 events (development
  workshops, training sessions, ...) were organized or coorganized,
  with more than a thousand trainees. Viviane herself organized or
  coorganized several of them, including two week-long workshops
  dedicated to women.
\end{participant}
%%% Local Variables:
%%% mode: latex
%%% TeX-master: "../proposal"
%%% End:

\begin{participant}[type=leadPI,PM=5,gender=male]{Nicolas M. Thiéry}
  Professor at the Laboratoire de Recherche en Informatique, Nicolas M. Thiéry is a senior
  researcher in Algebraic Combinatorics with 18 papers published in international
  journals. Among other things, he is a member of the permanent committee of FPSAC, the
  main international conference of the domain, and has collaborators
  in the US and Canada where he cumulatively spent more than three
  years (Colorado School of Mines, UC Davis, Providence, Montréal),
  and India. He also
  co-organised fourteen international workshops, in particular \Sage Days, and the semester
  long program on ``Automorphic Forms, Combinatorial Representation Theory and Multiple
  Dirichlet Series'' hosted in Providence (RI, USA) by the Institute for Computational and
  Experimental Research in Mathematics.

  Algebraic combinatorics is a field at the frontier between mathematics and computer
  science, with heavy needs for computer exploration. Pioneer in community-developed open
  source software for research in this field, Thiéry founded in 2000 the \SageCombinat
  software project (incarnated as \MuPADCombinat until 2008); with 50 researchers
  in Europe and abroad, this project has grown under
  his leadership to be one of the largest organised community of Sage developers, gaining
  a leading position in its field, and making a major impact on one hundred
  publications\footnote{\url{http://sagemath.org/library-publications-combinat.html},
    \url{http://sagemath.org/library-publications-mupad.html}}. Along the way,
%this occasion
%Thiéry gained a strong community building experience, and
  he coauthored part of the proposal for NSF \SageCombinat grant
  OCI-1147247, and co-organised or taught at a dozen training and
  dissemination actions (workshops, summer schools, etc.), in
  America, Africa, Europe, and India.

  With 150 tickets (co)authored and as many refereed, Thiéry is himself a core \Sage
  developer, with contributions including key components of the \Sage infrastructure
  (e.g. categories), specialised research libraries (e.g. root systems), thematic
  tutorials, and two chapters of the book ``Calcul Mathématique avec \Sage''
  and its English translation.

  Based on this experience, and to tackle the pressing funding needs
  in the ecosystem of open source mathematical software, Thiéry
  initiated and lead the European Research Infrastructures project
  OpenDreamKit \#676541 (2015-2019, 15 sites, 50 participants, 8M€),
  engaging the Jupyter project on board. This in turn increased his
  involvement in using, promoting, and contributing to Jupyter, for
  use in mathematics and education.
\end{participant}
%%% Local Variables:
%%% mode: latex
%%% TeX-master: "../proposal"
%%% End:


\begin{participant}[type=res,PM=21]{NN}
  We will hire a full time experienced software developer to work on
  task~\taskref{math-widgets} and~\taskref{helpdesk} %longtaskref
  under the leadership of Nicolas M. Thiéry.

  The fellow will have a strong software engineering and web
  development experience, ideally in the Python, Javascript, and/or
  Jupyter ecosystem. We further require good communication and team
  working skills, in particular to work in tight collaboration with
  international open-source developer communities.
\end{participant}

\begin{participant}[type=res,PM=12]{NN}
  We will hire a full time experienced software developer to work
  on task~\taskref{collaboration} %longtaskref
  under the leadership of Michel Beaudouin-Lafon.

  The fellow will have a strong software engineering and web
  development experience (HTML/CSS/Javascript), and ideally 
  good knowledge of the Python/Jupyter ecosystems
  and/or collaboration technologies. 
  We further require good communication and team
  working skills, in particular to work in tight collaboration with
  international open-source developer communities.
\end{participant}

% Participation:
% NT: Project Management: 4PM, Component Architecture: 4PM,
%     Training/Dissemination: 2PM, User Interfaces: 2PM
% Dev 1: Component Architecture: 48PM
% Dev 2: User Interface: 24PM, CA: 10PM, HPC: 2PM
% Florent: HPC: 4PM, User Interface: 2PM (Sphinx & co)
% Viviane: Training/Dissemination: 6PM
% Loic: Training/Dissemination: 3PM, User Interfaces: 2PM
% Sam: Training/Dissemination: 6PM
% Project manager: Project management: 24PM
% PHD: ???
% Total:

\subsubsection*{Publications, achievements}

\begin{compactenum}
\item Leadership of the \SageCombinat software project.
\item Coauthoring of the open source book ``Calcul Mathématique avec
  Sage'' and its English translation , the first of its kind
  comprehensive introduction to computational mathematics in \Sage for
  education.
\item Contribution of more than 500 tickets to \Sage.
\item
Michel Beaudouin-Lafon, Olivier Chapuis, James Eagan, Tony Gjerlufsen, Stéphane Huot, Clemens Klokmose, Wendy Mackay, Mathieu Nancel, Emmanuel Pietriga, Clément Pillias, Romain Primet, Julie Wagner (2012). Multi-surface Interaction in the WILD Room, \emph{IEEE Computer}, 45(4):48–56. IEEE Computer Society.
\item
Klokmose, C.N., Eagan J.R., Baader, S., Mackay, M. and Beaudouin-Lafon, M. (2015) Webstrates: Shareable Dynamic Media. In \emph{Proceedings of the 28th annual ACM symposium on User interface software and technology (UIST ’15)}. ACM.
\end{compactenum}

\subsubsection*{Relevant projects or activities}

\begin{compactenum}
\item OpenDreamKit (GA No. 676541) Open Digital Research Environment
  Toolkit for the Advancement of Mathematics, \textbf{coordination}.
\item Hosting or coorganisation of dozens of Sage Days (week-long training and development workshops).
\item \TODO{This is not a ``previous'' project''}
Ongoing ERC Advanced Grant ONE ``Unified Principles of Interaction'' (PI: Michel Beaudouin-Lafon) that develops new user interface concepts, in particular for multi-user, multi-device environments.
\end{compactenum}

\subsubsection*{Significant infrastructure}

\site{UPSUD} hosts a local OpenStack based cloud infrastructure
\software{Cloud@VD} (400 cores) for its personnel. The participants
are regular users of this infrastructure, and in close contact with
its maintainers. As a continuation of the existing deployment of a
JupyterHub service on this infrastructure, \software{Cloud@VD} will be
available to the participants as test bed for deploying Jupyter based
services (see e.g. \taskref{eosc}{jh-bh-deployment}).

\site{UPSUD} also manages the Digiscope (\url{http://digiscope.fr}) network of high-end visualisation platforms and hosts the \software{WILD} and \software{WILDER} platforms, two ultra-high resolution wall-sized displays with motion capture and touch input for conducting research on collaborative human-computer interaction and visualisation of
large datasets.

\end{sitedescription}



\begin{draft}
\vspace{1cm}\TOWRITE{VP}{Complete check list below -- delete completed items if you wish}

\begin{verbatim}
- [ ] checked that sum of person months put into finance request is
  the same as sum of person months associated with the Work Packages
  (in proposal.tex, as defined as part of the \begin{workpackage}"
  command.

  Take into account person months associated with work package 1, time
  of all staff to be hired and work on the project (including
  investigators). Figure 5 helps with a quick check of the sums over
  different work packages.

- [X] completed site specific resource summary in resources.tex,
  including table of non-staff costs. This is compulsory (EU
  regulations) if the non-staff cost exceed 15% of the total cost, and
  is likely to be the case for most of the partners. We ask everybody
  to do it, to be consistent and show transparently how we have
  planned our total budget.

- [X] Have all our tasks a designated lead institution? Check in the
  Work Packages that all the tasks you are involved in have a
  dedicated lead party. If the lead party is "USO", then use:
  \begin{task}[lead=USO]

- [ ] Have all our deliverables a designated lead institution [using
  the 'lead=' key]?

- [X] In the "Members of the consortium section", have we addressed "a
  description of the legal entity and its main tasks, with an
  explanation of how its profile matches the tasks in the
  proposal"? See Entry for Paris-Sud and Southampton as examples.

- [X] In the Members of the consortium section, have we given
  descriptions of all the people we intend to hire (even if we don't
  know who that is yet).

- [ ] Do all our tasks include us in the list of sites involved?
\end{verbatim}
\end{draft}

%KEY-MORE-TODOS


%%% Local Variables:
%%% mode: latex
%%% TeX-master: "../proposal"
%%% End:

%  LocalWords:  sitedescription Paris-Sud organisations Inserm Inria Holweck valorisation
%  LocalWords:  Saclay subsubsection faut formel des projets antérieurs Acronyme titre
%  LocalWords:  agence financement durée Pareil les publi année SageCombinat Calcul avec
%  LocalWords:  Mathématique Logilab Sud texttt Stratuslab Chapuis
