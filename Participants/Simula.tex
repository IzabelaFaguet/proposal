\begin{sitedescription}{SRL}

Dedicated to tackling scientific challenges with long-term impact and of genuine importance to real life, Simula Research Laboratory (Simula) offers an environment that emphasises and promotes basic research. At the same time, we are deeply involved in research education and application-driven innovation and commercialisation.

Simula was established as a non-profit, limited company in 2001, and is fully owned by the Norwegian Ministry of Education and Research. Its research is funded through competitive grants from national funding agencies and the EC, research contracts with industry, and a basic allowance from the state.  Simula's operations are conducted in a seamless integration with the two subsidiaries Simula School of Research and Innovation and Simula Innovation.

At its outset, the laboratory was given the mandate of becoming an internationally leading research institution within select fields in information and communications technology. These fields are (i) communication systems, including cyber-security; (ii) scientific computing, aiming at fast and reliable solutions of mathematical models in biomedicine, geoscience, and renewable energy; and (iii) software engineering, focusing on testing and verification of mission-critical software systems, and on planning and cost estimation of large software development projects. Recent evaluations state that Simula has met its challenge and is an acknowledged contributor to top-level research in its focus areas. Specifically, in the 2012 national evaluation of ICT research organised by the Research Council of Norway and conducted by an international expert panel, Simula received the highest average score (4.67) on a 1-5 scale among all evaluated institutions.  In comparison, the national average was 3.38. Only five of the 62 research groups evaluated were awarded the top grade (5), and two of these five groups are located at Simula.

Simula is currently hosting one Norwegian Centre of Excellence, Centre for Biomedical Computing (2007-2017), and one Norwegian Centre for Research-based Innovation, Certus (2011-2018). In addition, we participate as research partner in another Centre for Research-based Innovation, Centre for Cardiological Innovation (2011-2018), hosted by Oslo University Hospital. These two centre-oriented schemes are the most prestigious funding instruments offered by the Research Council of Norway.

\subsubsection*{Curriculum vitae}

% Curriculum of the personnel at this institution

% \input{CVs/Martin.Sandve.Alnaes.tex}
% \input{CVs/Simula-not-known.tex}
% \input{CVs/First.Last.tex}

\subsubsection*{Publications, products, achievements}

\begin{compactenum}
\item Jupyter, ...
\end{compactenum}

\subsubsection*{Previous projects or activities}

\begin{compactenum}
\item OpenDreamKit
\item Jupyter
\item Binder
\end{compactenum}

\subsubsection*{Significant infrastructure}

The fully owned Simula subsidiary Simula Innovation handles pre-commercial innovation projects, creation and follow-up of company spin-offs, and general support for entrepreneurs.
\end{sitedescription}



%KEY-MORE-TODOS


%%% Local Variables:
%%% mode: latex
%%% TeX-master: "../proposal"
%%% End:

%  LocalWords:  sitedescription Simula Simula commercialisation Certus subsubsection Logg
%  LocalWords:  Mardal Funke Rognes Sci Comput Langtangen FEniCS Aln ae lgaard vspace
%  LocalWords:  TOWRITE emphasises organised
