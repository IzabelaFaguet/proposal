\begin{sitedescription}{SRL}

Dedicated to tackling scientific challenges with long-term impact and of genuine importance to real life, Simula Research Laboratory (Simula) offers an environment that emphasises and promotes basic research. At the same time, we are deeply involved in research education and application-driven innovation and commercialisation.

Simula was established as a non-profit, limited company in 2001, and is fully owned by the Norwegian Ministry of Education and Research. Its research is funded through competitive grants from national funding agencies and the EC, research contracts with industry, and a basic allowance from the state.  Simula's operations are conducted in a seamless integration with the two subsidiaries Simula School of Research and Innovation and Simula Innovation.

At its outset, the laboratory was given the mandate of becoming an internationally leading research institution within select fields in information and communications technology. These fields are (i) communication systems, including cyber-security; (ii) scientific computing, aiming at fast and reliable solutions of mathematical models in biomedicine, geoscience, and renewable energy; and (iii) software engineering, focusing on testing and verification of mission-critical software systems, and on planning and cost estimation of large software development projects. Recent evaluations state that Simula has met its challenge and is an acknowledged contributor to top-level research in its focus areas. Specifically, in the 2012 national evaluation of ICT research organised by the Research Council of Norway and conducted by an international expert panel, Simula received the highest average score (4.67) on a 1-5 scale among all evaluated institutions.  In comparison, the national average was 3.38. Only five of the 62 research groups evaluated were awarded the top grade (5), and two of these five groups are located at Simula.

Simula has hosted one Norwegian Centre of Excellence, Centre for Biomedical Computing (2007-2017), and one Norwegian Centre for Research-based Innovation, Certus (2011-2018). In addition, we participate as research partner in another Centre for Research-based Innovation, Centre for Cardiological Innovation (2011-2018), hosted by Oslo University Hospital. These two centre-oriented schemes are the most prestigious funding instruments offered by the Research Council of Norway.

\subsubsection*{Curriculum vitae}

% Curriculum of the personnel at this institution

\begin{participant}[type=R,PM=28,gender=male]{Benjamin Ragan-Kelley}
  % PM=YYY:
  % A fair evaluation of the number of months you will be
  % spending on this specific project along the four years.
  % Typical numbers:
  % - full time hired personnel: 48 months
  % - lead PI or proposal coordinator: 8-12 months
  % - PI: 4-5 months
  % - participant: 2-6 months

  % salary=ZZZ:
  % Approximate monthly gross salary (in term of total cost for the
  % employer). This is optional. If you are uncomfortable having this
  % information in a public file, you can alternatively send the
  % information to Eugenia Shadlova, or to your institution
  % leader/manager if he is willing to fill in himself the budget
  % forms on the eu portal.

Benjamin Ragan-Kelley is one of the core maintainers and developers
of the Jupyter and IPython projects, and currently leads the JupyterHub
and BinderHub development teams.
He has been a contributor to these projects since 2006,
prior to the establishment of Jupyter as a separate project from IPython.
He is an expert in all levels of Jupyter development,
especially the aspects of deploying Jupyter-based services,
which is the focus of this proposal.
Benjamin will lead \TheProject.

Beyond Jupyter, Benjamin has contributed widely to open source software,
especially in the scientific Python community.
He is a maintainer of numerous scientific packages
in the conda-forge package management system,
building packages used widely in education and research,
such as PETSc, MPICH, and FEniCS.

Benjamin is a Research Engineer in the department of Scientific Computing and Numerical Analysis
at Simula Research Laboratory in Oslo, Norway,
where his primary responsibility is developing and maintaining the Jupyter software ecosystem,
as well as supporting research scientists in diverse fields,
including biomedical computing.

Prior to his current position at Simula,
Benjamin received his Bachelor's degree \textit{Magne cum Laude} in Engineering Physics in 2007
from Santa Clara University and his PhD in Applied Science and Technology
from the University of California, Berkeley in 2013.
He worked as a postdoctoral fellow at Simula Research Laboratory prior
to becoming a permanent Research Engineer.
He was honored along with the rest of the Jupyter steering council
with the 2017 ACM Software System Award for Jupyter.


\end{participant}

%%% Local Variables:
%%% mode: latex
%%% TeX-master: "../proposal"
%%% End:

% \input{CVs/Simula-not-known.tex}
% \input{CVs/First.Last.tex}

\subsubsection*{Publications, products, achievements}

\begin{compactenum}
\item 2017 ACM Software System Award for Jupyter
\item M. Bussonier, J. Forde, J. Freeman, B. Granger, T. Head, C. Holdgraf, K. Kelley, G. Nalvarte, A. Osheroff, M. Pacer et al. Binder 2.0 - Reproducible, interactive, sharable environments for science at scale In Python in Science ConferenceProceedings of the 17th Python in Science Conference. Austin, Texas: SciPy, 2018.
\item J. Forde, T. Head, C. Holdgraf, Y. Panda, G. Nalvarte, M. Pacer, F. Perez, B. Ragan-Kelley and E. Sundell. Reproducible Research Environments with Repo2Docker In ICML 2018 Reproducible Machine Learning. ICML, 2018.
\item T. Kluyver, B. Ragan-Kelley, F. Perez, B. Granger, M. Bussonier, J. Frederic, K. Kelley, J. Hamrick, J. Grout, S. Corlay et al. Jupyter Notebooks: a publishing format for reproducible computational workflows In 20th International Conference on Electronic Publishing. IOS Press, 2016.

\end{compactenum}

\subsubsection*{Previous projects or activities}

\begin{compactenum}
\item OpenDreamKit -
\item Jupyter - collaboration with UC Berkeley, Cal Poly, funded by Gordon \& Betty Moore Foundation,
      Alfred P. Sloan Foundation, and Helmsley Trust
\item Binder - collaboration with UC Berkeley, funded by Gordon \& Betty Moore Foundation
\end{compactenum}

\subsubsection*{Significant infrastructure}

The fully owned Simula subsidiary Simula Innovation handles pre-commercial innovation projects, creation and follow-up of company spin-offs, and general support for entrepreneurs.

\end{sitedescription}



%KEY-MORE-TODOS


%%% Local Variables:
%%% mode: latex
%%% TeX-master: "../proposal"
%%% End:

%  LocalWords:  sitedescription Simula Simula commercialisation Certus subsubsection Logg
%  LocalWords:  Mardal Funke Rognes Sci Comput Langtangen FEniCS Aln ae lgaard vspace
%  LocalWords:  TOWRITE emphasises organised
