\begin{sitedescription}{SIL}\label{desc:SIL}

The University of Silesia in Katowice was established in 1968. Now,
with 12 faculties and several interdisciplinary schools and centres,
over 30000 students and over 2000 academic staff the University is one
of the largest in Poland. Students are educated at three educational
levels: Bachelor, Master and Doctoral and their achievement are
accumulated using European Credit Transfer and Accumulation System
(ECTS). Located in the heart of Upper Silesia, Poland's old industrial
region with distinct history and cultural identity, the university
attracts many scientists and students.

The origins of the {\em Faculty of Mathematics, Fhysics and Chemistry} date
back to the academic year 1968/1969 and coincide with the
establishment of the University of Silesia. One of the largest
university units, the faculty incorporates, as its name indicates,
three separate departments: mathematics, physics and chemistry, each
with several divisions and subdivisions carrying out the research and
educational activities. There are over 1900 students, both full-time
and part-time, educated at three educational levels: Bachelor`s,
Master`s and Doctoral. The Faculty is entitled to grant doctoral
degrees in the natural sciences. The Faculty staff consists of 243
academics who are both teachers and researchers.


In the context of this project, University of Silesia has started offering notebook based resources for teaching and research since 2011, based on \Sage system. Now it offers courses in science and programming based on Jupyter notebook as well as collaborates with local high schools in this matter. 


\subsubsection*{Curriculum vitae of the investigators}

\begin{participant}[type=leadPI,PM=3,gender=male]{Marcin Kostur}
is an assistant Professor at the Institute of Physics. He is the author of over 50
publication cited over 2000 times in the field of statistical physics,
solid state physics (Josephson Junction dynamics), microfluidics and
biophysics. He is experienced in application of GPU architecture to
numerical simulations of stochastic processed in physics. His recent
computational interests are focused at the Open Source project
\software{Sailfish} -- HPC implementation of Lattice Boltzmann Method on GPU.
He is leader few projects including  computations in the science education and e-infrastructure:
\begin{compactitem}
\item Infrastructure for cloud-based system education: scalable implementation of Jupyter notebook system for scientific explorations, project funded by Erasmus+, Key Action 2 - ``Strategic Partnership'', (budget: \euro{160}k, 2017-2019)
\item Computing in high school science education - iCSE4schools,
  project funded by Erasmus+, Key Action 2 - ``Strategic Partnerships'',
  (budget: \euro{263}k, 2014-2017)
\item ``Computers in Science Education: iCSE'' http://icse.us.edu.pl
  (budget: \euro{1}m, funded by EFS, 2011-2014)
  
  \item  PAAD (Platform for Analysis and Archiving of Data) project funded by POIG program for 2014-2015 with a total budget
  of \euro{4}m. The task coordinator``Interactive HPC services for science''. 
\end{compactitem}
\end{participant}
\begin{participant}[type=PI,PM=5,gender=male]{Jerzy Łuczka}
Prof. Dr. Jerzy Łuczka (\url{http://zft.us.edu.pl/luczka}) is
a full professor of physics at the University of Silesia (Katowice,
Poland) and the Head of the the Department of Theoretical Physics.

He published more than 150 papers in journals  which have been cited almost 3000 times.

He is an Editor of European Physical Journal B, Chairman of the
Statistical and Nonlinear Physics Division (European Physical
Society), Fellow of the Institute of Physics (United Kingdom) and
Outstanding Referee (American Physical Society). He was Co-director of
the NATO Advanced Research Workshop ``Stochastic Systems. From
randomness to complexity'', 2002, Erice (Italy) and Member of the
Steering Committee of the program : ``Stochastic Dynamics: Fundamentals
and applications'' (European Science Foundation), 2003-2008.  He
received the DAAD research fellowship (Forschungsaufenthalte für
Hochschullehrer und Wissenschaftler) 1995, 2009 and 20012. He was a
leader of several Polish and two German-Polish grants. He has
collaborators in Germany, Italy and Spain. He has also co-organised
international conferences.

Łuczka’s research interests lie in areas of stochastic processes in
physics, quantum open systems, transport phenomena, physical
fundamentals of quantum information. He has teaching experience with
\Sage in physics, biophysics and econophysics.


\end{participant}

\begin{participant}[type=res,PM=15,salary=2500]{NN}
    We will hire a part time researcher with strong programming skills to work on task~\taskref{applications}{application-gpu} under leadership of Marcin Kostur. 
The fellow will have a strong knowledge of GPU comuting as well as 3d data visualisation. 
We further require good communication and team working skills, in particular to work in tight collaboration with international open-source developer communities.
\end{participant}

\subsubsection*{Publications, products, achievements}

\begin{compactenum}
\item Leadership on development K3D-jupyter project which is an 3d visualisation Juypyter widget, (\url{https://github.com/K3D-tools/K3D-jupyter})
\item Leadership on development Sailfish-cfd which is an GPU implementation of the lattice Boltzmann method. (\url{https://github.com/sailfish-team/sailfish})\cite{januszewski2014sailfish}
\item Marcin Kostur has received the Award of the Minister of Science and Higher Education for implementing "Computers in Science Education" programme.
\item The project  Computing in high school science education - iCSE4schools, has received an award of Foundation for the Development of the Education System.


\end{compactenum}

\subsubsection*{Relevant projects or activities}

\begin{compactenum}
\item OpenDreamKit (GA No. 676541) Open Digital Research Environment
  Toolkit for the Advancement of Mathematics, (site leader)
  \item Infrastructure for cloud-based system education: scalable 
  implementation of Jupyter notebook system for scientific explorations, project funded by Erasmus+, Key Action 2 - ``Strategic Partnership'', (budget: \euro{160}k, 2017-2019)
  \item Computing in high school science education - iCSE4schools,
    project funded by Erasmus+, Key Action 2 - ``Strategic Partnerships'',
    (budget: \euro{263}k, 2014-2017)
  \item ``Computers in Science Education: iCSE'' http://icse.us.edu.pl
    (budget: \euro{1}m, funded by EFS, 2011-2014)
  \item 2011-2014 - iCSE (innovative Computing in Science Education) -
      \euro 1m grant from European Social Fund, incorporating
      computational perspective in teaching of mathematics, physics and
      chemistry using cloud based \Sage system and \Python language.
    \item 2014-30.11.2015 PAAD (Platform for data analysis and archiving) 
    \euro 3.8m, funded is mostly HPC centre for research with
      interactive access based on web based notebook UI.
    \item 2014-30.11.2015 CNS: Centre of Applied Science - 2nd stage,
      Infrastructure grant includes \euro 0.5m funding for small HPC and
      cloud infrastructure for education. 

\end{compactenum}

\subsubsection*{Significant infrastructure}

The University of Silesia has finished or currently implements ESF grants
totalling to about \euro 120m for infrastructure, laboratories and
computing centres. New HPC centres created as a part of PAAD and
CNS projects provide necessary hardware for development
and implementation of cloud based research and teaching. In particular 
\site{SIL} hosts a local cloud infrastructure for education avaiable to student of the Faculty of Mathematics, Physics and Chemistry. It contains 320 cores system containing 8 GPU. Moreover there is small heteregenous HPC cluster dedicated to research, containing GPU, high-memory and Xeon Phi nodes. It is now available via Jupyterhub interface. 


\end{sitedescription}
%%% Local Variables:
%%% mode: latex
%%% TeX-master: "../proposal"
%%% End:
