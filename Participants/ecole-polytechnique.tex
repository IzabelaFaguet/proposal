\begin{sitedescription}{EP}

\'Ecole Polytechnique (l'X), is the leading member of Grandes \'Ecoles in France
for science and technology according to all French rankings last year, ranked
2nd best small university in the world by Times Higher Education in 2018, and
16th best European university by QS World University Rankings in 2018. With
fewer than 3000 Bachelor, Masters, and PhD students enrolled at any time, the
institution has produced 4 Fields Medalists, 4 Nobel prize winners, and 3 French
presidents, ranking it 6th worldwide in terms of number of Nobel prize
recipients. L'X is composed of 22 laboratories supporting research in physics,
engineering, mathematics, biology, and chemistry (among others) and is connected
to several French national research institutions such as CNRS, CEA, and INRIA.
L'X prides itself on promoting multidisciplinary research through collaboration
between its many labs and with external, national and international partners and
a strong connection with industry and start-up.

The Mathematics department at Ecole polytechnique has started a reform of their
various teaching offer based on Jupyter for two years and several courses of the
Bachelor program, 2nd and 3rd year of Engineering school and Master program have
already begun relying on a strong use of Jupyter notebooks /
JupyterHub\footnote{\url{http://www.cmap.polytechnique.fr/~massot/Personal_web_page_of_Marc_Massot/MAP551}} and this will continue with a strong support of the Dean of
undergraduate studies and of graduate studies. Besides several software and
research engineers in applied mathematics have been recruited and participate in
this effort, as well as a administration engineer in order to help in terms of
building an infrastructure dedicated to Jupyter with the support of the head of
the Ecole polytechnique in order to disseminate the effort into various other
departments (Physics, Mechanical Engineering, Biology...), where already some
courses are starting based in Jupyter. The link with the computer science club
of students of Ecole polytechnique (Binet R\'eseau) has also been created with
the project and a community is emerging.

\subsubsection*{Curriculum vitae}
% Curriculum of the personnel at this institution. This includes
% to-be-hired people for which there is a tentative candidate.

\begin{participant}[type=R,PM=8,gender=male]{Loic Gouarin}

  Loic Gouarin is Research Engineer in scientific computing at CMAP (Centre de
  Mathématiques Appliquées) at \'Ecole polytechnique. He works on several
  scientific compting open-source projects in different fields like
  Lattice-Boltzmann methods, Stokes solvers for fluid particles interaction,
  adaptive mesh refinment, ...

  He is also director of the ``GdR Calcul'' where his role is to animate the
  scientific and high performance computing community in France, in particular
  by organising conferences, meetings, and seminars. In this context, he
  organises himself 3 to 4 training and development workshops per year, and
  promotes the use of Python and c++ for teaching and research in France.
  
  For several years, he has been very involved in promoting the Jupyter project
  and its use for teaching and research in the French community. He is one of
  the core developers of xeus-cling and he is working on the possibility of
  easily deploying a JupyterHub or BinderHub on academic clouds. He also
  believes that reproducible research is an essential part of promoting new
  computation codes, new numerical methods, ... introduced in related
  publications and therefore uses Jupyter as a first approach to achieve this.
  
  \end{participant}
  
  %%% Local Variables:
  %%% mode: latex
  %%% TeX-master: "../proposal"
  %%% End:
  
\begin{participant}[type=R,PM=4,gender=male]{David Delavennat}

\end{participant}
\begin{participant}[type=R,PM=2,gender=male]{Marc Massot}

    Marc Massot obtained his PhD in Applied Mathematics from Ecole Polytechnique, 
    France, in 1996. After a
    year at Yale University, Department of Mechanical Engineering, he obtained a
    CNRS position in the Applied Mathematics Laboratory of the University of Lyon,
    France, where he stayed until 2005. He was offered an Associate Professor
    position at Ecole Centrale Paris when he installed a mathematics team in the
    EM2C mechanical engineering laboratory. From 2008 to 2010, he had the responsibility of 
    structuring scientific computing at INSMI (French National Mathematics Institute of CNRS) 
    at a national level in the French Mathematical Community. He was Visiting Professor 
    for one year at the Center for Turbulence Research, Stanford University, in 2011-2012 
    and created and chaired the Fédération de Mathématiques de l’Ecole Centrale Paris
    between 2013 and 2016. He initiated the Computing Center (Mésocentre) of Ecole
    Centrale Paris in 2010, of which he was the deputy director until 2016 and he
    is scientific adviser at ONERA DMPE and scientific collaborator at Maison de la
    Simulation since 2013. Full Professor since 2011, he has been recruited on a
    full Professor position at Ecole Polytechnique, Centre de Mathématiques
    Appliquées in 2017 and co-chairs the Initiative HPC@Maths, which foster the interaction of mathematics, 
    scientific computing and HPC with industry and in particular SMEs. 
    
    His main fields of
    research are mathematical modeling and numerical analysis, analysis of PDEs and
    dynamical systems for multi-scale systems, scientific computing and high
    performance computing with applications in combustion, two-phase flows, plasma
    physics and biomedical engineering. 
    
    Since his arrival at Ecole polytechnique, he has been very involved 
    in promoting the Jupyter project, creating courses entirely relying on 
    Jupyter notebooks (such as "Dynamical systems for the modeling and simulation 
    of multi-scale reacting media" MAP551) and making the link with the various departments
    and students of Ecole polytechnique, as well as Paris-Saclay community. 
    One of the leaders of the new Computing Center
    in the process of being created, he is also involved with the direction of Ecole polytechnique
    for the project of building an infrastructure for Jupyter at the level of the school, for its use in 
    both research and teaching.

\end{participant}
\begin{participant}[type=R,PM=2,gender=male]{Laurent Series}

    Laurent Series is Research Engineer in scientific computing at CMAP (Centre 
    de Mathématiques Appliquées) at \'Ecole polytechnique. He works on several
    scientific computing open-source projects in different fields like
    finite element method for solid mechanics, multiresolution for 
    reaction-diffusion equations modeling multi-scale reaction wave ...
  
    He was technical responsible of Computing Center (M\'esocentre) of Ecole
    CentraleSup\'elec from 2009 to 2018. In this context, he organises and 
    participates in the user support team (assistance in porting codes, 
    parallelization, vectorisation, optimisation). Since his arrival at \'Ecole 
    Polytechnique, he is involved in the process of creation of a new Computing 
    Center.
    
    For several years, he has been very involved in promoting the Jupyter project
    by writing Jupyter notebooks for course (such as "Dynamical systems for the 
    modeling and simulation of multi-scale reacting media" MAP551) and by promoting, 
    among researchers, its use to share their work. He is also involved with the 
    direction of \'Ecole polytechnique for the project of building an infrastructure 
    for Jupyter at the level of the school, for its use in both research and teaching.
  
\end{participant}

% For other to-be-hired person, please include here something like:
\begin{participant}[type=res,PM=18,salary=5500]{DevOps Engineer}
  We intend to hire a DevOps engineer for 18 months during the project to be supervised at \'Ecole polytechnique to work on developments defined in Tasks \taskref{core}{jh-bh-conv} and \taskref{eosc}{jh-bh-deployment}.
\end{participant}

\begin{participant}[type=res,PM=18,salary=5500]{Software Engineer}
  We intend to hire a DevOps engineer for 18 months during the project to be supervised at \'Ecole polytechnique to work on developments defined in Task \taskref{ecosystem}{teaching-tools}.
\end{participant}

\subsubsection*{Publications, products, achievements}

\begin{compactenum}
\item D. Delavennat, L. Gouarin, G. Philippon, \emph{Deploying JupyterHub with Kubernetes on OpenStack} \newline
\url{https://blog.jupyter.org/how-to-deploy-jupyterhub-with-kubernetes-on-openstack-f8f6120d4b1}
\item JupyterDay at \'Ecole polytechnique in 2018 \newline
\url{http://www.cmap.polytechnique.fr/~massot/Personal_web_page_of_Marc_Massot/JupyterX.html}
\item L. Gouarin, \emph{C++ course using xeus-cling} \newline
\url{https://github.com/gouarin/cours_cpp_moderne}
\item M.Massot, L. Series, \emph{Systèmes dynamiques pour la modélisation et la simulation des "milieux réactifs" multi-échelles} \newline
\url{http://www.cmap.polytechnique.fr/~massot/Personal_web_page_of_Marc_Massot/MAP551}
\end{compactenum}

\subsubsection*{Previous projects or activities}

\begin{compactenum}
\item OpenDreamKit (GA No. 676541) Open Digital Research Environment
Toolkit for the Advancement of Mathematics, participant
\end{compactenum}

\subsubsection*{Significant infrastructure}

\site{UPSUD} hosts a local OpenStack based cloud infrastructure
\software{Cloud@VD}. \site{EP} invested last year in the purchase of 100 cores for this infrastructure in order to study how to deploy JupyterHub and BinderHub and start to offer to researchers and students this kind of service.
In 2019, \site{EP} will have a new infrastructure based on OpenShift with 200-300 cores dedicated to teaching using Jupyter.

\end{sitedescription}
%%% Local Variables:
%%% mode: latex
%%% TeX-master: "../proposal"
%%% End:
