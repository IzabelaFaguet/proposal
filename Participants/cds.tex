\begin{sitedescription}{CDS}

The Observatoire Astronomique de Strasbourg (ObAS) is a Joint Research Unit
(UMR7550) of the CNRS and of the Université de Strasbourg. ObAS hosts the 
Centre de Données astronomiques de Strasbourg (Strasbourg astronomical Data 
Centre CDS, http://cds.unistra.fr). Since its creation in 1972, the CDS has 
been providing reference services which are widely used by the world-wide 
astronomical community with more than 1 million queries/day on average in 
2017. The CDS is labelled as a Research Infrastructure in the French national
Research Infrastructure Roadmap. Since 2006 CDS has been the coordinator,
on behalf of the CNRS, of all the projects funded by the European Commission
to support the implementation of European Virtual Observatory. 

% PIC:
% see: http://ec.europa.eu/research/participants/portal/desktop/en/organisations/
%
% See ../proposal.tex, section Members of the Consortium for a
% complete description of what should go there

\subsubsection*{Curriculum vitae}

% Curriculum of the personnel at this institution. This includes
% to-be-hired people for which there is a tentative candidate.

%\input{CVs/First.Last.tex}
%\input{CVs/First.Last.tex}

\begin{participant}[type=leadPI,PM=0,gender=male]{Mark Allen}
  % type is one of:
  % - leadPI: leader of the participating institution
  % - PI: Principal Investigator
  % - R: researcher?
  % Who is the coordinator is specified elsewhere

  % PM=YYY:
  % A fair evaluation of the number of months you will be
  % spending on this specific project along the four years.
  % Typical numbers:
  % - full time hired personnel: 48 months
  % - lead PI or proposal coordinator: 8-12 months
  % - PI: 4-5 months
  % - participant: 2-6 months

  % salary=ZZZ:
  % Approximate monthly gross salary (in term of total cost for the
  % employer). This is optional. If you are uncomfortable having this
  % information in a public file, you can alternatively send the
  % information to Eugenia Shadlova, or to your institution
  % leader/manager if he is willing to fill in himself the budget
  % forms on the eu portal.

  % The above information is used to fill in various tables in the
  % proposal file, and to evaluate the cost of the project for the
  % institutions.

  \par Mark Allen is the Director of the Centre de Données astronomiques de 
       Strasbourg (Strasbourg astronomical Data Centre CDS, 
       http://cds.unistra.fr).
    
  \par He is a CNRS director of research at the Observatoire Astronomique de
       Strasbourg,  He has 17 years experience implementing e-Science projects
       in Astronomy within the CDS and EC funded European Virtual Observatory
       (Euro- VO) projects. He is the Chair of the IVOA Executive Board, and 
       has served as the chair of the IVOA Committee for Science Priorities, 
       the chair of the IVOA  Applications Working Group and as IVOA Executive 
       Secretary. As Euro-VO project scientist he has engaged and coordinated 
       astronomical data centres, software developers and scientists in the 
       development and use of the Virtual Observatory framework including 
       support to the astronomy community via leading schools and workshops at 
       the national and European levels. His astronomical interests include 
       active galactic nuclei and comparison of theoretical plasma models to 
       multi-wavelength observations.




\end{participant}

%%% Local Variables:
%%% mode: latex
%%% TeX-master: "../proposal"
%%% End:

\begin{participant}[type=R,PM=3,gender=male]{Thomas Boch}
  % type is one of:
  % - leadPI: leader of the participating institution
  % - PI: Principal Investigator
  % - R: researcher?
  % Who is the coordinator is specified elsewhere

  % PM=YYY:
  % A fair evaluation of the number of months you will be
  % spending on this specific project along the four years.
  % Typical numbers:
  % - full time hired personnel: 48 months
  % - lead PI or proposal coordinator: 8-12 months
  % - PI: 4-5 months
  % - participant: 2-6 months

  % salary=ZZZ:
  % Approximate monthly gross salary (in term of total cost for the
  % employer). This is optional. If you are uncomfortable having this
  % information in a public file, you can alternatively send the
  % information to Eugenia Shadlova, or to your institution
  % leader/manager if he is willing to fill in himself the budget
  % forms on the eu portal.

  % The above information is used to fill in various tables in the
  % proposal file, and to evaluate the cost of the project for the
  % institutions.

  \par Thomas Boch is a Research Engineer in charge of service integration at CDS.

  \par He is the developer of Aladin Lite, an interactive sky atlas running in the browser, used and deployed by more than 50 professional astronomy sites. He collaborated with large agencies (ESA - European Space Agency, ESO - European Southern Observatory) to help them integrate Aladin Lite into their own web portal.

  \par He developed the CDS portal which provides with a single entry point to CDS services. He is also co-author of several Virtual Observatory standards, including HiPS and MOC.

  \par He is actively developing and supervising the development of several Python packages to allow for access and visualisation of astronomical data: mocpy, hipspy, astroquery.cds, ipyaladin (Jupyter widget enabling Aladin Lite in the Jupyter notebook).


\end{participant}

%%% Local Variables:
%%% mode: latex
%%% TeX-master: "../proposal"
%%% End:

\begin{participant}[type=R,PM=2,gender=male]{S\'ebastien Derriere}
  % type is one of:
  % - leadPI: leader of the participating institution
  % - PI: Principal Investigator
  % - R: researcher?
  % Who is the coordinator is specified elsewhere

  % PM=YYY:
  % A fair evaluation of the number of months you will be
  % spending on this specific project along the four years.
  % Typical numbers:
  % - full time hired personnel: 48 months
  % - lead PI or proposal coordinator: 8-12 months
  % - PI: 4-5 months
  % - participant: 2-6 months

  % salary=ZZZ:
  % Approximate monthly gross salary (in term of total cost for the
  % employer). This is optional. If you are uncomfortable having this
  % information in a public file, you can alternatively send the
  % information to Eugenia Shadlova, or to your institution
  % leader/manager if he is willing to fill in himself the budget
  % forms on the eu portal.

  % The above information is used to fill in various tables in the
  % proposal file, and to evaluate the cost of the project for the
  % institutions.

  \par S\'ebastien Derriere works as Astronome Adjoint for the CDS.

  \par He has a long experience in the management of astronomical metadata, and the cross-identification and statistical classification of astronomical sources. He has also been involved in the Virtual Observatory (VO) project, for the definition of standardized vocabularies, and implementation of the Registry for CDS services.

  \par He has contributed to defining and developing some astronomy portals based on widgets, at CDS, or for the ASTRODEEP FP7 project (Grant Agreement n.312725).

  \par He has participated in many technical workshops, VO schools and training events, to disseminate the usage of VO tools to the community. For example, in 2018, he led a tutorial during the ADASS conference, including the usage of a Python notebook (http://cds.unistra.fr/adass2018/), and contributed to the 4th ASTERICS (Horizon 2020,  Grant Agreement n.653477) School. He also created a number of YouTube video tutorials on the usage of CDS services.

\end{participant}

%%% Local Variables:
%%% mode: latex
%%% TeX-master: "../proposal"
%%% End:


% For other to-be-hired person, please include here something like:
% \begin{participant}[type=res,PM=3,salary=5900]{NN}
%  <a _short_ description of the qualifications of whom you want to hire>
% \end{participant}

\subsubsection*{Publications, products, achievements}

\begin{compactenum}
  \item F. Genova, M. G. Allen, C. Arviset, A. Lawrence, F. Pasian, E. Solano, J. Wambsganss, Euro-VO - Coordination of Virtual Observatory activities in Europe, Astronomy and Computing, 2015, Volume 11, p. 181-189
  \item F. Genova et. al 2017, Building a Disciplinary, World‐Wide Data Infrastructure. Data Science Journal, 16: 16, pp. 1–13, DOI: https://doi.org/10.5334/dsj-2017-016
  \item P. Fernique, M. G Allen, T. Boch, A. Oberto, F-X. Pineau, D. Durand, C. Bot, L. Cambresy, S. Derriere, F. Bonnarel, F. Genova, Hierarchical Progressive Surveys - Multi-resolution HEALPix data structures for astronomical images, catalogues and 3-dimensional data cubes.  2015, A \& A, 578, A114 
  \item M. Baumann, T.Boch, New Python developments to access CDS services, Proceedings of The Astronomical Data Analysis Software and Systems conference 2018.
\end{compactenum}

\subsubsection*{Previous projects or activities}

\begin{compactenum}
  \item ESCAPE (EC funded project, 2019-2023, \#824064, INFRAEOSC-0402018)
  \item ASTERICS (EC funded project, 2015-2019, \#653477, Research and Innovation Action)
  \item AENEAS (EC funded project,  2017-2020, \#731016, Research and Innovation Action)
  \item CoSADIE - Collaborative and Sustainable Astronomical Data Infrastructure for Europe (EC funded project \#312559, 2012-2015, CSA)
  \item RDA Europe - the European plug-in to the Research Data Alliance (RDA) (EC funded project \#632756, 2014-2016, CSA)
\end{compactenum}

\subsubsection*{Significant infrastructure}

CDS is data centre for reference astronomy data. CDS runs physical infrastructure for 1Petabyte. Connected to French national network RENATER. Computing power for X-Match and generation of all-sky survey data. Currently running prototype notebook servers.


\end{sitedescription}
%%% Local Variables:
%%% mode: latex
%%% TeX-master: "../pro%%% End:
