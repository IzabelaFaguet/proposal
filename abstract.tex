\begin{abstract}
  The Jupyter notebook and Jupyter ecosystem is of increasing
  importance in computational science and data science, in academia,
  industry and services. In addition to supporting high productivity
  of researchers, they have great potential to push open science
  forward: the notebook provides a complete description of a
  computational and data science study, and the notebook can -- in
  principle -- be turned into a publication, or can be used to provide
  the required computation for a part of a publication, such as a
  figure. In this way, the notebook enables reproducibility of complex
  tasks with hardly any additional effort on the user side (if used
  appropriately). The Binder project allows to execute such notebooks
  in tailored computational environments; an aspect of reproducibility
  that is not widely supported yet. Furthermore, for users wanting to
  connect to a local Jupyter notebook server on their machine, or to
  connect to a server somewhere else on the Internet, the users only
  need a webbrowser to display the notebook locally. Because of these
  characteristics, the Notebook is already planned to become an
  important service on the European Open Science Cloud (EOSC), for
  example through the EOSC-04 funded PaNOSC project.

  In this project, we will extend the capabilities of the Jupyter
  tools and ecosystem (such as Jupyter Lab, Widgets, Binder) to pave
  the way for additional functionality that we envisage have great
  importance for the European Open Science Cloud, and Open Science more
  widely. These include refactoring of Jupyter core services to enable
  the generation of advances, improved GUI-like widgets elements in
  the notebook, a workflow management environment based on Jupyter
  notebook, and an infrastructure providing an archive for reproducible
  and re-usable computational and data science studies.

  Most of the contributing partners have longstanding experience and
  roles in the design and development of the Jupyter ecosystem, and
  provision of these to many users across the globe. Complementary, we
  have integrated partners focussing on the application of the newly
  developed tools from a wide range of disciplines, which can each act
  as demonstrators for the new capabilities; before they are picked up
  more widely through and for EOSC.
\end{abstract}

%%% Local Variables:
%%% mode: latex
%%% TeX-master: "proposal"
%%% End:
