\TOWRITE{ALL}{Proofread 1.4 Ambition pass 2}

\subsection{Ambition}

\eucommentary{1-2 pages}

\eucommentary{-- Describe the advance your proposal would provide beyond the
state-of-the-art, and the extent the proposed work is ambitious. Your answer
could refer to the ground-breaking nature of the objectives, concepts
involved, issues and problems to be addressed, and approaches and methods to be used.\\
-- Describe the innovation potential which the proposal represents. Where relevant, refer to
products and services already available, e.g. in existing
e-Infrastructures.}

BOSSEE's ambition is to improve the global accessibility of scientific tools and data,
enabling collaboration among researchers and between researchers and the public.
The world's computational resources are constantly growing and science is producing ever-more
useful and interesting data.
But how do we enable the European and global communities to make use of that data?
And not just researchers, but the public as well?
Open science is a principle of making research results as broadly accessible and useful as possible.
The first, minimal step for this is making publications free to access.
The second step for computational research is to make code and data publicly accessible,
enabling transparency and facilitating reproducibility and verifiability of results.
But merely making these resources technically available is not the best we can do.
There can be many challenges with software,
such as environment specifications and resource requirements,
that can be an impediment to the transition from `technically available' to `practically useful.'
With the tools of the open source open science community
and the resources of the European Open Science Cloud, we can do better.

The \Jupyter ecosystem consists of a large, global community of developers and researchers producing software
focused on interactive computation and communication,
and is widely used by millions of individuals in numerous scientific fields,
ranging from molecular biology \cite{Wang2016} to materials science \cite{Hughes2014},
astronomy \cite{Baron2017} and climate science \cite{Laken2015,Laken2015b}.
\Jupyter software is permissively licensed under the Berkeley Software Distribution (BSD)
license, allowing anyone to use Jupyter software for free,
and even build derivative commercial products,
as has been done in the cases of Google Colaboratory,
Microsoft Azure Notebooks,
IBM Watson Studio, and others \TOWRITE{cite}.
By contributing to the \Jupyter ecosystem, \TheProject maximises its impact, immediately benefiting the existing large \Jupyter community,
and increasing the likelihood that \TheProject's results will be maintained by the community after the end of formal funding.

When it comes to \Jupyter and Open Science,
we aim to improve the \textit{status quo} by bring the two together:

\begin{itemize}
    \item Improving software in the Jupyter ecosystem to better serve open science.
    \item Enable researchers and the public to better perform and benefit from open science, through software, services, and education.
\end{itemize}

Open Science that truly benefits society must be more than merely technically accessible.
Individuals must be able to find the resources they want and interact with them.
Ideally, they should be able to ask new questions of models and data published by those that came before.
This is where \TheProject fits in.
Excellent research is being performed in all scientific fields,
but making those results practically accessible and engaging to others is a challenge.
\Jupyter notebooks enable publishing code and data in a form that is interactive,
where readers can see code, run it, and see results.
They can then modify the code and produce new data and charts that the first authors may not have considered.
\Jupyter does not solve the software installation problem, however,
which can be significant for scientific software.
For a publication to be truly interactive or reproducible,
it must include a computational environment
(or a sufficiently precise description of one such that it can be recreated)
in order to reliably be able to run for another individual.
Services and tools such as Binder and repo2docker facilitate this.
By publishing a description of the requirements to run the software,
repo2docker is able to recreate a computational environment with everything needed to run the software,
including a \Jupyter environment for interactively exploring the resource.
Binder wraps this in a web service, enabling immediate, free sharing of computational results on the web,
with no requirements of readers other than a web browser.
By deploying Binder or similar services on EOSC infrastructure,
EOSC enables researchers to make their results available to the public,
and enables the public to interact with the science they are funding.

All together, \Jupyter and Binder enable the migration of the open science community from static publication to truly interactive, reproducible publications.

\TOWRITE{NOTES}

points to hit:

\begin{itemize}
    \item practical reproducibility
    \item interactive publications
    \item environments
    \item builds on opendreamkit
    \item jupyterhub
    \item binder + repo2docker
    \item demonstrators
    \item educating the public
\end{itemize}


%%% Local Variables:
%%% mode: latex
%%% TeX-master: "proposal"
%%% End:

%  LocalWords:  eucommentary textsuperscript textregistered textsuperscript specialised
%  LocalWords:  textregistered recomputation textbf textbf rigourous centred flagshsip
%  LocalWords:  subsubsection realisation textit
