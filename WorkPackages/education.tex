\TOWRITE{ALL}{Proofread WP 1 Management pass 1}
\begin{draft}
\TOWRITE{PS (Work Package Lead)}{For WP leaders, please check the following (remove items
once completed)}
\begin{verbatim}
- [ ] have all the tasks in this Work Package a lead institution?
- [ ] have all deliverables in the WP a lead institution?
- [ ] do all tasks list all sites involved in them?
- [ ] does the table of sites and their PM efforts match lists of sites for each task?
      (each site from the table is listed in all relevant tasks, and no site is listed
      only in the table or only at some task)
\end{verbatim}
\end{draft}

\begin{workpackage}[id=education,wphases=0-48,swsites,
  title=Education and Dissemination,
  short=Education,
  lead=INSERM,
  CDSRM=3,
  % EGIRM=4,
  % EPRM=TBC,
  INSERMRM=12,
  QSRM=4,
  SILRM=4,
  SRLRM=12,
  UIORM=12,
  UPSUDRM=12,
  WTTRM=4,
  XFELRM=8,
]
\begin{wpobjectives}
  The objective of this work package is to develop the community at the
  European scale, foster cross team collaboration, spread the
  expertise, and engage the greater community to participate in the
  definition and refinement of the requirements, and the implementation and use of the
  produced solutions. This includes:
 \begin{compactitem}
   \item Ensure awareness of the results in the user community;
   \item Train researchers in best practices for open and reproducible science
   \item Educate the community on the value of open science
   \item Produce training and education material to disseminate the ability to do reproducible computational science using the tools we develop.
   \item Define individual exploitation plans;
 \end{compactitem}
\end{wpobjectives}

% Potential sources of inspiration: ODK's WP2 work package about dissemination:
% PDF: p.36 of https://github.com/OpenDreamKit/OpenDreamKit/raw/master/Proposal/proposal-www.pdf
% Sources: https://github.com/OpenDreamKit/OpenDreamKit/blob/master/Proposal/WorkPackages/DisseminationCommunityBuilding.tex

\begin{wpdescription}

Open science is entirely dependent on researchers adopting open practices.
In \TheProject, we are developing tools to facilitate these practices,
but they only work if researchers actually adopt them.
Going further, it is also clear that open science is not just of value 
to researchers: one of the largest benefits of open science is that it makes 
science accessible to the broader public who may not be members of the 
research community.

To this end, in addition to training researchers, we will also train the public in how to 
make use of the open science research and services facilitated by \TheProject.
This will be done through regular open dissemination and training workshops, as well as
by producing and maintaining material for online courses and documentation.

\TheProject will develop, through WP4, a number of applications and demonstrators that
will be disseminated in different ways.
We will also participate in the concertation activities,
consultations and other meetings and events of the European
E-Infrastructure projects.

All the code, documents, test and build infrastructure produced as
part of the project will be made available as open source.
Open access to all publications resulting from the project will be ensured.


\end{wpdescription}

\begin{tasklist}
% add tasks from task directory here
% template for a task
% each task should be added to exactly one workpackage
% in the workpackage task list
\begin{task}[
  title=Dissemination and communication activities,
  id=website,
  lead=SRL,
  PM=12,
  wphases={0-48},
  partners={XFEL,QS,WTT,SIL,INSERM,UPSUD}
]

% flat copy/paste from similar task in ODK proposal
This task comprises all forms of direct dissemination and public
communication activities such as press releases, creation of the
project web-site including visitor analysis and monitoring tools,
scientific and technical publications, outreach activities
(seminars, keynote talks, media interviews, press releases),
promotion through social media (e.g. Twitter, Facebook, LinkedIn),
creation of advertisement materials such as flyers, posters, and
electronic feeds as well as their distribution. We will use standard
community building technology such as mailing lists, Wikis and
Forums, to ensure dissemination and engagement of the community to
support this. We will also generate press releases at appropriate
moments.

\end{task}

% template for a task
% each task should be added to exactly one workpackage
% in the workpackage task list
\begin{task}[
  title=Training Workshops and community building,
  id=workshops,
  lead=UIO,
  PM=35, % UPSUD: 1PM
  wphases={0-48},
  partners={SRL,XFEL,QS,CDS,WTT,SIL,UPSUD,EP,INSERM}
]
This task will be in charge of:

 
  \begin{compactitem}

   \item Defining and implementing a strategy to enable a shared vision of the Jupyter ecosystem across all the actors from developers, users to every stakeholder: the current misalignment hinders the full exploitation of Open Software practices where co-design is a de facto approach.

For instance, the official Jupyter documentation (https://jupyter.org/documentation) solely reflects the view of developers where the Jupyter ecosystem is defined as a set of software packages (jupyter-core, jupyter-client, kernels, widgets (ipywidgets, ipyleaflet, etc.). The user vision is relegated to examplars (blogs, newsletters, etc.) which inevitably tend to be restrictive but often become de facto standards. This can lead to misconceptions and makes it more difficult for on-boarding novices and new communities.


\item Triggering a cultural change to help under-represented groups to actively participate to the development of open source project to ensure the sustainability of the \TheProject services deployed on EOSC-HUB. 
 

\item Foster Open innovation by collaborating with others from different background and activities (school, universities, industries, journalists, artists, etc.)

  \end{compactitem}
 

To achieve these goals, the following actions/activities will take place:


  \begin{compactitem}
   \item co-design hackathons: the co-design efforts between domain scientists, \TheProject developers and service providers will be carried out at any point in time of the project and will be registered in a co-design register to help for future engagement with new communities of users. To be fully effective,  co-design hackathons will be organized to set the stage, define rules for co-design interactions and more importantly align all actors into a common user-centred vision of \TheProject services and associated development towards a successful EOSC deployment. 


   \item Workshops on Findable, Accessible, Interoperable and Reusable (FAIR) software and data to facilitate the adoption of Open Science best practices (transparent, sharable and collaborative Science): this would not be restricted to the Jupyter ecosystem and will teach users how to make data, lab notes and other research processes freely available, under terms that enable reuse (licensing), redistribution and reproducibility of methods and/or results.

   \item Trainings on how to use \TheProject software and services to fully exploit \TheProject developments for EOSC: develop training materials and organize training events for researchers and the public to enable Open Science and maximise the usefulness of \TheProject developments.

   \item \TheProject Admin trainings: training event for learning on how to deploy \TheProject services such as BinderHub.


   \item Open call for open innovation mini-projects: mentored by \TheProject staff and targeting SMEs, municipalities, journalists, artists, etc.

  \end{compactitem}
 
    (\localdelivref{workshops})
\end{task}

% template for a task
% each task should be added to exactly one workpackage
% in the workpackage task list
\begin{task}[
  title=Online resources for open science,
  id=online-resources,
  lead=INSERM,
  PM=14,
  wphases={0-48},
  partners={SRL,XFEL,QS,CDS,WTT,SIL,UPSUD,EP}
]
  The task includes the following activities
  \begin{compactitem}
  \item The interactive book on applied Stochastic processes in Physics. Unlike classical books in this subject, it will be supplemented by numerical examples of solving real case problems. We will emphasize the
role of high performance computing in solving problems modeled by stochastic differential equations (SDE). Materials will contain examples of using GPU computing in solving SDE. The deliverable of this part will be available via the EOSC hub, and will use HPC hardware available. (\localdelivref{sde-book})
  \end{compactitem}
\end{task}

\begin{task}[
  title=Local Help Desk,
  id=helpdesk,
  lead=UPSUD,
  PM=3PM, % UPSUD=2PM, Simulate=1PM
          % please update according to your site involvement,
  wphases={0-48}, % At Paris-Sud, and unless complemented by other
                  % funding, this will stop after 18 months
  partners={SRL,XFEL,QS,CDS,WTT,SIL,UPSUD,INSERM,EP}
  ]

  Dissemination events and tutorials are very effective tools for
  engaging scientists and giving them the desire to acquire new
  technologies and best practices. The next barrier to adoption comes
  when, back home, the scientists start using them on their daily
  problem. Having access to a local expert -- even for a small amount
  of time -- makes a huge difference, saving on the wasted time and
  frustration on the inevitable rough corners, and getting first hand
  advice and guidance in the rich landscape of available tools that
  could otherwise soon feel overwhelming.

  At several of our sites, our Research Software Engineers will
  dedicate some fraction of their time to deliver such help to the
  local community, experimenting with various formats: help desk hours
  where scientists can drop by to get help; regular meet-ups where
  scientists can reconvene to work on their problems or on-demand
  tutorials with expert supervision and mutual help; in-lab visits to
  the scientists for more in-depth discussions; etc.

  An explicit aim of this task is to foster the creation of
  sustainable Research Software Engineer groups within institutions to
  support their scientists.

  This will be the occasion for our Research Software Engineers to
  witness first hand how users adopt or struggle with the projects
  technologies and services, and escalate the hurdles and barriers to
  adoptions as well as success stories. The sites will keep in close
  contact to exchange on the effectiveness of the various formats, and
  the outcome will be reported on in deliverables
  \localdelivref{report1}, \localdelivref{report2}, \localdelivref{report3}.
\end{task}

\end{tasklist}


\TODO{Choose milestone for each report}
\begin{wpdelivs}
\begin{wpdeliv}[due=18,id=report1,dissem=PU,miles=xxx,nature=R,lead=INSERM]
  {Community building: Impact of development workshops, dissemination and training activities, reporting period 2}
\end{wpdeliv}
\begin{wpdeliv}[due=36,id=report2,dissem=PU,miles=xxx,nature=R,lead=INSERM]
  {Community building: Impact of development workshops, dissemination and training activities, reporting period 2}
\end{wpdeliv}
\begin{wpdeliv}[due=48,id=report3,dissem=PU,miles=xxx,nature=R,lead=INSERM]
  {Community building: Impact of development workshops, dissemination and training activities, reporting period 2}
\end{wpdeliv}
\end{wpdelivs}

\end{workpackage}
%%% Local Variables:
%%% mode: latex
%%% TeX-master: "../proposal"
%%% End:

%  LocalWords:  workpackage wphases wpobjectives wpdescription pageref wpdelivs wpdeliv
%  LocalWords:  dissem mailinglists swrepository final-mgt-rep compactitem swsites ipr
%  LocalWords:  TOWRITE tasklist delivref
