\TOWRITE{ALL}{Proofread WP 1 Management pass 1}
\begin{draft}
\TOWRITE{PS (Work Package Lead)}{For WP leaders, please check the following (remove items
once completed)}
\begin{verbatim}
- [ ] have all the tasks in this Work Package a lead institution?
- [ ] have all deliverables in the WP a lead institution?
- [ ] do all tasks list all sites involved in them?
- [ ] does the table of sites and their PM efforts match lists of sites for each task?
      (each site from the table is listed in all relevant tasks, and no site is listed
      only in the table or only at some task)
\end{verbatim}
\end{draft}

\begin{workpackage}[id=ecosystem,wphases=0-48,swsites,
  title=Developing the Jupyter Ecosystem,
  short=Ecosystem,
  lead=QS,
  % EGIRM=4,
  % INSERMRM=4,
  EPRM=21,
  QSRM=20,
  SILRM=16,
  SRLRM=30,
  % UIORM=4,
  UPSUDRM=2,
  WTTRM=6,
  XFELRM=54,
  EPRM=20,
]
\begin{wpobjectives}
 \begin{compactitem}
   \item develop projects for creating open science services built out of Jupyter components and exploring new models for such services
   \item develop tools for interactive visualization in Jupyter
   \item develop workflows for data science using Jupyter software
 \end{compactitem}
\end{wpobjectives}

\begin{wpdescription}

Open source software in general, and Jupyter in particular,
is developed not as a monolithic application,
but rather as a collection of related components,
which can be assembled in numerous combinations to meet diverse needs.
The Jupyter community is no different.
Jupyter itself is composed of several projects,
but there are even more projects that build on top of Jupyter to create
things like cloud services or data pipelines.
The goal of \TheProject is to facilitate open science through Jupyter,
and this includes working with projects all around the Jupyter ecosystem.
We will focus this work package on developing
Jupyter ecosystem projects with an emphasis on open science.

repo2docker is a project for creating
reproducible environments in which Jupyter notebooks (and other user interfaces) can be run.
It reads a number of common formats to list required software packages,
and prepares a Docker container with those packages installed.
BinderHub is software for operating a web service using repo2docker,
which enables sharing of interactive and reproducible Jupyter (and Rstudio) environments on the web with a single link.
We will develop repo2docker and BinderHub further to meet the needs of the open science community.

Widgets are an extension to Jupyter, which can define new kinds of interactivity.
3D visualisation of data is important to many kinds of science,
and there is lots of room for development of the 3D visualisation landscape in Jupyter.

In addition to the interactive aspects of Jupyter,
notebooks can be used in a "workflows" style,
where job systems run analyses and produce reports,
either on a scheduled basis or triggered by events.
There is a great deal of interest in using notebooks in this way,
and much room for development of tools supporting workflows in data-driven open science.

\end{wpdescription}

\begin{tasklist}
% add tasks from task directory here
% % template for a task
% each task should be added to exactly one workpackage
% in the workpackage task list
\begin{task}[title=Task title,
  id=task-id,
  lead=XXX,
  PM=1,
  wphases={0-48},
  partners={XXX,SRL}
]
  The task includes the following activities
  \begin{compactitem}
  \item ...
    (\delivref{workpackage}{deliv-id})
  \end{compactitem}
\end{task}

\begin{task}[
  title=Further development of repo2docker and Binder,
  id=r2d-and-binder,
  lead=SRL,
  PM=36,
  wphases={0-48},
  partners={XFEL,WTT}
]
  Running someone else's analyses is a particularly difficult problem.
  There are differences between operating systems, versions of installed software and access to the required data sets.
  These challenges mean that is currently considered to be beyond the scope of an expert peer reviewer to verify data science analysis codes before publication.
  BinderHub, part of Project Jupyter, enables one-click running of git repositories.
  BinderHub provides a web interface to the repo2docker tool.

  The task includes the following activities
  \begin{compactitem}
  \item extend repo2docker with support for execution on cloud resources
  \item extend repo2docker with support for execution on HPC resources with Docker support
  \item improved "first use" experience of running repo2docker locally
  \item add support for using archives such as Zenodo as source for repo2docker and BinderHub
  \item define procedures and recommendations for long term reproducibility of repo2docker compatible repositories
  \item create educational material describing repo2docker and benefits to researchers
  \item Enable Openshift based deployments of BinderHub
  \item User surveys about pain points using BinderHub
  \item User authentication in BinderHub
  \end{compactitem}
\end{task}

% template for a task
% each task should be added to exactly one workpackage
% in the workpackage task list
\begin{task}[title=Interactive C++ in Jupyter with XEUS,
  id=xeus-cpp,
  lead=QS,
  PM=12,
  wphases={0-48},
  partners={QS,UPSUD}
]
%\hypertarget{interactive-computing-with-interpreter-c}{%
%\section{Interactive Computing with Interpreter
%C++}\label{interactive-computing-with-interpreter-c}}

\begin{compactenum}
\item
  cling \& xeus-cling

  \begin{compactenum}
  \item
    further development of xeus \& xeus-cling for a fully-fledged
    \textbf{Jupyter} experience with the C++ programming language.
  \item
    improved \textbf{packaging} of cling and related packages for conda
    and other package managers, including full \textbf{windows} support.
  \item
    continued development of C++ \textbf{interactive widgets} and
    backends for ipyvolume, bqplot, ipyleaflet (xvolume, xplot,
    xleaflet), and better testing for the feature parity with
    ipywidgets.
  \item
    production of \textbf{teaching material} for C++ with xeus-cling and
    interactive widgets.
  \item
    integration with read-only notebook viewers such as
    \href{https://github.com/QuantStack/voila}{\textbf{voila}} to
    produce a fully compiled application from a notebook interacting with
    Jupyter frontend components such as interactive widgets.
  \item
    enable different levels of \textbf{compiler optimization} for code
    interpreted by cling.
  \item
    improve the \textbf{magics} plugin system to simplify authoring of
    custom magics for xeus-cling.
  \end{compactenum}
\item
  core xeus library

  \begin{compactenum}
  \item
    improve language bindings for the C++ interactive widgets to enable
    interactive widgets in all xeus-based kernels.
  \item
    enable pluggable history managers, with specialized implementations
    for both SQLite-based history and in-memory history.
  \end{compactenum}
\item
  cling support for common C++ scientific computing packages

  \begin{compactenum}
  \item
    cling offers the possibility to automatically load the runtime of
    libraries upon the inclusion of its headers using special pragmas.
    Several libraries such as \texttt{xtensor}, and \texttt{symengine}
    now make use of this possibility. We propose to generalize this
    approach to the main C++ libraries for scientific computing that may
    accept such pragmas to be included upstream.
  \end{compactenum}
\end{compactenum}

Scientists, educators, and engineers not only use programming languages
to build software systems, but also in \textbf{interactive workflows},
using the tools at hand to explore and reason about problems.

Running some code, looking at a visualization, loading data, and running
more code. Quick iteration is especially important during the
exploratory phase of a project.

While C++ is ubiquitous in scientific computing for close-to-the-metal
performance number crunching, \emph{we lack a good story for interactive
computing in C++.} This hurts the productivity of C++ developers:

\begin{itemize}
\item
  Progress in software projects often comes from
  \textbf{incrementalism}. Obstacles to fast iteration hinder progress.
\item
  his also makes C++ more \textbf{difficult to teach}. The first hours
  of a C++ class are rarely rewarding as the students must learn how to
  set up a small project before writing any code. And then, a lot more
  time is required before their work can result in any visual outcome.
\end{itemize}

The cling interpreter fills the gap of interactivity for the C++
programming language and its use at scale in at LHC proves that C++ can
be a language for interactive scientific computing.

\textbf{The xeus-cling kernel}

The goal of the xeus-cling project is to improve the integration with
Project Jupyter and make C++ a \textbf{first-class citizen} of the
ecosystem.

By leveraging the rich ecosystem of tools built upon Project Jupyter,
which is language agnostic, we can lift C++ from a language reserved to
high-performance performance computing to a high-level language of data
science like Python, R, or Julia.

\textbf{Teaching the C++ programming language}

Since September 2017, the 400 first-year students at Paris-Sud
University who take the ``Info 111: Introduction to Computer Science''
class write their first lines of C++ in a Jupyter notebook, with the
xeus-cling kernel.

The use of project Jupyter for teaching C++ is especially useful for the
first classes where students can focus on the syntax of the language
without distractions such as compiling and linking a program.

The availability of \textbf{Jupyter interactive widgets} in that
environment offers a simple means to obtain a \textbf{visual outcome} in
a few lines of code, for a more \textbf{rewarding learning experience}.
It is also not typical for the C++ programming language.

Finally, the \textbf{cloud hosting} of the environment removes the
hurdle of installing a development environment on a large variety of
student's machines.

\textbf{C++ as a common denominator}

The fragmentation of the ecosystem between the main languages of data
science causes a lot of duplication of work and harms sustainability in
the long run. Often, implementation of standard protocols, file formats,
and reference implementations of numerical methods are duplicated in
each language.

A common denominator of the three main languages of data Science (Julia,
Python, R, forming the Jupyter name) is their ability to call into
natively built libraries. All three interpreters have a clean C API that
can also be used from the C++ programming language.

However, a solid C++ implementation can always be exposed to Julia,
Python and R, making the common implementation more sustainable in the
long run. The static typing of the language may make the initial
implementation less immediate, but for greater stability.

For this reason, we strive to provide solid C++ implementation of

\begin{compactenum}
\item
  standard protocols, such as the Jupyter kernel protocol (with xeus),
  now used to make other kernels than the C++ kernel.
\item
  standard in-memory and file formats such as apache arrow, FITS,
  NetCDF, HDF5.
\item
  data structures, such as N-D arrays and dataframes
\item
  reference implementation of numerical methods.
\end{compactenum}

Reference scientific Python projects such as Sympy have moved their core
to a solid C++ engine (such as symengine), which has then been exposed
to Julia and R.

\end{task}

\begin{task}[
  title=Jupyter Interactive Widgets,
  id=jupyter-widgets,
  lead=QS,
  PM=28,
  wphases={0-48},
  partners={XFEL,SIL}
]

The task includes the following activities:

\begin{compactenum}
\item
  Improvements to the core Jupyter Widgets package

  \begin{compactenum}
  \item
    Improve the testing tools for Widget libraries, including means to
    test messaging, schemas, and actual rendering of widgets with a headless
    browser.
  \item
    Modernize of the \emph{@jupyter-widgets\/base} JavaScript package:
    drop \emph{backbone.js} and adopt a more modern MVC framework, support
    streaming of widgets messages to multiple frontends for future support of
    live collaboration.
  \item
    Iterate on the core \emph{@jupyter-widgets\/controls} package. This may
    involve the adoption of a modern framework such as \emph{React.js} for
    the widget view implementation, rather than the current custom implementation.
  \item
    Create new controls for the core Jupyter widgets package such as token
    inputs, typeahead, and tree views. These common controls tend to be implemented
    in several downstream packages, which causes unnecessary duplication of work
    and harms sustainability.
  \end{compactenum}

\item
  Dealing with large widget state

  \begin{compactenum}
  \item Create a generic mechanism for widgets to refer to external data services
  or companion files rather than storing their state entirely in the notebook format
  for offline view.
  \end{compactenum}

\item
  Simplify the authoring of complex GUIs with Jupyter widgets.

  \begin{compactenum}
  \item
    Provide pre-defined Jupyter widget layouts based on the recently
    introduced CSS Grid Layout, with named areas to which widgets can be
    assigned, such as a centra area, left and right tab bars, footer and
    headers, or layouts including areas for logging results on long-running
    tasks.
  \item
    Experiment with the generation of a Jupyter-widgets based GUI from
    a declarative configuration file, or by introspecting a configuration object.
  \end{compactenum}


\item
  Develop interoperable Jupyter widgets for 3d data visualisation.

  \begin{compactenum}
  \item 
    Adopt existing 3d jupyter widget new mechanisms and API of Jupyter widgets

  \item
    Experiment with large dataset visualization and inspection in Jupyter widgets. This will include work on the flexible split of preprocessing between server and widget and the development of data formats for special use cases. 
    
  \item 
    Improve and standardize the interoperablity of  different components and dataformats. 
    
  \item 
    Create comprehensive documentation of 3d widgets with scientifically relevant examples.

  \item 
    Create developer documentation and engage the community for contributions implementing specialisations of the 3d widget.
  
  \end{compactenum}

\end{compactenum}

\end{task}

% template for a task
% each task should be added to exactly one workpackage
% in the workpackage task list
\begin{task}[
  title=Archiving software for reproducible workflows,
  id=reproducibility,
  lead=XFEL,
  PM=36,
  wphases={0-48},
  partners={XFEL}
]

  Reproducible research can inspire greater confidence in scientific results,
  and make it easier for future research to build on those results.

  Reproducibility is seen as an essential pillar of scientific truth,
  nevertheless there is a real shortcoming of truly reproducible
  research in the areas of computational and data science. In part,
  this is a cultural matter. However, there is also a lack of
  computational e-infrastructure supporting reproducibility.
  \medskip

  Jupyter Notebooks combine explanation with code and output and are
  thus valuable tools for making scientific computing more
  reproducible. However, the code in a notebook invariably relies on
  external code: libraries and programs which are not saved as part of
  the notebook.

  This task concerns ways to record the versions of these tools in use, and to
  make them available for practical reproduction of the computation.

  Binder and its tool repo2docker are a first step in the right
  direction: given the description of a computational environment,
  they allow to create that computational environment as a docker
  container automatically on demand, which in turn allows to execute a
  given notebook within this container environment. By archiving the
  notebooks together with the environment specification, the container
  computation environment can be created on demand. We see a number of
  publications being complemented by such git repositories that allow
  reproducing figures and results from papers by re-executing
  notebooks; often archiving these repositories via the Zenodo
  service.

  \TODO{add citations: LIGO paper, one from Hans' papers}.

  Container technologies, such as Docker, offer exciting possibilities
  for capturing a computational environment, but much of the
  development of these tools is focused on short-term operational
  uses, not long-term preservation.

  There are currently at least two shortcomings in the existing
  repo2docker approach:
\begin{enumerate}
\item the environment specifications need to be written carefully and
  need to explicitly define particular version numbers of operating
  systems, libraries, and software to be used in the
  environment. While there are no guidelines (yet) for best practice
  in writing such specifications, in principle users can do this
  correctly.
\item when repo2docker builds a container environment, it relies on
  the required software being available on the Internet: Commands that
  clone software from Github assume that the software is actually
  available on Github. If a relevant repository disappears (or Github
  disappears), it will be impossible to clone that software from
  there, and this will break the binder execution and thus
  reproducibility. Some environments are specified through
  Dockerfiles, and often start from an Ubuntu Linux distribution
  container, followed by a \texttt{apt-get update} command. This
  command will fail once the age of the specified distribution exceeds
  the support period, and similarly subsequent \texttt{apt-get
    install} commands will fail.
\end{enumerate}

The task includes the following activities:
\begin{compactitem}

\item Literature review and technology exploration: research Binder
  model and horizon scan for related technology to support long term
  reproducibility.

\item Establish and document best practice for Binder use: Create
  public guidelines for building containers for scientific computing
  purposes so that they remain useful in the longer term, building on
  existing technology such as repo2docker. ($\rightarrow$
  \localdelivref{binder-guidelines})

\item Facilitating reproducible creation and long-term archiving of
  container images for reproducibility: Develop new software to
  provide long-term executable computational environments that support
  the Binder model.

  There is a trade-off between preserving binary container images, and
  preserving the source code and instructions to build a container.
  Preserving sources is more transparent, and makes it easier to
  modify the code to explore a result, but without special care, the
  instructions may not continue to work in the future, or may not
  build an equivalent container.  We will explore both approaches,
  with a particular interest in how to make build instructions that
  can still work many years in the future.  ($\rightarrow$
  \localdelivref{jupyter-archive})
  \TODO{Add more detail here or in summary above.}
\end{compactitem}

%\begin{compactitem}
%  \item
%    (\localdelivref{deliv-id})
%  \end{compactitem}

  This techology will be developed with a real scientific use case at European
  XFEL (see WP3). \TODO{Add link to particular task}
  \TODO{Mention other use cases: universities and publishers}

\end{task}

% template for a task
% each task should be added to exactly one workpackage
% in the workpackage task list
\begin{task}[
  title=Teaching tools,
  id=teaching-tools,
  lead=EP,
  PM=20, % 18 for RSE, 2 for supervision
  wphases={0-48},
  partners={EP,UPSUD}]

  The partners will be delivering a large number of courses using
  Jupyter technology. This will require tools for easy sharing,
  collecting, self assessment and semi-automatic grading of course
  material, class management, and integration with the local
  e-learning infrastructure. The variety of use cases and
  infrastructure will provide a rich test bed for the further
  development of tools (nbgrader, okpy, ...) and best practices around
  them.

  \TODO{review existing tools; mention what needs to be done, and
    specific actions to be taken}

  The task includes the following activities:
  \begin{compactitem}
  \item Review and follow up on related efforts: gryd.us, cocalc, fun,
    Berkeley, ...
  \item Collaborative grade management
  \item Insulation through container of the automatic grading
  \item Integration with e-learning platforms (e.g. Moodle, OpenEDX
    (Coursera/Fun)), through an LTI connector.
    %(\localdelivref{deliv-id})
  \item Develop course templates for various use cases
  \item ...
  \end{compactitem}
\end{task}

\end{tasklist}


\begin{wpdelivs}
\begin{wpdeliv}[due=12,miles=startup,id=binder-guidelines,dissem=PU,nature=DEC,lead=XFEL]
  {Guidelines for Binder use to improve reproducibility of
  environments, based on existing technology such as repo2docker}
\end{wpdeliv}
\begin{wpdeliv}[due=12,miles=startup,id=teaching-report,dissem=PU,nature=R,lead=EP]
  {Study of the practices of using Jupyter for teaching and the needs to
  effectively manage classes and associated courses in the education community}
\end{wpdeliv}
\begin{wpdeliv}[due=36,miles=community,id=nbgrader-like,dissem=PU,nature=OTHER,lead=EP]
  {Unified framework to effectively manage classes and associated courses
  using Jupyter technology}
\end{wpdeliv}
\begin{wpdeliv}[due=36,miles=community,id=jupyter-archive,dissem=PU,nature=OTHER,lead=XFEL]
  {Long-term reproducibility: Computational environment software
    archive system that extends lifetime of computational environments
  used in Binder service. TODO: Fix milestone}
\end{wpdeliv}
\begin{wpdeliv}[due=36,miles=community,id=k3d-jupyter,dissem=PU,nature=OTHER,lead=SIL]
  {Implement interoperable 3d visualization widget based on K3D-jupyter code}
\end{wpdeliv}


\end{wpdelivs}
\end{workpackage}
%%% Local Variables:
%%% mode: latex
%%% TeX-master: "../proposal"
%%% End:

%  LocalWords:  workpackage wphases wpobjectives wpdescription pageref wpdelivs wpdeliv
%  LocalWords:  dissem mailinglists swrepository final-mgt-rep compactitem swsites ipr
%  LocalWords:  TOWRITE tasklist delivref
