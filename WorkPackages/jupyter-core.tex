\TOWRITE{ALL}{Proofread WP 1 Management pass 1}
\begin{draft}
\TOWRITE{PS (Work Package Lead)}{For WP leaders, please check the following (remove items
once completed)}
\begin{verbatim}
- [ ] have all the tasks in this Work Package a lead institution?
- [ ] have all deliverables in the WP a lead institution?
- [ ] do all tasks list all sites involved in them?
- [ ] does the table of sites and their PM efforts match lists of sites for each task?
      (each site from the table is listed in all relevant tasks, and no site is listed
      only in the table or only at some task)
\end{verbatim}
\end{draft}

\begin{workpackage}[id=core,wphases=0-48,swsites,
  title=Structural improvements to Jupyter,
  short=Core,
  lead=SRL,
  SRLRM=16,
  UPSUDRM=4,
  XFELRM=4,
]
\begin{wpobjectives}
  \begin{compactitem}
    \item to support and maintain core Jupyter infrastructure in order to keep it healthy
         and useful for open science
    \item to develop new features in the core of Jupyter to bring it to a wider community
    \item to develop new features in the core of Jupyter to make it more effective
         in facilitating open science

 \end{compactitem}
\end{wpobjectives}

\begin{wpdescription}

Community-led open source software is critical to a sustainable future for open science.
Commonly used tools make up a shared infrastructure,
where investment in core components benefits the widest user community.
\TheProject is centered around the Jupyter project,
which is a collection of projects for interactive computing and
communicating computational ideas.

This work package is focused on developing and maintaining
the core of Jupyter.
In particular, we will help maintain these projects to meet the needs of the
Jupyter community, with a focus on needs for open science.
In addition, we will develop new features in the core of Jupyter
to bring it to a wider audience,
and to improve its usefulness to those working toward open science practices,
including via collaboration features
and accessibility.


\end{wpdescription}

\begin{tasklist}

% template for a task
% each task should be added to exactly one workpackage
% in the workpackage task list
\begin{task}[title=Developing core JupyterHub infrastructure,
  id=jupyterhub,
  lead=SRL,
  PM=2,
  wphases={0-48},
  partners={SRL,XFEL}
]
  Developing the core JupyterHub framework for is key to ...
  The task includes the following activities
  \begin{compactitem}
  \item Improve deployments
    (\delivref{core}{jupyterhub-feature})
  \end{compactitem}
\end{task}


% \input{tasks/<name>}

\end{tasklist}


\begin{wpdelivs}
\begin{wpdeliv}[due=1,miles=startup,id=infrastructure,dissem=PU,nature=DEC,lead=SRL]
  {Some Deliverable}
\end{wpdeliv}

\end{wpdelivs}
\end{workpackage}
%%% Local Variables:
%%% mode: latex
%%% TeX-master: "../proposal"
%%% End:

%  LocalWords:  workpackage wphases wpobjectives wpdescription pageref wpdelivs wpdeliv
%  LocalWords:  dissem mailinglists swrepository final-mgt-rep compactitem swsites ipr
%  LocalWords:  TOWRITE tasklist delivref
