\TOWRITE{ALL}{Proofread WP 1 Management pass 1}
\begin{draft}
\TOWRITE{PS (Work Package Lead)}{For WP leaders, please check the following (remove items
once completed)}
\begin{verbatim}
- [ ] have all the tasks in this Work Package a lead institution?
- [ ] have all deliverables in the WP a lead institution?
- [ ] do all tasks list all sites involved in them?
- [ ] does the table of sites and their PM efforts match lists of sites for each task?
      (each site from the table is listed in all relevant tasks, and no site is listed
      only in the table or only at some task)
\end{verbatim}
\end{draft}

\begin{workpackage}[id=core,wphases=0-48,swsites,
  title=Structural improvements to Jupyter,
  short=Core,
  lead=SRL,
  % EGIRM=4,
  % INSERMRM=4,
  QSRM=14,
  % SILRM=4,
  SRLRM=30,
  % UIORM=4,
  UPSUDRM=12,
  WTTRM=6,
  XFELRM=18,
  EPRM=13,
]
\begin{wpobjectives}
  \begin{compactitem}
    \item to support and maintain core Jupyter infrastructure in order to keep it healthy
         and useful for open science
    \item to develop new features in the core of Jupyter to bring it to a wider community
    \item to develop new features in the core of Jupyter to make it more effective
         in facilitating open science

 \end{compactitem}
\end{wpobjectives}

\begin{wpdescription}

Community-led open source software is critical to a sustainable future for open science.
Commonly used tools make up a shared infrastructure,
where investment in core components benefits the widest user community.
\TheProject is centered around the Jupyter project,
which is a collection of projects for interactive computing and
communicating computational ideas.

This work package is focused on developing and maintaining
the core of Jupyter.
In particular, we will help maintain these projects to meet the needs of the
Jupyter community, with a focus on needs for open science.
In addition, we will develop new features in the core of Jupyter
to bring it to a wider audience,
and to improve its usefulness to those working toward open science practices,
including via collaboration features
and accessibility.


\end{wpdescription}

\begin{tasklist}

\begin{task}[
  title=Maintenance of Jupyter and JupyterHub,
  id=maintenance,
  lead=SRL,
  PM=44,
  wphases={0-48},
  partners={XFEL,UPSUD,QS}
]
  Developing software that people will use requires maintenance of that software,
  not just new development.
  Millions of people rely on Jupyter software,
  including all participants in \TheProject,
  and with this proposal, we fund general support of the Jupyter infrastructure.

  Maintenance of core software is often an implicit and un-paid cost,
  or one hidden in over-describing the resources required to deliver
  proposed developments.
  In \TheProject, we make it clear and explicit that we will spend a significant amount
  of time developing and maintaining the core Jupyter and JupyterHub
  e-Infrastructure to respond to the needs of \TheProject and others,
  and help keep it a healthy and active software community.

  We will contribute to supporting Jupyter e-Infrastructure software,
  ensuring that it meets the needs (\localdelivref{jupyter-contributions})
  of \TheProject,
  and aid in the release process to ensure that stable releases
  of Jupyter software can be used in mature \TheProject services
  (\localdelivref{jupyter-releases}).
\end{task}

% template for a task
% each task should be added to exactly one workpackage
% in the workpackage task list
\begin{task}[
  title=JupyterHub / BinderHub convergence,
  id=jh-bh-conv,
  lead=EP,
  PM=15, % EP: 13PM, WTT: 2PM
  wphases={0-48},
  partners={EP,WTT}]

  An institution - typically a university, a national lab, a transnational
  research infrastructure such as the European XFEL, or transnational 
  infrastructure provider like EGI - wishes to provide its members and 
  users with a Jupyter service.

  The service lets user spawn and access personal or collaborative virtual
  environments: namely a web interface to a light weight virtual machine,
  in which they can use Jupyter notebooks, run calculations, etc.

  
  To cater for a large variety of use cases in teaching and research,
  the main aim of the upcoming specifications is to make the service as
  versatile as possible. In particular, it should empower the users to 
  customize the service (available software stack, storage setup, ...),
  without a need for administrator intervention.

  JupyterHub already provides authentication, persistent storage and some
  default environments for its users. On the other hand, BinderHub offers
  the possibility to define more precisely what you need for your teaching
  or research environmment which makes it very flexible. But, unfortunetly,
  it's not possible yet to have authentication and persistent storage with
  BinderHub.

  The purpose of this task is to have the same services offered by JupyterHub
  (authentication, persistent storage, ...) with the flexibility of BinderHub
  (construction of your own environment for teaching or research).

  % Motivation: \TODO{reuse material from the blog post:
  % \url{https://opendreamkit.org/2018/03/15/jupyterhub-binder-convergence/}}
  % \TODO{reuse material from the hackmd notes with Tim:
  % \url{https://hackmd.io/0DLDCXcmRzC_dOwjEak3hg#}}
  The task includes the following activities:
  \begin{compactitem}
  \item Extend where needed JupyterHub's authentication features% (OIDC, ...)
  \item Credential management
  \item Customizable persistence at the admin, and user level
  \item Choice of container registry at the admin and user level % Could be moved to the repo2docker/binder task
  \item Support for authenticated git repo for repo2docker? % Could be moved to the repo2docker/binder task
  \item Runtime resource configuration % (disk, memory, cpus, time, ...)
  \end{compactitem}

  \TODO{Deliverable: report on the JupyterHub/Binder convergence, M18?
    M36? Sooner is better, but only if really feasible.}
\end{task}

\begin{task}[
  title=Accessibility in Jupyter,
  id=task-id,
  lead=SRL,
  PM=12,
  wphases={0-48},
  partners={XXX,SRL}
]
  Improving the accessibility of Jupyter software...

  \begin{compactitem}
  \item ...
    (\localdelivref{deliv-id})
  \end{compactitem}
\end{task}

% template for a task
% each task should be added to exactly one workpackage
% in the workpackage task list
\begin{task}[
  title=Multi-device Real-time Collaboration,
  id=collaboration,
  lead=UPSUD,
  PM=23,
  wphases={0-48},
  partners={SRL,QS}
]
  % For applications such as real-time collaboration and others,
  % it can be beneficial.
  % We will explore different possible mechanisms for moving document state
  % to the server-side in JupyterLab.

Current practices, in research and industry, are often collaborative and would benefit tremendously from the ability to collectively edit notebooks. Unfortunately there is no ``Google doc'' equivalent to a Jupyter notebook that is widely available today. 
Moreover, we routinely use multiple devices to manipulate or present content, although it is often  difficult to juggle devices and apps to transfer content from one device to the next. 
Jupyter already recognizes the distributed nature of the digital environment by supporting remote kernels for computation. 
But the user interfaces are stuck in single devices. 
BOSSEE will embrace this multi-device world and facilitate the distribution and real-time collaborative editing of content across multiple devices for presentation, interaction and collaboration purposes.

We will create a real-time collaborative notebook or JupyterLab-like application by leveraging well-known synchronization techniques such as Operational Transformation (OT) or Conflict-free Replicated Data Types (CRDTs), with server-side hosting of the document state. 
We will build on our experience with Webstrates (\url{http://webstrates.net}), a web-based environment that supports real-time sharing of web content. The CodeStrates extension to Webstrates uses the layout of Jupyter notebooks, and we have created a proof-of-concept showing that it can work with Jupyter kernels. However CodeStrates, and there are interesting unresolved questions about code execution in such a shared environment: Should it be synchronized or not among participants? Should there be a single kernel or one per participant? etc.

We will also enable selective distribution, aggregation and control of content across devices. 
We have used Webstrates to distribute and synchronize content across multiple devices such as a tablet, a laptop and a large wall-sized display. Yet this does not cover all use cases. 
For example, in a meeting, the participants should each be able to run their own notebook and pick which content to share with the group on a large display. 
We will create an environment where a notebook can collect specific cells from another notebook, or where a JupyterLab widget aggregates data from a collection of widgets running on each user’s device. 
We also want to support remote interaction using one device to control another, e.g. a widget on a smartphone to control a parameter in a computation taking place in a particular notebook, whose result is shown on a large shared display.

  % \begin{compactitem}
  % \item ...
  %   (\localdelivref{deliv-id})
  % \end{compactitem}

\end{task}


\end{tasklist}


\begin{wpdelivs}
  % \begin{wpdeliv}[due=1,miles=startup,id=infrastructure,dissem=PU,nature=DEC,lead=SRL]
  %   {Some Deliverable}
  % \end{wpdeliv}

  \begin{wpdeliv}[due=24,miles=prototype,id=jupyter-contributions,dissem=PU,nature=OTHER,lead=SRL]
    {Contributions to core Jupyter and JupyterHub software}
  \end{wpdeliv}

  \begin{wpdeliv}[due=48,miles=final,id=jupyter-releases,dissem=PU,nature=OTHER,lead=SRL]
    {Public releases of core Jupyter and JupyterHub software supporting \TheProject services}
  \end{wpdeliv}

  \begin{wpdeliv}[due=36,miles=community,id=jh-bh-conv-report,dissem=PU,nature=R,lead=EP]
    {Guidelines for a JupyterHub/Binder convergence}
  \end{wpdeliv}

  \begin{wpdeliv}[due=18,miles=prototype,id=accessibility-report,dissem=PU,nature=R,lead=SRL]
    {Report and plan for Jupyter accessibility}
  \end{wpdeliv}

  \begin{wpdeliv}[due=36,miles=community,id=accessibility,dissem=PU,nature=OTHER,lead=SRL]
    {Improved accessibility of Jupyter software}
  \end{wpdeliv}

  \begin{wpdeliv}[due=36,miles=community,id=server-state,dissem=PU,nature=OTHER,lead=UPSUD]
    {Real-time collaborative notebook supporting multiple devices and selective aggregation and distribution}
  \end{wpdeliv}

\end{wpdelivs}

\end{workpackage}
%%% Local Variables:
%%% mode: latex
%%% TeX-master: "../proposal"
%%% End:

%  LocalWords:  workpackage wphases wpobjectives wpdescription pageref wpdelivs wpdeliv
%  LocalWords:  dissem mailinglists swrepository final-mgt-rep compactitem swsites ipr
%  LocalWords:  TOWRITE tasklist delivref
