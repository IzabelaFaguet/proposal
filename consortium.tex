%\TO WRITE{ALL}{Proofread 3.4 consortium pass 2 [Done by Hans]}
%remove this, as we have more pressing things left.

\eucommentary{\begin{compactitem}
\item
Describe the consortium. How will it match the project's objectives?
How do the members complement one another (and cover the value chain,
where appropriate)? In what way does each of them contribute to the
project? How will they be able to work effectively together?
\item
If applicable, describe the industrial/commercial involvement in the
project to ensure exploitation of the results and explain why this is
consistent with and will help to achieve the specific measures which
are proposed for exploitation of the results of the project (see section 2.3).
\item
Other countries: If one or more of the participants requesting EU funding
is based in a country that is not automatically eligible for such funding
(entities from Member States of the EU, from Associated Countries and
from one of the countries in the exhaustive list included in General
Annex A of the work programme are automatically eligible for EU funding),
 explain why the participation of the entity in question is essential to carrying out the project
\end{compactitem}
}

\TODO{Add story:  how did the group come together; Jupyter
  days, preperation of the call: partners have developed they can work
  together, developed over 2 months, emphasise open nature of grant writing}


The BOSSEE consortium spans the broad spectrum of actors required for successfully developing an apt and easy-to-navigate sustainable service accessible through the EOSC hub catering to the needs of the European scientific community. The consortium brings in:
\begin{compactitem}
\item A set of use cases that cover several application domains and users, and that impose very diverse requirements on EOSC infrastructure (XFEL, CDS ASTRO);
\item Lead developers in of the Jupyter Ecosystem, including IPython, the Jupyter Notebook, JupyterLab, JupyterHub, Binder, MyBinder.org, IPyWidgets (name??) located at Simula, European XFEL, QuantStack and WildTreeTech [import here to have the section on "what is the Jupyter ecosystem? somewhere, so we can make reference to it ?...;
\item Experts and major ?promoters? of the JUPYTER collaborative user interfaces for interactive and exploratory computing in a variety of scientific domains (European XFEL, QuantStack, Paris-Sued, Ecole Polytechnique, Simula, Silesia, [others?]).
\item A long experience and proven track record of success with large and complex collaborative projects, including projects focused on large-scale infrastructures and large experimental services (EGI, XFEL?) as well as experience in running large scale open source projects (Jupyter project)
\item A comprehensive range of skill sets and competencies in several relevant domains, from applied research to standardisation to business
analysis.
\end{compactitem}

\TOWRITE{ALL}{description of collaborations}

The project partners know have been interacting through a number of
activities:

\begin{enumerate}
\item Joint software development
  \begin{itemize}
  \item Jupyter Notebook \TODO{Add sites, also for following items}
  \item Widgets (including k3d)
  \item Nbval
  \item \TODO{What else?}
  \end{itemize}

\item Joint projects
  \begin{itemize}
  \item OpenDreamKit (EuXFEL, Simula, UPSud, Silesia)
  \item PaNOSC (EGI and EuXFEL)
  \item \TODO{others?}
  \end{itemize}
\item Joint publications
  \begin{itemize}
  \item \cite{Kluyver2017} (EuXEL, Simula, QuantStack, \TODO{...})
  \end{itemize}

\item Colloboration
  \begin{itemize}
  \item \TODO{Could list informal collobartion activities,
      i.e. advising each other on how to use software, binder,
      widgets. Maybe that's too detailed and too much?}

  \item MyBinder for teaching and reproducibility (EuXFEL, WTT)
  \item Widgets for computational science (EuXFEL, QS, SIL )
  \end{itemize}
\end{enumerate}

Table \TODO{Can we reference its number?} shows a summary of the links
between partners.

\TOWRITE{ALL}{Add previous collaborations}

% joint software/database development
% Jupyter Project software

\jointsoft{XFEL,SRL}
\jointsoft{WTT,SRL}
\jointsoft{WTT,XFEL}

% Binder
\jointsoft{SRL,WTT}

% k3d
\jointsoft{SIL,SRL}
\jointsoft{SIL,XFEL}
\jointsoft{SRL,XFEL}

% nbval
\jointsoft{XFEL,SRL}

%%s

% OpenDreamKit: UPSUD, SIL, XFEL, SRL
\jointproj{XFEL,UPSUD}
\jointproj{XFEL,SRL}
\jointproj{XFEL,SIL}

\jointproj{UPSUD,SRL}
\jointproj{UPSUD,SIL}

\jointproj{SRL,SIL}
\jointproj{UIO,SIL}

% jupyterhub deployment
\jointproj{EP,UPSUD}

% Binder persistent storage
\jointproj{UPSUD,WTT}
\jointproj{EP,WTT}

% xeus-cling
\jointsoft{QS,EP}
\jointsoft{QS,UPSUD}

% panosc
\jointproj{XFEL,EGI}

% Jupyter project publication ? XXX TIM

% Binder
\jointsoft{SRL,WTT}

% research bazaar
\jointproj{SRL,UIO}

% JupyterDays Orsay + ecole
\jointproj{UPSUD,EP}
\jointproj{UPSUD,CDS}
\jointproj{UPSUD,SRL}
\jointproj{UPSUD,QS}

\jointproj{EP,CDS}
\jointproj{EP,SRL}
\jointproj{EP,QS}

\jointproj{CDS,SRL}
\jointproj{CDS,QS}

\jointproj{SRL,QS}

% OpenGATE
\jointproj{INSERM,UPSUD}

% Life Sciences Grid
\jointproj{INSERM,EGI}

% \jointpub{A,B} % some publication

%joint supervision
% \jointsup{A,B} %

%joint organization
% \jointorga{A,B} % some org
% \jointorga{SA,UJF} % PASCO'15

% joint publications
% \jointpub{A,B} % some publication

% Jupyter publication
\jointpub{SRL,XFEL}
\jointpub{SRL,QS}
\jointpub{XFEL,QS}

\coherencetable[swsites]

%%% Local Variables:
%%% mode: latex
%%% TeX-master: "proposal"
%%% End:

%%% Local Variables:
%%% mode: latex
%%% TeX-master: "proposal"
%%% End:
