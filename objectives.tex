\eucommentary{1-2 pages}
\eucommentary{\emph{Describe the specific objectives for the project,
which should be clear, measurable, realistic and achievable within the
duration of the project. Objectives should be consistent with the expected
exploitation and impact of the project (see section 2).}}

\TODO{Desirable keywords: sustainability, impact, reproducibility,
  interoperability, ...}

\noindent The aims and objectives of \TheProject are:
\begin{compactenum}[\textbf{Objective} 1:]
\item Develop an infrastructure for the EOSC hub and the wider open
  science community, that can be tailored and used to provide a
  multitude of specific services on the EOSC to support open science
  in a wide range scientific domains and projects. This infrastructure
  builds on the Jupyter project and ecosystem.

\item \label{aim:jupyter}
  Support and steer the Jupyter ecosystem, as a general purpose
  toolbox for better science through interactive computing and
  visualization that supports the entire life-cycle open science, from
  initial exploration to publication, research and development in
  industry, teaching, and outreach. In particular shape and develop
  the Jupyter ecosystem of tools further so that they can become key
  technology for the EOSC hub.

\item \label{aim:jupyter-reusability}%
  Extend the Jupyter project's capabilities to better support FAIR
  data requirements. In particular, the archival of execution
  environments to support re-usability of notebooks in the future
  needs attention. Such notebooks may be published together with
  publication manuscripts to detail the computation of published data
  and figures, to address the Re-usable in FAIR data.

\item \label{aim:jupyter-collaboration}%
  Extend the Jupyter project's capabilities to better support
  collaborative work, in particular enabling concurrent editing of a
  document.

\item \label{aim:jupyter-widgets}%
  Extend the feature set of Jupyter Notebooks in terms of Graphical
  User Interface (GUI) elements. The existing Notebook widgets
  demonstrate that such GUI-like interfaces inside the notebook are
  possible, but need more features and better integration to be widely
  applicable and support FAIR data science. Having a generic widget
  library and framework for extensions, will increase interoperability
  of data science between different projects.

  \TODO{Hans: Sylvain, Min, did we actually want to work on widgets/
    What do you think about the above? In terms of 'better
    integration', I was thinking of saving state of the widgets with
    the notebook, and start with the same values when the notebook is
    re-opened.}



\item \label{aim:demonstrators}%
  Demonstrate and evaluate the value and versatility of the design and
  the services building on it through applications to a number of
  domains in academic research, research infrastructures, SMEs and for
  the public sector, driven through our project partners. In
  particular, demonstrators for research infrastructure facilities and
  Photon Science (EuXFEL), SMEs (QuantStack, WildTreeTech), public
  sector (Paul), academic research (UPSud), \ldots.

  \TODO{Hans: Paul and all, please add your own domain above}

\item \label{aim:binderservice}%
  Design and provide a pilot MyBinder service on EOSC-Hub as a service
  that can be used immediately for EOSC users from any domain to
  support their reproducible open science and assess the importance of
  this service for the future.

\item \label{aim:outreach-and-engagement} Reach out to researchers
  outside this project to encourage engagement and exploitation of the
  EOSC-Hub and the Jupyter-based Open Science Services for their
  research domains.

  %Engaging a larger community will also make the
  %future maintenance and development of the infrastructure more sustainable.

\end{compactenum}

Objectives and impact will be measured using Key Performance
Indicators (KPI).

\TODO{Hans: Can we say something about sustainability? See commented
  source in last item.}
