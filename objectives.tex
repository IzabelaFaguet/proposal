\eucommentary{1-2 pages}
\eucommentary{\emph{Describe the specific objectives for the project,
which should be clear, measurable, realistic and achievable within the
duration of the project. Objectives should be consistent with the expected
exploitation and impact of the project (see section 2).}}

\TODO{Desirable keywords: sustainability, impact, reproducibility,
  interoperability, ...}

\noindent The aims and objectives of \TheProject are:
\begin{compactenum}[\textbf{Objective} 1:]

\item \label{aim:eosc}%
  Develop an infrastructure for the EOSC-hub and the wider open
  science community, that can be tailored and used to provide a
  multitude of specific services on the EOSC to support open science
  in a wide range of scientific domains and projects. This infrastructure
  builds on the Jupyter project and ecosystem.

\item \label{aim:jupyter}%
  Support and steer the Jupyter ecosystem, as a general purpose
  toolbox for better science through interactive computing and
  visualisation that supports the entire life-cycle of open science:
  from initial exploration to publication, research and development in
  industry, teaching, and outreach. In particular shape and develop
  the Jupyter ecosystem of tools further so that they can become key
  technology for the EOSC-hub.

\item \label{aim:jupyter-reusability}%
  Extend the Jupyter project's capabilities to better support FAIR
  data requirements. In particular, the archival of execution
  environments to support re-usability of notebooks in the future
  needs attention. Such notebooks may be published together with
  publication manuscripts to detail the computation of published data
  and figures, to address the Re-usable requirement of FAIR data.

\item \label{aim:jupyter-collaboration}%
  Extend the Jupyter project's capabilities to better support
  collaborative work, in particular enabling concurrent editing of a
  document.

\item \label{aim:jupyter-widgets}%
  Improve on the core Jupyter Widget libraries to better support advanced
  use cases such as the case of large widgets state saved in the notebook, or
  the use of Jupyter Interactive Widgets outside of Jupyter. This also includes
  improvements to the tooling built around interactive widgets such as the
  testing of widget packages.

\item \label{aim:demonstrators}%
  Demonstrate and evaluate the value and versatility of the design and
  the services building on it through applications to a number of
  domains in academic research, research infrastructures, SMEs and for
  the public sector, driven through our project partners. In
  particular, demonstrators for research infrastructure facilities and
  Photon Science (EuXFEL), SMEs (QuantStack, WildTreeTech), public
  sector (Paul), academic research (UPSud), \ldots.

  \TODO{Hans: Paul and all, please add your own domain above}

\item \label{aim:binderservice}%
  Design and provide a pilot MyBinder service on EOSC-Hub as a service
  that can be used immediately for EOSC-hub users from any domain to
  support their reproducible open science and assess the importance of
  this service for the future.

\item \label{aim:outreach-and-engagement}%
  Reach out to researchers outside this project to encourage engagement
  and exploitation of the EOSC-Hub and the Jupyter-based Open Science
  Services for their research domains.

  %Engaging a larger community will also make the
  %future maintenance and development of the infrastructure more sustainable.

\end{compactenum}

\TODO{Hans: Can we say something about sustainability? See commented
  source in last item.}
