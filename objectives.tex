\eucommentary{1-2 pages}
\eucommentary{\emph{Describe the specific objectives for the project,
which should be clear, measurable, realistic and achievable within the
duration of the project. Objectives should be consistent with the expected
exploitation and impact of the project (see section 2).}}

%  specific
Aim 1: 
Aim 2: Leverage this technology for all scientists, in Europe and beyond, through generic services that are tightly integrated into the EOSC,

Aim 3: Demonstrate the value and versatility of such services through applications to mathematics, medicine, (... what's done at XFEL), ...

Desirable keywords: sustainability, reproducibility, interoperability,

The aims of \TheProject are:
\begin{compactenum}[\textbf{Aim} 1:]
\item \label{aim:collaboration}
  Support (and steer?) the Jupyter ecosystem, as a general purpose toolbox for interactive computing and visualization that supports the entire life-cycle of open science, from initial exploration to publication, engineering (better word(s) about usage in industry here?), teaching, and outreach.
\item \label{aim:vre} Make it easy for teams of researchers of any
  size to set up custom, collaborative \emph{Virtual Research
    Environments} tailored to their specific needs, resources and
  workflows. The \VREs should support the entire life-cycle of
  computational work in mathematical research, from initial
  exploration to publication, teaching and outreach.
  % and bridge the gaps between
  % code, published results, and educational material.
\item \label{aim:sharing} Identify and promote best practices in
  computational mathematical research including: making results easily
  reproducible; producing reusable and easily accessible
  software; sharing data in a semantically sound way; exploiting and
  supporting the growing ecosystem of computational tools.
\item \label{aim:impact} Maximise sustainability and impact in
  mathematics, neighbouring fields, and scientific computing.
\end{compactenum}

