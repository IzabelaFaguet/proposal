\eucommentary{1-2 pages}
\eucommentary{\emph{Describe the specific objectives for the project,
which should be clear, measurable, realistic and achievable within the
duration of the project. Objectives should be consistent with the expected
exploitation and impact of the project (see section 2).
Desirable keywords: sustainability, impact, reproducibility,
interoperability, ...
}
}

\noindent The aims of \TheProject are:

\begin{compactenum}[\textbf{Aim} 1:]
\item Support and steer the Jupyter ecosystem, as a general purpose
  toolbox for interactive computing and visualization \textbf{that
    facilitates the entire life-cycle of Open Science}, from initial
  exploration to publication, research and development in industry,
  teaching, and outreach.

\item Leverage this technology for all scientists, in Europe and
  beyond, across domains, disciplines, and demographics, through the
  operation of free, public, and general purpose services that are
  tightly integrated into the European Open Science Cloud (EOSC).

\item Demonstrate the value and versatility of such services through
  innovative applications in a variety of disciplines and contexts.

\item Support open science through development and dissemination of
  best practices, training, and community building around the usage
  and development of the above tools.
\end{compactenum}
% develop and support the Jupyter ecosystem in a direction that benefits and facilitates open science
%     make these tools accessible to as many people as possible via operation of free, public services
%     demonstrate and ensure that these developments are useful to real scientists and the public
%     foster open science through training of students and researchers in best practices using these tools

  
% \item \label{aim:facilitation}
%   Facilitate Open Science through the development
%   of tools enabling reproducibility, sharing, and collaboration.

% \item \label{aim:accessibility}
%   Maximise accessibility and interoperability of Open Science services and tools,
%   across domains, disciplines, and demographics.

% \item \label{aim:sustainability}
%   Maximise sustainability of software tools for Open Science
%   by developing the community and contributing
%   to and supporting community-led software efforts.


  
%   Support open source software for open science, and notably the
%   Jupyter ecosystem, 

  
  
% \end{compactenum}

\noindent We will achieve our aims through the following objectives:

\begin{compactenum}[\textbf{Objective} 1:]

\item \label{obj:deployment} \TODO{potential title: Generic Jupyter infrastructure and services for the EOSC}%
  Contribute an infrastructure to the European Open Science Cloud
  (EOSC) and the wider Open Science community that can be tailored to
  provide a multitude of generic or specialized services to facilitate
  open science in a wide range of scientific domains and projects.
  This infrastructure will build on the Jupyter project and ecosystem,
  taking the form of a federation of JupyterHub/Binder instances,
  tightly integrated into the EOSC-Hub.

\item \label{obj:jupyter} \TODO{potential title: Supporting and steering the Jupyter ecosystem}%
  Support and steer the Jupyter ecosystem, as a general purpose
  toolbox for better science through interactive computing and
  visualisation that supports the entire life-cycle of open science:
  from initial exploration to publication, research and development in
  industry, teaching, and outreach. In particular, shape and develop
  the Jupyter ecosystem of tools further so that they can become key
  technology for the EOSC-hub.

\item \label{obj:interactivity} \TODO{potential title: Even more interactivity in Jupyter}%
  Improve the interactive capabilities of the Jupyter environment,
  through developments of interactive widgets,
  visualization tools, collaboration features, dashboards,
  and expanded support for kernels such as interactive C++.
  While Jupyter is already widely used, there are many areas
  of interactive exploration that can be improved upon.

\item \label{obj:reusability} \TODO{potential title: Reproducibility and FAIR data}%
  Extend the facilities for reproducibility of computational environments
  and facilitating FAIR data practices.
  We will contribute to the recording and reproducibility
  of environments with repo2docker and Binder
  and extend capabilities to better support FAIR
  data requirements. In particular, the archival of execution
  environments to support re-usability of notebooks in the future
  needs attention. Such notebooks may be published alongside
  publication manuscripts to detail the computation of published data
  and figures, to address the Re-usable requirement of FAIR data.

\item \label{obj:demonstrators} \TODO{potential title: Demonstrators in sciences}%
  Demonstrate and evaluate the  versatility and value of the design and
  the services building on it, through applications to a number of
  domains in academic research, education, research infrastructures, SMEs and for
  the public sector, driven through our project partners. In
  particular, we will contribute demonstrators in the following areas: research infrastructure facilities and
  Photon Science (\site{XFEL}), SMEs (\site{QS}, \site{WTT}),
  astronomy (\site{CDS}), life sciences (\site{INSERM}),
  geosciences (\site{UIO}), physics (\site{SIL}),
  and mathematics and education (\site{UPSUD}, \site{EP}).

\item \label{obj:outreach-and-engagement} \TODO{potential title: Outreach, engagement, and sustainability}%
  Reach out to scientists and the wider Open Science and Open Data
  communities to encourage engagement
  and exploitation of the EOSC-Hub and the Jupyter-based Open Science
  Services for their research domains and interests.
  Engaging a larger community will help ensure the sustainability of
  the services and underlying infrastructure by distributing its
  development, hosting, and maintenance over stakeholders from a
  variety of institutions.

\end{compactenum}

\noindent Progress toward these aims can be monitored via the following
Key Performance Indicators (KPIs):

\begin{compactenum}[\textbf{KPI} 1:]
  \item \label{kpi:workshop-attendees}
    Aim \ref{aim:accessibility}:
    Attendees at Open Science workshops organised by \TheProject participants.
  \item \label{kpi:binder-publications}
    Aim \ref{aim:facilitation}:
    Open publications available on \TheProject services.
  \item \label{kpi:binder-visits}
    Aim \ref{aim:accessibility}:
    Visitors to \TheProject services, engaging with open, interactive communications.
  \item \label{kpi:dissemination}
    Aim \ref{aim:facilitation}:
    Publications and presentations by \TheProject documenting the use of \TheProject services for Open Science.
  \item \label{kpi:contributions}
    Aim \ref{aim:sustainability}:
    Contributions by \TheProject and the wider community to Jupyter software and others.
\end{compactenum}


\TODO{table relating objectives and tasks/deliverables}
