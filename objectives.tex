\eucommentary{1-2 pages}
\eucommentary{\emph{Describe the specific objectives for the project,
which should be clear, measurable, realistic and achievable within the
duration of the project. Objectives should be consistent with the expected
exploitation and impact of the project (see section 2).}}

\TODO{Desirable keywords: sustainability, impact, reproducibility,
  interoperability, ...}


\TODO{
  Consider consolidating objectives
  to higher-level user/society-focused goals.
  Maybe reduce by 1-3.
}

\noindent The aims of \TheProject are:

\begin{compactenum}[\textbf{Aim} 1:]

\item \label{aim:facilitation}
  Facilitate Open Science through the development
  of tools enabling reproducibility, sharing, and collaboration.

\item \label{aim:accessibility}
  Maximise accessibility of Open Science services and tools,
  across domains, disciplines, and demographics.

\item \label{aim:sustainability}
  Maximise sustainability of software for Open Science
  by developing the community and contributing
  to and supporting community-led software efforts.

\end{compactenum}

\noindent We will achieve our aims through the following objectives:

\begin{compactenum}[\textbf{Objective} 1:]

\item \label{obj:deployment}
  Develop an infrastructure for the EOSC-Hub and the wider open
  science community, that can be tailored and used to provide a
  multitude of specific services on the EOSC to support open science
  in a wide range of scientific domains and projects. This infrastructure
  builds on the Jupyter project and ecosystem.
  We will operate such a service in the form of JupyterHub and Binder,
  to be accessible via EOSC-Hub.

\item \label{obj:interactivity}
  Improve the interactive capabilities of Jupyter software,
  through developments of interactive widgets,
  visualization tools, collaboration features,
  and expanded support for kernels such as interactive C++.
  While Jupyter is already widely used, there are many areas
  of interactive exploration that can be improved upon.

\item \label{obj:jupyter}
  Support and steer the Jupyter ecosystem, as a general purpose
  toolbox for better science through interactive computing and
  visualisation that supports the entire life-cycle of open science:
  from initial exploration to publication, research and development in
  industry, teaching, and outreach. In particular shape and develop
  the Jupyter ecosystem of tools further so that they can become key
  technology for the EOSC-hub.

\item \label{obj:reusability}
  Extend the facilities for reproducibility of computational environments
  and facilitating FAIR data practices.
  We will contribute to the recording and reproducibility
  of environments with repo2docker and Binder
  and extend capabilities to better support FAIR
  data requirements. In particular, the archival of execution
  environments to support re-usability of notebooks in the future
  needs attention. Such notebooks may be published together with
  publication manuscripts to detail the computation of published data
  and figures, to address the Re-usable requirement of FAIR data.

\item \label{obj:demonstrators}
  Demonstrate and evaluate the value and versatility of the design and
  the services building on it through applications to a number of
  domains in academic research, education, research infrastructures, SMEs and for
  the public sector, driven through our project partners. In
  particular, demonstrators for research infrastructure facilities and
  Photon Science (\site{XFEL}), SMEs (\site{QS}, \site{WTT}),
  astronomy (\site{CDS}), life sciences (\site{INSERM}),
  geosciences (\site{UIO}), physics (\site{SIL}),
  and math and education (\site{UPSUD}, \site{EP}).

\item \label{obj:outreach-and-engagement}
  Reach out to researchers and the wider Open Science and Open Data
  communities outside this project to encourage engagement
  and exploitation of the EOSC-Hub and the Jupyter-based Open Science
  Services for their research domains and interests.
  Engaging a larger community will help ensure the
  future maintenance and development of the infrastructure more sustainable
  by integrating stakeholders from a variety of institutions.

\end{compactenum}

\noindent Progress toward these aims can be monitored via the following
Key Performance Indicators (KPIs):

\begin{compactenum}[\textbf{KPI} 1:]
  \item \label{kpi:workshop-attendees}
    Aim \ref{aim:accessibility}:
    Attendees at Open Science workshops organised by \TheProject participants.
  \item \label{kpi:binder-publications}
    Aim \ref{aim:facilitation}:
    Open publications available on \TheProject services.
  \item \label{kpi:binder-visits}
    Aim \ref{aim:accessibility}:
    Visitors to \TheProject services, engaging with open, interactive communications.
  \item \label{kpi:dissemination}
    Aim \ref{aim:facilitation}:
    Publications and presentations by \TheProject documenting the use of \TheProject services for Open Science.
  \item \label{kpi:contributions}
    Aim \ref{aim:sustainability}:
    Contributions by \TheProject and the wider community to Jupyter software and others.
\end{compactenum}


\TODO{table relating objectives and tasks/deliverables}
