\subsection{Context and motivation}

In many scientific disciplines, it is common for researchers to rely on
heterogeneous computational tools and technologies to collect data,
explore the input data sets, run simulations, visualise the outcome,
and share their result with peers or a with a larger audience. Often,
such data analysis cycles are iteratively refined.

For simple data sets, processes may remain manageable. However, when
dealing with larger and more complex use cases, including big data
from research facilities or High Performance Computing resources, the
complexity makes iteration cycles slower for the researchers. A
complex iteration cycle also makes research results more difficult to
reproduce. Results that are reproducible, can much more easily be
re-used in future work.

This situation is exacerbated by the expected increase of the amount
of scientific data being available, including the data becoming
accessible through the EOSC-Hub.


%%HF: the following seemed to be to specialised to list in the opening
%%pararaphs?
%
%which
%is especially harmful in scientific software engineering where most innovation
%is achieved through \emph{incrementalism}.

\subsection{Project Jupyter}

\emph{Project Jupyter} \cite{Jupyter}, which has grown increasingly popular in the scientific
computing community, has become the \emph{lingua franca} of interactive
computing in both academia and industry. The main goal of Project Jupyter
project is to provide a consistent set of tools to improve researchers'
workflows from the exploratory phase of the analysis to the communication
of the results \cite{Kluyver2016}.

Started in 2014 from the \emph{IPython Project} \cite{IPython}, Jupyter has grown in
popularity and adoption both in the industry and academia. We estimate the user
base of the Jupyter notebook to be of several millions. Users range from data
scientists to researchers, educators and students from many fields,
including journalists and librarians. In 2017, the Jupyter
team was awarded the \emph{ACM Software System Award}, an annual award that
honors people or an organization "for developing a software system that had a
lasting influence". Prior recipients include \emph{Unix}, \emph{TCP/IP}, and
the \emph{World Wide Web}.

\subsection{This proposal}

In this proposal, core team developers of the project, including a
number of recipients of the ACM award, and key contributors to the
open source scientific computing ecosystem detail improvements to the
capabilities of Project Jupyter.  The goal is to improve the
accessibility of EOSC resources to researchers and the general public,
and improve the accessibility, interactivity, reproducibility and
re-usability of computational research.

\subsubsection{Proposed improvements concern core components of Jupyter (WP2)}

\begin{itemize}
\item The \emph{Notebook} is a development environment in the form of an
interactive document including executable code, rich text, interactive data
visualization and GUI components.
\item The \emph{Binder Project} is a tool allowing the execution of such
notebooks within a tailored computational environment. It is key to reproducibility
of research results as it makes it possible for
users use these environment without installing anything locally.
Improvements will concern packages underlying the binder infrastructure such
as \emph{repo2docker}, \emph{BinderHub}, and \emph{JupyterHub}.
\item The \emph{Jupyter Interactive Widgets}, which are used to include GUI
components and interactive visualisations in notebooks. This includes 2d
and 3d interactive plotting, geographic data visualization and much more.
\end{itemize}

\subsubsection{Proposed improvements of the Jupyter ecosystem (WP3)}

We further propose improvements of the wider Jupyter ecosystem for
better scientific workflows, including
\begin{itemize}
\item The \emph{ipyvolume} and \emph{k3d} packages, which enable interactive
3-D visualization in the Jupyter ecosystem, including the EOSC-Hub.
\item The \emph{Voila} package, built upon Jupyter components, which enable the
sharing of notebook-based interactive dashboards for non-technical users.
\item The \emph{Xeus} kernel instrastructure, and more specifically,
  the \emph{xeus-cling} C++ kernel, built upon CERN's C++ interpreter,
  "cling", which has a lot of adoption in the High-Energy-Physics
  community. The xeus-based C++ kernel is already in use for the
  teaching of the C++ programming language.
\item The \emph{JupyterArchive} which will store a notebook together
  with its execution environment to allow re-execution of the notebook
  in the future.
\end{itemize}

\subsubsection{Beyond the improvement of the Jupyter Project (WP4,
  WP5, WP6)}
Beyond the improvement to the software stack, we plan on
\begin{itemize}
\item Application, demonstration and evaluation of such novel services
  in multiple demonstrators, that cover research fields such as
  health, astrophysics, photon and neutron science, geosciences and
  mathematics, and also interests of participating SMEs (WP4)
\item Operating a \emph{European Binder Service} on the EOSC-hub and
  enabling provision of Jupyter Services through the EOSC-sub (WP5).
\item Producing \emph{training and education material} to disseminate
  the ability to do reproducible computational science using the tools
  we develop, among others. (WP6)
\end{itemize}

\TODO{HF: This is a nice summary. Is that good here in the excellence
  section? Or should we have a separate 'execute summary' before the
  main document starts, that shows the above?}

\clearpage

%%% Local Variables:
%%% mode: latex
%%% TeX-master: "proposal"
%%% End:
