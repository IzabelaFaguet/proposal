In most applied scientific disciplines, researchers often rely on
heterogeneous tools and technologies to collect data, run simulations,
visualize the outcome, explore the input data sets, and share their
result with peers or a with a larger audience.

For simple data sets, processes remain manageable, but when dealing
with larger and more complex use cases, or with super-computing resources,
the complexity makes iteration cycles slower for the the researchers, which
is especially harmful in scientific software engineering where most innovation
is achieved through \emph{incrementalism}. It also makes research results more
difficult to reproduce.

\emph{Project Jupyter}, which has grown increasingly popular in the scientific
computing community, has become the \emph{lingua franca} of interactive
computing in both academia and industry. The main goal of Project Jupyter
project is to provide a consistent set of tools to improve researchers'
workflows from the exploratory phase of the analysis to the communication
of the results.

Started in 2014 from the \emph{IPython Project}, Jupyter has grown in
popularity and adoption both in the industry and academia. We estimate the user
base of the Jupyter notebook to be of several millions. Users range from data
scientists to educators and students, or even journalists. In 2017, the Jupyter
team was awarded the \emph{ACM Software System Award}, an annual award that
honors people or an organization "for developing a software system that had a
lasting influence". Prior recipients include \emph{Unix}, \emph{TCP/IP}, and
the \emph{World Wide Web}.

In this proposal, core team developers of the project, including a number of
recipients of the ACM award, and key contributors to the open source scientific
computing ecosystem detail improvements to the capabilities of Project Jupyter.
The goal is to improve the accessibility of EOSC resources to researchers and
the general public, and improve the accessibility, interactivity, and
reproducibility of computational research.

Proposed improvements concern core components of Jupyter:

\begin{itemize}
\item The \emph{Notebook} is a development environment in the form of an
interactive document including executable code, rich text, interactive data
visualization and GUI components.
\item The \emph{Binder Project} is a tool allowing the execution of such
notebooks with tailored computational environment. It is key to reproducibility
of research results and often not well addressed as it makes it possible for
users to make use of these environment without installing anything locally.
Improvements will concern packages underlying the binder infrastructure such
as \emph{repo2docker}, \emph{BinderHub}, and \emph{JupyterHub}.
\item The \emph{Jupyter Interactive Widgets}, which are used to include GUI
components and interactive visualizations in notebooks. This includes with 2-D
and 3-D interactive plotting, geographic data visualization and much more.
\end{itemize}

As well as downstream packaging making use of Jupyter for better scientific
workflows:

\begin{itemize}
\item The \emph{Voila} package, built upon Jupyter components, which enable the
sharing of notebook-based interactive dashboards for non-technical users.
\item The \emph{Xeus} kernel instrastructure, and more specifically, the
\emph{xeus-cling} C++ kernel, built upon CERN's C++ interpreter, "cling",
which has a lot of adoption in the High-Energy-Physics community. The
xeus-based C++ kernel is already in use for the teaching of the C++ programming
language.
\item The \emph{ipyvolume} and \emph{k3d} packages, which enable interactive
3-D visualization in the Jupyter ecosystem, including the EOSC.
\end{itemize}

Beyond the improvement to the software stack, we plan on 

\begin{itemize}
\item Operating a \emph{European Binder Service} with the goal of running on EOSC.
\item Producing \emph{training and education material} in the ability to do
reproducible computational science using the tools we develop, among others.
\end{itemize}

\clearpage

%%% Local Variables:
%%% mode: latex
%%% TeX-master: "proposal"
%%% End:
